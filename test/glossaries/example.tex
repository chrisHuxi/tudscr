\documentclass[english,ngerman]{tudscrreprt}
\usepackage[T1]{fontenc}
\ifpdftex{\usepackage[ngerman=ngerman-x-latest]{hyphsubst}}{}
\usepackage{babel}

\usepackage{xfrac}

\usepackage{tudscr-glossaries}
\usepackage[colorlinks,linkcolor=blue]{hyperref}
\usepackage[%
  acronym,% Abkürzungen
  symbols,% Formelzeichen
  nomain,% kein Glossar
  nogroupskip,%
  toc,%
  section=chapter,% 
  nostyles,%
  translate=babel,%
% mit Tex Live einfach verwendbar
  xindy={language=german-din},
]{glossaries}
\makeglossaries

\defcaptionname{ngerman}{\acronymname}{Abkürzungen}

\begin{document}


\printacronyms[style=acrotabu]
\printsymbols[style=symbsplitlongtabu]

\clearpage

\section*{Die Verwendung von Akronymen und Symbolen}
\newacronym{apsp}{APSP}{All-Pairs Shortest Path}
\newacronym{spsp}{SPSP}{Single-Pair Shortest Path}
\newacronym{sssp}{SSSP}{Single-Source Shortest Path}

In der Graphentheorie wird häufig die Lösung des Problems des kürzesten
Pfades zwischen zwei Knoten gesucht. Dieses Problem wird häufig auch
mit \gls{spsp} bezeichnet. Es lässt sich auf die Variationen \gls{sssp}
und \gls{apsp} erweitern. Für die Lösung von \gls{spsp}, \gls{sssp}
oder \gls{apsp} kommen unterschiedliche Algorithmen zum Einsatz.

\newformulasymbol{l}{Länge}{l}{m}
\newformulasymbol{m}{Masse}{m}{kg}
\newformulasymbol{a}{Beschleunigung}{a}{\sfrac{m}{s^2}}
\newformulasymbol{t}{Zeit}{t}{s}
\newformulasymbol{f}{Frequenz}{f}{s^{-1}}
\newformulasymbol{F}{Kraft}{F}{m \cdot kg \cdot s^{-2} = \sfrac{J}{m}}

\newformulasymbol{omega}{Winkelgeschwindigkeit}{\omega}{\sfrac{1}{s}}

Die Einheiten für die \gls{f} sowie die \gls{F} werden aus den
SI"=Einheiten der Basisgrößen \gls{l}, \gls{m} und \gls{t} abgeleitet.
Und dann gibt es noch die Grundgleichung der Mechanik, welche für den
Fall einer konstanten Kraftwirkung in die Bewegungsrichtung einer
Punktmasse lautet:
\[\gls{F} = \gls{m} \cdot \gls{a}\]

Und noch ein Test für griechische Buchstaben: \gls{omega}
\end{document}
