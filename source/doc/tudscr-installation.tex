\setchapterpreamble{\tudhyperdef'{sec:install:ext}}
\chapter{Weiterführende Installationshinweise}
%
\noindent\Attention{%
  Hier werden unterschiedliche Varianten erläutert, wie \TUDScript in der 
  Version~\vTUDScript{} genutzt werden kann, falls eine frühere Variante als 
  \textbf{lokale Nutzerinstallation} verwendet wurde.
}
\ToDo[rls]{Funktionalität der Installationsskripte testen}

\bigskip\noindent
Bis zur Version~v2.01 wurde \TUDScript ausschließlich über das \Forum zur 
lokalen Nutzerinstallation angeboten. In erster Linie hat das historische 
Hintergründe und hängt mit der Entstehungsgeschichte von \TUDScript zusammen. 
Eine lokale Nutzerinstallation bietet einen~-- eher zu vernachlässigenden~-- 
Vorteil. Treten bei der Verwendung von \TUDScript Probleme auf, können diese im 
Forum gemeldet und diskutiert werden. Ist für ein solches Problem tatsächlich 
eine Fehlerkorrektur respektive Aktualisierung von \TUDScript nötig, kann diese 
schnell und unkompliziert über das \GitHubRepo bereitgestellt und durch den 
Anwender sofort genutzt werden.

Dies hat allerdings für alle Anwender, welche das Forum relativ wenig oder gar 
nicht besuchen, den großen Nachteil, dass Sie nicht von Aktualisierungen, 
Verbesserungen und Fehlerkorrekturen neuer Versionen profitieren können. Auch 
alle nachfolgenden Bugfixes und Aktualisierungen des \TUDScript-Bundles müssen 
durch den Anwender manuell durchgeführt werden. Daher wird die Verbreitung via 
\CTAN[pkg/tudscr] präferiert, sodass \TUDScript stets in der gerade aktuellen 
Version verfügbar ist~-- eine durch den Anwender aktuell gehaltene 
\hologo{LaTeX}"=Distribution vorausgesetzt. Der einzige Nachteil bei diesem 
Ansatz ist, dass die Verbreitung eines Bugfixes und die anschließende 
Bereitstellung durch die verwendete Distribution für gewöhnlich bis zu zwei 
Tagen dauert.

Die gängigen \hologo{LaTeX}"=Distributionen durchsuchen im Regelfall zuerst das 
lokale \Path{texmf}"=Nutzerverzeichnis nach Klassen und Paketen und erst daran 
anschließend den \Path{texmf}"=Pfad der Distribution selbst. Dabei spielt es 
keine Rolle, in welchem Pfad die neuere Version einer Klasse oder eines Paketes 
liegt. Sobald im Nutzerverzeichnis die gesuchte Datei gefunden wurde, wird die 
Suche beendet.
\Attention{%
  In der Konsequenz bedeutet dies, dass sämtliche Aktualisierungen über CTAN 
  nicht zum Tragen kommen, falls \TUDScript als lokale Nutzerversion 
  installiert wurde.
}

Deshalb wird Anwendern empfohlen, eine gegebenenfalls vorhandene lokale 
Nutzerinstallation von \TUDScript zu deinstallieren, falls diese nicht 
\emph{bewusst} installiert wurde. Die Deinstallation wird in 
\autoref{sec:local:uninstall} erläutert. Nach dieser können Updates des 
\TUDScript-Bundles durch die Aktualisierungsfunktion der Distribution erfolgen. 

Wie das \TUDScript-Bundle trotzdem als lokale Nutzerversion installiert oder 
aktualisiert werden kann, ist in \autoref{sec:local:install} beziehungsweise 
\autoref{sec:local:update} zu finden. Der Anwender sollte in diesem Fall 
allerdings genau wissen, was er damit bezweckt, da er in diesem Fall für die 
Aktualisierung von \TUDScript selbst verantwortlich ist.

\minisec{Nutzung der veralteten Schriftfamilien}
Soll ein Dokument noch mit den veralteten Schriftfamilien \Univers und \DIN 
gesetzt werden, so ist eine lokale Installation der PostScript"=Schriften 
notwendig, welche in \autoref{sec:install:fonts} beschrieben wird. Zusätzlich 
sei auf die Optionen \Option*{tudscrver=v2.05} und \Option*{cdoldfont} 
hingewiesen. Für die Verwendung der \OpenSans wird lediglich die aktuelle 
\ToDo[doc]{Name des Paketes für \OpenSans ergänzen/korrigieren}[v2.06]  
Version des Paketes \Package{opensans} benötigt, welches auf 
\CTAN[pkg/opensans] zu finden ist.


\section{Lokale Deinstallation des \TUDScript-Bundles}
\tudhyperdef*{sec:local:uninstall}%
\index{Deinstallation}%
%
Über die Kommandozeile beziehungsweise das Terminal kann mit
%
\begin{quoting}
\Path{kpsewhich --all tudscrbase.sty}
\end{quoting}
%
überprüft werden, ob eine lokale Nutzerinstallation von \TUDScript vorhanden 
ist. Es werden alle Pfade ausgegeben, in denen die Datei \File*{tudscrbase.sty} 
gefunden wird. Erscheint nur der Distributionspfad, ist die \TUDScript-Version 
von CTAN aktiv und der Anwender kann mit dem \TUDScript-Bundle arbeiten.

Wird \emph{nur} das lokale Nutzerverzeichnis oder gar kein Verzeichnis 
gefunden, so wird höchstwahrscheinlich eine veraltete Distribution 
verwendet. In diesem Fall wird eine Aktualisierung dieser \emph{unbedingt} 
empfohlen. Sollte dies nicht möglich sein, \emph{muss} \TUDScript als lokale 
Nutzerversion installiert (\autoref{sec:local:install}) beziehungsweise~-- 
falls ein Pfad ausgegeben wurde~-- aktualisiert (\autoref{sec:local:update}) 
werden.

Sollte neben dem Pfad der Distribution noch mindestens ein weiterer Pfad 
angezeigt werden, so ist eine lokale Nutzerversion installiert. In 
diesem Fall hat der Anwender drei Möglichkeiten:
%
\begin{enumerate}
\item Entfernen der lokalen Nutzerinstallation (skriptbasiert)
\item Entfernen der lokalen Nutzerinstallation (manuell)
\item Aktualisierung der lokalen Nutzerversion (\autoref{sec:local:update})
\end{enumerate}
%
Um die lokale Nutzerinstallation zu entfernen, kann für Windows
\hrfn{\Download{uninstall/tudscr\_uninstall.bat}}{\File{tudscr\_uninstall.bat}}
sowie für unixartige Betriebssysteme
\hrfn{\Download{uninstall/tudscr\_uninstall.sh}}{\File{tudscr\_uninstall.sh}}
verwendet werden. Nach der Ausführung des jeweiligen Skriptes kann mit dem zu 
Beginn gezeigten Aufruf in der Kommandozeile respektive Terminal geprüft 
werden, ob die Deinstallation erfolgreich war. Wird immer noch mindestens ein 
lokaler Pfad ausgegeben, sollte \TUDScript manuelle deinstalliert werden, was 
nachfolgend beschrieben wird.

Nur die Deinstallation aller lokalen Nutzerinstallationen von \TUDScript 
ermöglicht die Verwendung der jeweils aktuellen CTAN"=Version. Hierfür ist~-- 
unter der Annahme, dass das automatisierte Deinstallieren mithilfe der zuvor 
genannten Skripte zur Deinstallation nicht erfolgreich war~-- etwas Handarbeit 
durch den Anwender vonnöten. Der in der Kommandozeile respektive im Terminal mit
%
\begin{quoting}
\Path{kpsewhich --all tudscrbase.sty}
\end{quoting}
%
gefundene~-- zum Ordner der Distribution \emph{zusätzliche}~-- Pfad hat die 
folgende Struktur:
%
\begin{quoting}
\Path{\PName{Installationspfad}/tex/latex/tudscr/tudscrbase.sty}
\end{quoting}
%
Um die Nutzerinstallation vollständig zu entfernen, muss als erstes zu 
\Path{\PName{Installationspfad}} navigiert werden. Anschließend ist in diesem 
Pfad Folgendes durchzuführen:
%
\settowidth\tempdim{\Path{tex/latex/tudscr/}}%
\begin{description}[labelwidth=\tempdim,labelsep=1em]
\item[\Path{tex/latex/tudscr/}]\File*{.cls}- und \File*{.sty}"~Dateien löschen
\item[\Path{tex/latex/tudscr/}]Ordner \Path{logo} vollständig löschen
\item[\Path{doc/latex/}] Ordner \Path{tudscr} vollständig löschen
\item[\Path{source/latex/}] Ordner \Path{tudscr} vollständig löschen
\end{description}
%
Zum Abschluss ist in der Kommandozeile beziehungsweise im Terminal der Befehl 
\Path{texhash} aufzurufen. Damit wurde die lokale Nutzerversion entfernt und es 
wird von nun an die Version von \TUDScript genutzt, welche durch die verwendete 
Distribution bereitgestellt wird.



\section{Lokale Installation des \TUDScript-Bundles}
\tudhyperdef*{sec:local:install}%
\index{Installation!Nutzerinstallation|(}%
%
Eine lokale Nutzerinstallation von \TUDScript sollte nur durch Anwender 
ausgeführt werden, die genau wissen, aus welchen Gründen dies geschehen soll.
Hierfür werden für Windows sowie unixartige Betriebssysteme die passenden 
Skripte angeboten. Alternativ dazu kann auch einfach der Inhalt des Archivs
\hrfn{\Download{\vTUDScript/tudscr\_\vTUDScript.zip}}%
{\File{tudscr\_\vTUDScript.zip}}
in das lokale \Path{texmf}"=Nutzerverzeichnis kopiert und abschließend in der 
Kommandozeile respektive im Terminal \Path{texhash} aufgerufen werden. 

Für eine lokale Nutzerinstallation des \TUDScript-Bundles unter Windows für die 
Distributionen \Distribution{\hologo{TeX}~Live} oder 
\Distribution{\hologo{MiKTeX}} werden die Dateien aus
\hrfn{\Download{\vTUDScript/TUD-Script\_\vTUDScript\_Windows\_full.zip}}%
{\File*{TUD-Script\_\vTUDScript\_Windows\_full.zip}}
benötigt. Für unixoide Betriebssysteme sind es die Dateien aus 
\hrfn{\Download{\vTUDScript/TUD-Script\_\vTUDScript\_Unix\_full.zip}}%
{\File*{TUD-Script\_\vTUDScript\_Unix\_full.zip}}.

Beim Ausführen des spezifischen Installationsskripts~-- für Windows 
\File{tudscr\_\vTUDScript\_install.bat} respektive für unixoide Betriebssysteme
\File{tudscr\_\vTUDScript\_install.sh}~-- werden alle Dateien in das lokale 
Nutzerverzeichnis der jeweiligen Distribution installiert, falls kein anderes 
Verzeichnis explizit angegeben wird. 
\Attention{%
  In jedem Fall sollte vor der Ausführung des Installationsskripts ein Update 
  der \hologo{LaTeX}"=Distribution durchgeführt werden. Dies gilt insbesondere 
  bei der Nutzung von \Distribution{\hologo{MiKTeX}}. Andernfalls wird es unter 
  Umständen im Installationsprozess oder bei der Nutzung von \TUDScript zu 
  Problemen kommen.
}



\section{Lokales Update des \TUDScript-Bundles}
\tudhyperdef*{sec:local:update}%
\index{Update|!}%
%
Update und lokale Installation unterscheiden sich vom prinzipiellen Vorgehen 
nicht. Die Archive für ein lokales Update verzichten zur Verringerung der 
Dateigröße lediglich auf die Logos für \TUDScript. Diese sollten  bereits durch 
eine frühere Version installiert worden sein.

\subsection{Update des \TUDScript-Bundles ab Version~\NoCaseChange{v}2.02}
Soll eine lokale Aktualisierung des \TUDScript-Bundles auf \vTUDScript{} 
erfolgen, so muss~-- abhängig vom verwendeten Betriebssystem~-- das Archiv
\hrfn{\Download{\vTUDScript/TUD-Script\_\vTUDScript\_Windows\_update.zip}}%
{\File*{TUD-Script\_\vTUDScript\_Windows\_update.zip}} respektive 
\hrfn{\Download{\vTUDScript/TUD-Script\_\vTUDScript\_Unix\_update.zip}}%
{\File*{TUD-Script\_\vTUDScript\_Unix\_update.zip}}
entpackt und daran anschließend entweder \File{tudscr\_\vTUDScript\_update.bat} 
oder \File{tudscr\_\vTUDScript\_update.sh} ausgeführt werden.
\Attention{%
  Die lokale Aktualisierung funktioniert nur, wenn \TUDScript bereits 
  mindestens in der Version~v2.02 entweder als lokale Nutzerversion oder über 
  die Distribution installiert ist.%
}

\subsection{Update des \TUDScript-Bundles ab Version~\NoCaseChange{v}2.00}
%
Mit der Version~v2.02 gab es einige tiefgreifende Änderungen. Deshalb wird für 
vorausgehende Versionen~-- sprich v2.00 und v2.01~-- kein dediziertes Update 
angeboten. Die Aktualisierung kann durch den Anwender entweder~-- wie in 
\autoref{sec:local:install} erläutert~-- mit einer skriptbasierten oder mit 
einer manuellen Neuinstallation erfolgen.
\index{Installation!Nutzerinstallation|)}%


\subsection{Update des \TUDScript-Bundles von Version \NoCaseChange{v}1.0}
%
Ist \TUDScript in der veralteten \emph{Version~v1.0} installiert, so wird vor 
der Aktualisierung dringlichst zu einem vollständigen Entfernen dieser Version 
geraten. Andernfalls werden nach einem Update bei der Verwendung massive 
Probleme und Fehler auftreten. Zur Deinstallation werden die Skripte 
\hrfn{\Download{uninstall/tudscr_uninstall.bat}}{\File{tudscr\_uninstall.bat}}
respektive
\hrfn{\Download{uninstall/tudscr_uninstall.sh}}{\File{tudscr\_uninstall.sh}}
bereitgestellt. Die aktuelle Version~\vTUDScript{} kann nach der vollständigen 
Deinstallation aller veralteten Versionen wie in \autoref{sec:install:fonts} 
beschrieben installiert werden.

Im Vergleich zur \emph{Version~v1.0} hat sich an der Benutzerschnittstelle 
nicht sehr viel verändert, ein Umstieg auf die Version~\vTUDScript{} dürfte 
keine Schwierigkeiten bereiten. Treten danach dennoch Probleme auf, sollte der 
Anwender als erstes die Beschreibung des Paketes \Package{tudscrcomp}'full' 
lesen, welches eine Schnittstelle zur Nutzung alter und ursprünglich nicht mehr 
vorgesehener Befehle sowie Optionen bereitstellt. Allerdings werden einige von 
diesen auch durch das Paket \Package{tudscrcomp} nicht mehr bereitgestellt. 
Aufgeführt sind diese in \autoref{sec:obsolete}. Sollten trotz aller Hinweise 
dennoch Fehler oder Probleme beim Umstieg auf die neue \TUDScript-Version 
auftreten, ist eine Meldung im \Forum oder im \GitHubRepo[issues] die beste 
Möglichkeit, um Hilfe zu erhalten.



\section{Installation veralteter Schriftfamilien}
\tudhyperdef*{sec:install:fonts}%
\index{Installation!Schriftinstallation|!(}%

Bis Anfang des Jahres~2018 nutzte das \TUDCD als Hausschriften nicht \OpenSans 
sondern die Schriftfamilien \Univers und \DIN. Diese können auch weiterhin mit 
\TUDScript verwendet werden, um alte Dokumente kompilieren zu können. Hierfür 
sei auf die Optionen \Option*{tudscrver=v2.05} und \Option*{cdoldfont} 
hingewiesen. Da es sich bei diesen um lizenzierte Schriften handelt, müssen 
diese beim Universitätsmarketing auf 
\hrfn{https://tu-dresden.de/cd/1_basiselemente/03_hausschrift/}{Anfrage} mit 
dem Hinweis auf die Verwendung von \hologo{LaTeX} bestellt und nach Erhalt der 
notwendigen Archive \File{Univers\_PS.zip} und \File{DIN\_Bd\_PS.zip} für 
Windows (\autoref{sec:install:win}) beziehungsweise unixoide Betriebssysteme 
(\autoref{sec:install:unix}) installiert werden. 

Das \TUDScript-Bundle unterstützt die Nutzung besagter Schriften auch im 
OpenType"=Format, welche ebenfalls über das Universitätsmarketing auf 
\href{https://tu-dresden.de/cd/1_basiselemente/03_hausschrift/}{Anfrage}
bestellt werden müssen. Die in den beiden Archiven \File{Univers\_8\_TTF\/.zip} 
und \File{DIN\_TTF\/.zip} enthaltenen Schriften lassen sich~-- sobald diese für 
das Betriebssystem installiert wurden~-- mit dem Paket \Package{fontspec} 
verwenden. Hierfür sollten die Hinweise in \autoref{sec:fonts:fontspec} 
beachtet werden.

Im \GitHubRepo* sind die zur Schriftinstallation benötigten 
\hrfn{https://github.com/tud-cd/tudscr/releases/tag/fonts}{Skripte}
ebenso zu finden wie die
\hrfn{https://github.com/tud-cd/tudscr/releases/tag/oldfonts}{Skripte}
für die Klassen von Klaus Bergmann. Die unterschiedlichen Installationsskripte 
begründen sich insbesondere dadurch, dass bei der Installation für das 
\TUDScript-Bundle sowohl die Metriken als auch das Kerning der Schriften für 
Fließtext und den Mathematikmodus angepasst werden. Sollen die so verbesserten 
Schriften für die Klassen von Klaus Bergmann verwendet werden, kann dies mit 
dem Paket \Package{fix-tudscrfonts} erfolgen, was allerdings das Ergebnis der 
erzeugten Ausgabe beeinflusst, weshalb die Installationsskripte in 
unterschiedlichen Varianten weiterhin vorgehalten werden.

\minisec{Erforderliche Pakete bei der Schriftinstallation}
Für die Installation der Schriften sind die Pakete \Package*{fontinst}, 
\Package*{cmbright}, \Package*{hfbright}, \Package*{cm-super} und 
\Package*{iwona} von \emph{essentieller} Bedeutung und daher \emph{zwingend} 
notwendig. Das Vorhandensein dieser wird durch die jeweiligen 
Schriftinstallationsskripte geprüft und die Installation beim Fehlen eines oder 
mehrerer Pakete mit einer entsprechenden Warnung abgebrochen.

\minisec{Anmerkung zu \NoCaseChange{\hologo{MiKTeX}}}
Vor der Installation der Schriften für \TUDScript sollte unbedingt ein Update 
von \Distribution{\hologo{MiKTeX}}|?| durchgeführt werden. Außerdem ist es sehr 
ratsam, die Installation von \Distribution{\hologo{MiKTeX}}|?| in der 
Mehrbenutzervariante mit Administratorrechten durchzuführen, da die 
Einzelbenutzervariante relativ unregelmäßig und nicht immer nachvollziehbar zu 
Problemen führen kann. 

Möglicherweise sind einige der für den Schriftinstallationsprozess notwendigen 
Pakete noch nicht installiert. Ist die automatische Nachinstallation fehlender 
Pakete aktiviert, so reicht es im Normalfall das Installationsskript zu 
starten. Andernfalls müssen diese Pakete manuell durch den Benutzer über den 
\Distribution{\hologo{MiKTeX}}"=Paketmanager hinzugefügt werden.

Das Installationsskript scheitert außerdem bei einigen Anwendern~-- aufgrund 
eingeschränkter Nutzerzugriffsrechte~-- beim Eintragen der Schriften in die 
Map"~Datei. Dies muss gegebenenfalls durch den Anwender über die Kommandozeile 
%
\begin{quoting}
\Path{initexmf -{}-edit-config-file updmap}
\end{quoting}
%
erfolgen. In der sich öffnenden Datei sollte sich der Eintrag 
\Path{Map~tudscr.map} befinden. Ist dies nicht der Fall, muss diese Zeile 
manuell eingetragen und die Datei anschließend gespeichert werden. Danach muss 
der Nutzer in der Kommandozeile noch folgenden Aufruf ausführen:
%
\begin{quoting}
 \Path{initexmf~-{}-mkmaps}
\end{quoting}


\minisec{Anmerkung zu 
  \NoCaseChange{\hologo{TeX}}~Live und Mac\NoCaseChange{\hologo{TeX}}%
}
Sollte keine Vollinstallation von \Distribution{\hologo{TeX}~Live} durchgeführt 
worden sein, müssen die erforderlichen Pakete zur Schriftinstallation manuell 
über den \Distribution{\hologo{TeX}~Live}"=Paketmanager hinzugefügt werden.

Sind nach einem fehlerfreien Durchlauf des Installationsskriptes die Schriften 
dennoch nicht verfügbar, so muss die Synchronisierung aller Schriftdateien 
angestoßen werden. Daran anschließend müssen die Map"~Datei und die 
dazugehörigen Schriftdateien registriert werden. Die hierfür notwendigen 
Aufrufe lauten:
%
\begin{quoting}
\Path{updmap-sys -{}-syncwithtrees}\newline
\Path{updmap-sys -{}-enable Map=tudscr.map}\newline
\Path{updmap-sys -{}-force}
\end{quoting}
%
Sind die Schriften danach immer noch nicht verfügbar, so wurden bestimmt schon 
weitere Schriften auf dem System \emph{lokal} installiert. In diesem Fall 
sollte der Vorgang nochmals für eine lokale Schriftinstallation mit 
%
\begin{quoting}
\Path{updmap -{}-syncwithtrees}\newline
\Path{updmap -{}-enable Map=tudscr.map}\newline
\Path{updmap -{}-force}
\end{quoting}
%
ausgeführt werden. Dieses Vorgehen macht allerdings den Befehl 
\Path{updmap-sys} von nun an wirkungslos. Nach einer systemweiten Installation 
neuer Schriften~-- beispielsweise bei der Aktualisierung der Distribution~-- 
müssen diese über den manuellen Aufruf von \Path{updmap} zukünftig durch den 
Anwender lokal bei \Distribution{\hologo{TeX}~Live}|?| respektive 
\Distribution{Mac\hologo{TeX}}|?| registriert werden.

\Attention{%
  Für die Schriftinstallation werden die Skripte \Path{tftopl}, \Path{pltotf} 
  und \Path{vptovf} benötigt, welche bei \Distribution{\hologo{TeX}~Live}|?| 
  beziehungsweise \Distribution{Mac\hologo{TeX}}|?| über das Paket 
  \Package*{fontware} aus \Package*{collection-fontutils}<> 
  bereitgestellt werden und zwingend installiert sein müssen.
}


\subsection{Installation der PostScript-Schriften unter Windows}
\tudhyperdef*{sec:install:win}%
%
Zur Installation der Schriften des \CDs für das \TUDScript-Bundle ist das Archiv
\hrfn{\Download{fonts/TUD-Script_fonts_Windows.zip}}%
{\File*{TUD-Script\_fonts\_Windows.zip}}
vorgesehen. Dieses ist sowohl für \Distribution{\hologo{TeX}~Live}|?| als auch
\Distribution{\hologo{MiKTeX}}|?| nutzbar und enthält~-- bis auf die jeweiligen 
Schriftarchive selbst~-- alle benötigten Dateien. Diese sollten nach dem 
Entpacken des Archivs in das gleiche Verzeichnis kopiert werden. Vor der 
Verwendung des Skripts \File{tudscr\_fonts\_install.bat} sollte sichergestellt 
werden, dass sich \emph{alle} der folgenden Dateien im selben Verzeichnis 
befinden:
%
\settowidth\tempdim{\File{tudscr\_fonts\_install.zip}~}%
\begin{description}[labelwidth=\tempdim,labelsep=1em]
  \item[\File{tudscr\_fonts\_install.bat}]Installationsskript
  \item[\File{Univers\_PS.zip}]Archiv mit Schriftdateien für \Univers
  \item[\File{DIN\_Bd\_PS.zip}]Archiv mit Schriftdateien für \DIN
  \item[\File{tudscr\_fonts\_install.zip}]Archiv mit Metriken für die
    Schriftinstallation via \Package{fontinst}
\end{description}
%
Beim Ausführen des Installationsskripts werden alle Schriften standardmäßig in 
ein lokales Nutzerverzeichnis installiert. Wird das Skript über das Kontextmenü 
mit Administratorrechten ausgeführt, erfolgt die Installation in einem Pfad, 
der \emph{für alle Nutzer} gültig und lesbar ist.



\subsection{Installation der PostScript-Schriften unter Linux und OS~X}
\tudhyperdef*{sec:install:unix}%
%
Zur Installation der Schriften des \CDs für das \TUDScript-Bundle ist das Archiv
\hrfn{\Download{fonts/TUD-Script_fonts_Unix.zip}}%
{\File*{TUD-Script\_fonts\_Unix.zip}}
 vorgesehen. Dieses ist sowohl für \Distribution{\hologo{TeX}~Live}|?| als auch 
\Distribution{Mac\hologo{TeX}}|?| nutzbar und enthält~-- bis auf die jeweiligen 
Schriftdateien selbst~-- alle benötigten Dateien. Diese sollten nach dem 
Entpacken des Archivs in das gleiche Verzeichnis kopiert werden. Vor der 
Verwendung des Skripts \File{tudscr\_fonts\_install.sh} sollte sichergestellt 
werden, dass sich \emph{alle} der folgenden Dateien im selben Verzeichnis 
befinden:
%
\begin{description}[labelwidth=\tempdim,labelsep=1em]
  \item[\File{tudscr\_fonts\_install.sh}]Installationsskript
    (Terminal: \Path{bash tudscr\_fonts\_install.sh})
  \item[\File{Univers\_PS.zip}]Archiv mit Schriftdateien für \Univers
  \item[\File{DIN\_Bd\_PS.zip}]Archiv mit Schriftdateien für \DIN
  \item[\File{tudscr\_fonts\_install.zip}]Archiv mit Metriken für die
    Schriftinstallation via \Package{fontinst}
\end{description}
%
\minisec{Anmerkung zu Linux und OS~X}
\Attention{%
  Nach dem Entpacken eines Release-Archivs im passenden Pfad\footnote{%
    beispielsweise \Path{cd~"\$HOME/Downloads/\PName{Unterordner}"}%
  } \textbf{muss das Skript zwingend} mit \Path{bash~\PName{Skript}.sh} im 
  Terminal in diesem Pfad mit den benötigten Dateien aufgerufen werden.
}
Dabei werden alle Schriften standardmäßig in das lokale Nutzerverzeichnis 
(\Path{\$TEXMFHOME}) installiert. Wird das Skript mit \Path{sudo} verwendet, 
erfolgt die Installation \emph{für alle Nutzer} in den lokalen Systempfad 
(\Path{\$TEXMFLOCAL}).

Es ist unbedingt darauf zu Achten, das beim Ausführen des Skriptes das Terminal 
im richtigen Verzeichnis aktiv ist. Bei den meisten unixoiden Betriebssystemen 
ist es problemlos möglich, das Terminal aus der Benutzeroberfläche heraus über 
das Kontextmenü im gewünschten Pfad zu öffnen. Geht dies nicht, so muss nach 
dem Öffnen des Terminals mit dem Befehl \Path{cd} erst zum entsprechenden 
Pfad~-- exemplarisch \Path{cd~"\$HOME/Downloads/\PName{Unterordner}"}~-- 
navigiert werden. Ein beispielhafter Aufruf im Terminal könnte also lauten:
%
\begin{quoting}
\Path{%
  cd~"\$HOME/Downloads/TUD-Script\_fonts\_Unix"{}\,\POParameter{ENTER}
}\newline
\Path{bash tudscr\_fonts\_install.sh\,\POParameter{ENTER}}
\end{quoting}


\subsection{Installationshinweise für portable Installationen}
\tudhyperdef*{sec:install:portable}%
%

Prinzipiell ist die Installation der PostScript-Schriften des \CDs bei der 
Nutzung von \Distribution{\hologo{TeX}~Live~Portable}|?| respektive  
\Distribution{\hologo{MiKTeX}~Portable}|?| äquivalent zur nicht"~portablen 
Variante, welche in \autoref{sec:install:fonts} beschrieben wird. Alle dort 
gegebenen Hinweise sollten sorgfältig berücksichtigt werden.

\minisec{\NoCaseChange{\hologo{TeX}}~Live~Portable}
Das folgende Vorgehen wurde mit Windows getestet. Empfehlungen für die portable 
Installation für unixoide Betriebssysteme können an \mailto{\tudscrmail} 
gesendet werden.
\begin{enumerate}
\item Installation von \Distribution{\hologo{TeX}~Live~Portable} in 
  \Path{\PName{Laufwerksbuchstabe}:\textbackslash LaTeX\textbackslash texlive}
\item Die Datei \File*{tl-tray-menu.exe} im Installationspfad öffnen
\item Im Infobereich der Taskleiste mit einem Rechtsklick auf das Symbol von 
  \Distribution{\hologo{TeX}~Live~Portable} das Kontextmenü öffnen und entweder 
  über die grafische Oberfläche (\emph{Package Manager}) oder die Kommandozeile 
  (\emph{Command Prompt}) ein Update durchführen
\item Über das Kontextmenü die Kommandozeile ausführen und in dieser das Skript 
  für die Installation der Schriften \File{tudscr\_fonts\_install.bat} starten. 
  Dabei gegebenenfalls zuvor in den Pfad des Skriptes wechseln~-- exemplarisch:
  \begin{quoting}[leftmargin=1.5em,rightmargin=0pt]
  \Path{%
    cd~/d~\%USERPROFILE\%\textbackslash{}Downloads%
    \textbackslash{}TUD-Script\_fonts\_Windows\,\POParameter{ENTER}
  }\newline%
  \Path{tudscr\_fonts\_install.bat}\,\POParameter{ENTER}
  \end{quoting}
  Der durch das Installationsskript voreingestellte Installationspfad kann im 
  Normalfall so belassen werden. Wird der Pfad geändert, so sollte dieser sich 
  logischerweise auf dem externen Speichermedium 
  \Path{\PName{Laufwerksbuchstabe}:\textbackslash} befinden.
  \Attention{%
    Ein Ausführen ohne die über \Distribution{\hologo{TeX}~Live~Portable} 
    geöffnete Kommandozeile führt zu Fehlern.
  }%
\end{enumerate}

\minisec{\NoCaseChange{\hologo{MiKTeX}}~Portable}
\begin{enumerate}
\item Installation von \Distribution{\hologo{MiKTeX}~Portable} in 
  \Path{\PName{Laufwerksbuchstabe}:\textbackslash LaTeX\textbackslash MiKTeX}%
  \footnote{Der Pfad darf \emph{nicht} auf der obersten Verzeichnisebene 
  \Path{\PName{Laufwerksbuchstabe}:\textbackslash} liegen.
  }
\item Die Datei \File{miktex-portable.cmd} im Installationspfad öffnen
\item Im Infobereich der Taskleiste mit einem Rechtsklick auf das Symbol von 
  \Distribution{\hologo{MiKTeX}~Portable} das Kontextmenü öffnen und ein Update 
  durchführen
\item Über das Kontextmenü die Kommandozeile ausführen und in dieser das Skript 
  für die Installation der Schriften \File{tudscr\_fonts\_install.bat} starten.
  Dabei gegebenenfalls zuvor in den Pfad des Skriptes wechseln~-- exemplarisch:
  \begin{quoting}[leftmargin=1.5em,rightmargin=0pt]
  \Path{%
    cd~/d~\%USERPROFILE\%\textbackslash{}Downloads%
    \textbackslash{}TUD-Script\_fonts\_Windows\,\POParameter{ENTER}
  }\newline
  \Path{tudscr\_fonts\_install.bat}\,\POParameter{ENTER}
  \end{quoting}
  Der durch das Installationsskript voreingestellte Installationspfad kann im 
  Normalfall so belassen werden. Wird der Pfad geändert, so sollte dieser sich 
  logischerweise auf dem externen Speichermedium 
  \Path{\PName{Laufwerksbuchstabe}:\textbackslash} befinden. Bei diesem Schritt 
  werden die Pakete \Package{fontinst}, \Package{cmbright} und \Package{iwona} 
  unter Umständen nachinstalliert.
  \Attention{%
    Ein Ausführen ohne die über \Distribution{\hologo{MiKTeX}~Portable} 
    geöffnete Kommandozeile führt zu Fehlern.
  }%
\item Bei der erstmaligen Verwendung einer der \TUDScript-Klassen werden die 
  Pakete \Package*{tudscr}, \Package{koma-script}, \Package{etoolbox},
  \Package{environ}, \Package{trimspaces}, \Package{xcolor} und
  \Package{mptopdf}<> von \Distribution{\hologo{MiKTeX}~Portable}
  nachinstalliert, falls diese nicht schon vorhanden sind und die automatische 
  Nachinstallation von Paketen aktiviert ist.
\end{enumerate}


\subsection{Probleme bei der Installation der PostScript-Schriften}
%
Wird Windows verwendet, kann es unter Umständen vorkommen, dass notwendige 
Befehlsaufrufe für das Installationsskript nicht ausgeführt werden können. In 
diesem Fall ist der Pfad zu den benötigten Dateien, welche normalerweise unter 
\Path{\%SystemRoot\%\textbackslash System32} zu finden sind, nicht in der 
Umgebungsvariable \Path{PATH} enthalten. Einen Hinweis zur Problemlösung ist 
\hrfn{http://latex.wcms-file3.tu-dresden.de/phpBB3/viewtopic.php?t=359}{%
  in diesem Beitrag im Forum%
}
zu finden.

Treten bei der Installation wider Erwarten Probleme auf, so ist zur Lösung eine 
Logdatei zu erstellen. Hierfür sollte unter \textbf{Windows} das Skript, 
welches Probleme verursacht, \emph{nicht} aus der Kommandozeile oder dem 
Explorer heraus sondern über \emph{Windows PowerShell} ausgeführt werden. 
Hierfür ist die Eingabe von \enquote{PowerShell} im Startmenü von Windows mit 
einem nachfolgenden Öffnen mittels \POParameter{ENTER}"~Taste ausreichend. 
Danach muss mit \Path{cd} zum Ordner des Skriptes navigiert und dieses mit 
\Path{.\textbackslash\PName{Skript}.bat|Tee-Object -file \PName{Skript}.log} 
ausgeführt werden. Ein Aufruf aus der PowerShell"~Konsole könnte lauten:
%
\begin{quoting}[rightmargin=0pt]
%  \small
  \Path{%
    cd~"\$env:USERPROFILE\textbackslash{}Downloads\textbackslash{}%
    TUD-Script\_fonts\_Windows"{}\,\POParameter{ENTER}%
  }\newline%
  \Path{%
    .\textbackslash{}tudscr\_fonts\_install.bat%
    |Tee-Object -file install.log\,\POParameter{ENTER}%
  }%
\end{quoting}
%
Für \textbf{unixartige Systeme} ist der Aufruf \Path{bash \PName{Skript}.sh > 
\PName{Skript}.log} aus dem Terminal heraus zu verwenden. Ein exemplarischer  
Aufruf im könnte lauten:
%
\begin{quoting}
  \Path{%
    cd~"\$HOME/Downloads/TUD-Script\_fonts\_Unix"{}\,\POParameter{ENTER}%
  }\newline
  \Path{bash tudscr\_fonts\_install.sh > install.log\,\POParameter{ENTER}}%
\end{quoting}
%
Die so erstellte Logdatei kann \emph{mit einer kurzen Fehlerbeschreibung} 
entweder im \Forum* gepostet oder direkt per E"~Mail an \mailto{\tudscrmail} 
gesendet werden.
\index{Installation!Schriftinstallation|!)}%
