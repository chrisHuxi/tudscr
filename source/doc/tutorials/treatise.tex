\RequirePackage{fix-cm}
\documentclass[%
  english,ngerman,%
  cdgeometry=no,DIV=12,automark%
]{tudscrartcl}
\usepackage[T1]{fontenc}
\ifpdftex{\usepackage[ngerman=ngerman-x-latest]{hyphsubst}}{}

\usepackage{tudscrtutorial}
\lstset{%
  inputencoding=utf8,extendedchars=true,
  literate=%
    {ä}{{\"a}}1 {ö}{{\"o}}1 {ü}{{\"u}}1
    {Ä}{{\"A}}1 {Ö}{{\"O}}1 {Ü}{{\"U}}1
    {~}{{\textasciitilde}}1 {ß}{{\ss}}1
}
\lstset{escapechar=§}

\usepackage{lmodern}
\TUDoptions{cdfont=false}
\KOMAoptions{headings=normal}

\usepackage{tudscrsupervisor}
\usepackage{array}
\usepackage{tabu}
\usepackage{tabularx}
\usepackage{tabulary}
\usepackage{booktabs}
\usepackage{siunitx}
\sisetup{%
  locale = DE,%
  input-decimal-markers={,},%
  input-ignore={.},%
  group-separator={\,},%
  group-minimum-digits=3%
}
\usepackage{xfrac}
\AfterPackage*{hyperref}{%
  \usepackage[%
    automake,%
    xindy={language=german-din},%
    acronym,%
    symbols,%
    nomain,%
    translate=babel,%
    nogroupskip,%
    nostyles,%
  ]{glossaries}
  \renewcommand*{\glsglossarymark}[1]{}
  \newignoredglossary{abbreviation}
  \newcommand*\newabbreviation[4][]{%
    \newacronym[type=abbreviation,#1]{#2}{#3}{#4}%
  }%
  \makeglossaries
}
\usepackage{csquotes}
\usepackage[backend=biber,style=alphabetic]{biblatex}
\addbibresource{\jobname.bib}
\DefineBibliographyStrings{ngerman}{urlseen = {Am:}}

\usepackage{enumitem}

\usepackage{caption}
\DeclareCaptionSubType[alph]{figure}
\DeclareCaptionSubType[alph]{table}
\usepackage{floatrow}
\renewcommand{\floatpagefraction}{0.7}

\usepackage{tikz}
\usetikzlibrary{chains}
\usetikzlibrary{decorations.markings}
\tikzset{on grid}

\usepackage{pstricks,pst-node}

\makeatletter
\newcommand*\pcolumnfuzz[1]{\pretocmd{\@endpbox}{\hfuzz=#1}{}{}}
\makeatother

\usepackage{bookmark}

\begin{filecontents}{\jobname.bib}
@book{talbot2012,
  author    = {Nicola L. C. Talbot},
  title     = {{\LaTeX} for Complete Novices},
  publisher = {Dickimaw Books},
  series    = {Dickimaw {\LaTeX} Series},
  volume    = {1},
  isbn      = {978-1-909440-00-5},
  address   = {Norfolk, UK},
  year      = {2012},
  url       = {http://www.dickimaw-books.com/latex/novices/},
  urldate   = {2014-12-01},
}
@book{talbot2013,
  author    = {Nicola L. C. Talbot},
  title     = {Using {\LaTeX} to Write a PhD Thesis},
  publisher = {Dickimaw Books},
  series    = {Dickimaw {\LaTeX} Series},
  volume    = {2},
  isbn      = {978-1-909440-02-9},
  address   = {Norfolk, UK},
  year      = {2013},
  url       = {http://www.dickimaw-books.com/latex/thesis/},
  urldate   = {2014-12-01},
}
@online{reichert2012,
  author       = {Reichert, Axel and Voß, Herbert},
  title        = {\LaTeX~-- Satz von Tabellen},
  organization = {Freie Universität Berlin},
  date         = {2012-01-12},
  url          = {http://userpage.fu-berlin.de/latex/Materialien/tabsatz.pdf},
  urldate      = {2014-12-01}
}
@article{neubauer1996,
  author       = {Marion Neubauer},
  title        = {Feinheiten bei wissenschaftlichen Publikationen~--
                  Mikrotypographie"=Regeln, Teil~I},
  journaltitle = {Die \TeX{}nische Komödie},
  year         = {1997},
  month        = {2},
  volume       = {4/96},
  pages        = {23-40},
  url          = {http://www.dante.de/tex/Dokumente/dtk-neubauer.pdf},
  urldate      = {2014-12-01},
}
@article{neubauer1997,
  author       = {Marion Neubauer},
  title        = {Feinheiten bei wissenschaftlichen Publikationen~--
                  Mikrotypographie"=Regeln, Teil~II},
  journaltitle = {Die \TeX{}nische Komödie},
  year         = {1997},
  month        = {5},
  volume       = {1/97},
  pages        = {25--44},
  url          = {http://www.dante.de/tex/Dokumente/dtk-neubauer.pdf},
  urldate      = {2014-12-01},
}
@online{struckmann2007,
  author  = {Struckmann, Werner},
  title   = {Einige typographische Grundregeln und ihre Umsetzung in \LaTeX},
  date    = {2007-09-03},
  url     = {http://www2.informatik.hu-berlin.de/sv/lehre/typographie.pdf},
  urldate = {2014-12-01}
}
@online{bier2009,
  author  = {Bier, Christoph},
  title   = {typokurz~-- Einige wichtige typografische Regeln},
  date    = {2009-05-21},
  url     = {http://zvisionwelt.wordpress.com/downloads/#typokurz},
  urldate = {2014-12-01}
}
\end{filecontents}


\begin{document}
\date{29.03.2017}
\author{Falk Hanisch\thanks{\noexpand\scriptsize\noexpand\mailto{\tudscrmail}}}
\title{%
  Ein Anwenderleitfaden für das Erstellen einer wissenschaftlichen Abhandlung%
}
\makeatletter
\begingroup%
  \def\and{, }%
  \let\thanks\@gobble%
  \let\footnote\@gobble%
  \hypersetup{%
    pdfauthor = {\@author},%
    pdftitle = {\@title},%
    pdfsubject = {Tutorial für \hologo{LaTeXe}},%
    pdfkeywords = {LaTeX, \TUDScript, Tutorial, Anwenderleitfaden},%
  }%
\endgroup%
\makeatother
\StartTutorial[%
  Der Versuch, ein allumfassendes Tutorial für eine wissenschaftliche Arbeit 
  zur Verfügung zu stellen gleicht der beschwerlichen Suche nach einer 
  eierlegenden Wollmilchsau. Es ist quasi nicht möglich, alle potenziellen 
  Anforderungen an eine wissenschaftliche Arbeit in einem Dokument abzudecken, 
  insbesondere weil es für diese fast von jedem Lehrstuhl mehr oder weniger 
  unterschiedliche Vorgaben zu Formatierung und Layout gibt. Dennoch soll hier 
  versucht werden, einen Großteil der für gewöhnlich auftretenden Erfordernisse 
  zu bearbeiten.
  
  Dieses Tutorial hat \emph{nicht} die Intention, \hologo{LaTeX}"=Einsteigern 
  sämtliche Grundlagen zu erläutern. Vielmehr wird davon ausgegangen, dass Sie 
  bereits erste Erfahrungen mit \hologo{LaTeXe} gesammelt haben. Dennoch wird 
  versucht, alle Schritte möglichst leicht nachvollziehbar zu gestalten. Sollte 
  Ihnen beim Lesen und Durcharbeiten des Tutorials etwas auf- oder missfallen, 
  so dürfen Sie mich gerne per E"~Mail kontaktieren. Auch Anregungen und 
  Wünsche dürfen sie mir gegenüber gerne kommunizieren.
  
  Für absolute Neueinsteiger gibt es einige freie Tutorials, welche die ersten 
  Schritte mit \hologo{LaTeXe} stark erleichtern. Sehr empfehlenswert ist die 
  ausführliche \hrfn{http://www.fadi-semmo.de/latex/workshop/}{Workshop-Reihe} 
  von Fadi~Semmo. Außerdem stellt Nicola~L.~C.~Talbot sehr gute Tutorials für 
  \hrfn{http://www.dickimaw-books.com/latex/novices/}{\hologo{LaTeX}-Novizen} 
  \cite{talbot2012} sowie 
  \hrfn{http://www.dickimaw-books.com/latex/thesis/}{Dissertationen} 
  \cite{talbot2013} zur freien Verfügung.
  
  In erster Linie ist dieser Leitfaden für Anwender gedacht, die für ihre
  wissenschaftliche Arbeit eine \TUDScript"=Dokumentklasse verwenden wollen. 
  Das vorgestellte Vorgehen kann jedoch~-- natürlich mit gewissen Abstrichen~-- 
  auch mit anderen Klassen, insbesondere denen aus dem \KOMAScript"=Bundle, 
  umgesetzt werden. Viele der hier im Folgenden verwendeten Optionen und
  Befehle aus dem \TUDScript-Bundle werden lediglich sporadisch in ihrer 
  Grundfunktion erläutert. Allerdings lassen sich detaillierte Erläuterung über 
  die türkisfarbigen Hyperlinks zum \tudhyperref{}{\TUDScript-Handbuch}  
  jederzeit sehr einfach öffnen. Des Weiteren wird auf eine Vielzahl von 
  Paketen verwiesen, deren Dokumentation über die magentafarbigen Links im 
  \CTAN* geöffnet werden kann. Alternativ kann dies auch über das Terminal 
  respektive die Kommandozeile mit dem Aufruf \Path{texdoc~\PName{Paket}} 
  direkt lokal erfolgen. Ein Großteil dieser Pakete wird zusätzlich kurz im 
  \TUDScript-Handbuch beschrieben.
  
  Der Anwenderleitfaden muss nicht zwingend vollständig nachvollzogen werden. 
  Dieser ist in einzelne Abschnitte untergliedert, damit Sie sich bestimmte 
  Aspekte erarbeiten können. Sollten Querbezüge zu den einzelnen Abschnitten 
  bestehen, werden diese auch genannt. Zu guter Letzt findet sich am Ende 
  dieses Dokumentes das komplette Tutorial als ausführbarer Quelltext. 
]
\tableofcontents
\listoffigures
\listoftables



\section{Einleitung}
\label{sec:introduction}%
%
Zu Beginn werden einige Pakete geladen, die quasi in jedem neu erstellten 
Dokument verwendet werden sollten. Alle weiteren in diesem Tutorial genutzten 
und/oder empfohlenen Pakete werden in den relevanten Abschnitten des Tutorials 
genauer erläutert. 

Beim Laden der Klasse mit \Macro*{documentclass} können Sie globale Optionen 
angeben. Diese werden sowohl an die Klasse selbst weitergereicht als auch allen 
in der Präambel geladenen Pakete zur Verfügung gestellt. Falls das Paket 
\Package{babel} für die Sprachauswahl im Dokument verwendet wird, sollten auf 
alle Fälle die verwendeten Sprachen als Klassenoption angegeben werden, wobei 
die zuletzt angegebene als aktuelle Sprache aktiviert wird. Dadurch werden 
diese nicht nur an das Paket \Package{babel} sondern auch an andere Pakete 
übergeben, welche sprachspezifische Einstellungen vornehmen.
%
\begin{Preamble}
\documentclass[english,ngerman]{tudscrreprt}
\usepackage{babel}
\end{Preamble}
%
Mit dem Paket \Package{fontenc} lässt sich die Kodierung der für die Ausgabe 
der verwendeten Schriften im Dokument spezifizieren, wobei die europäische 
Zeichenkodierung mit der Option~\Option{T1}(\Package{fontenc}) aktiviert wird. 
Kommt mit \Engine{LuaLaTeX} oder \Engine{XeLaTeX} ein Unicode"=Textsatzsystem 
zum Einsatz, kann alternativ das Paket \Package{fontspec} geladen werden.

Weiterhin kann die Eingabekodierung der kompilierten Datei eingestellt werden. 
Diese ist standardmäßig \PValue{utf8}. Bei der Verwendung von \Engine{pdfLaTeX} 
sind auch andere Kodierungen möglich.%
\footnote{%
  Sollte die Datei in einer anderen Kodierung gespeichert sein, so kann das 
  Paket \Package{selinput} genutzt werden, um automatisch zu erkennen, welche 
  Kodierung der genutzte Editor verwendet. Alternativ lässt sich die 
  Eingabekodierung mit dem Paket \Package{inputenc} auch manuell festgelegen. 
  In beiden Fällen sollte das Paket \Package{fontenc} \emph{vorher} geladen 
  worden sein.
} 

Wurde ein Sprachpaket für die deutsche Sprache geladen (\Package{babel} oder 
\Package{polyglossia}), so werden für \Engine{LuaLaTeX} und \Engine{XeLaTeX} 
erweiterte Trennmuster verwendet. Für \Engine{pdfLaTeX} lässt sich eine 
wesentlich verbesserte Worttrennung für die deutsche Sprache mit dem Paket 
\Package{hyphsubst} aktivieren. In \autoref{sec:hyphenation} können Sie weitere 
Informationen zur Verwendung von \Package{hyphsubst} sowie zum Thema 
\emph{Worttrennung} finden.
%
\begin{Preamble}
\usepackage[T1]{fontenc}
\ifpdftex{\usepackage[ngerman=ngerman-x-latest]{hyphsubst}}{}
\end{Preamble}
%
Für die \KOMAScript-Klassen wird außerdem das Paket \Package{scrhack} 
bereitgestellt. Dieses behebt einige Probleme bei der Kompatibilität mit 
anderen Paketen bezüglich des Erstellens von Verzeichnissen. Deshalb sollte 
dieses frühzeitig in der Präambel geladen werden. 
%
\begin{Preamble}
\usepackage{scrhack}
\end{Preamble}
%
Damit sind die allgemein notwendigen Pakete eingebunden. Es werden zwar weitere 
benötigt, diese werden allerdings in den einzelnen Abschnitten dieses Tutorials 
aufgeführt.



\section{Makrotypografie}
\subsection{Satzspiegel}
Gleich zu Beginn und bevor das eigentliche Verfassen der Arbeit beginnt, sollte 
sich der Verfasser Gedanken über das zu nutzenden Layout und den Satzspiegel 
machen, um bei der Finalisierung des Dokumentes keine böse Überraschung bei 
Seitenumbrüchen oder der Position von Gleitobjekten zu erleben.

Zuerst gilt es zu entscheiden, ob das Dokument einseitig oder beidseitig 
gesetzt werden soll. Ist Letzteres der Fall, so sollte die \KOMAScript-Option 
\Option{twoside}(\Package{typearea}) als Klassenoption angegeben werden. Im 
nächsten Schritt ist der zu verwendende Satzspiegel festzulegen. Hierfür kann 
die \TUDScript-Option \Option{cdgeometry}'manual' verwendet werden, welche im 
\TUDScript-Handbuch beschrieben ist. Normalerweise wird das Dokument im 
asymmetrischen Layout des \CDs gesetzt.
%
Dieses Verhalten wird mit \Option*{cdgeometry=false}'manual' deaktiviert und 
der Satzspiegel durch das Paket \Package{typearea} nach typografischen 
Gesichtspunkten konstruiert. Dabei ist für dessen Aufteilung maßgeblich die 
\KOMAScript-Option \Option{DIV}(\Package{typearea}) verantwortlich, mit der 
\KOMAScript"=Option \Option{BCOR}(\Package{typearea}) wird die Bindekorrektur 
angegeben. Diese wird in \autoref{sec:bcor} erläutert.

\subsection{Änderung des Zeilenabstandes (Durchschuss)}
Auf die Änderung des standardmäßigen Zeilenabstandes sollte aus typografischer 
Sicht bei der Verwendung einer Standardschrift im Normalfall verzichtet werden. 
Die Gründe hierfür werden im Handbuch von \TUDScript bei der Vorstellung des 
Paketes \Package{setspace} erklärt, welches im Zweifelsfall für dieses Anliegen 
den Befehl \Macro{setstretch}(\Package{setspace}) bereitstellt. Wird ein 
Dokument in der \OpenSans-Hausschrift der \TnUD gesetzt, sollte der Wert von 
\Macro{setstretch}(\Package{setspace}) im Bereich 1.05--1.1 liegen.

\subsection{Bindekorrektur}
\label{sec:bcor}%
%
Falls die Arbeit nach der Fertigstellung gebunden werden soll, so ist auf den 
notwendigen Binderand zu achten, quasi der Teil einer Seite, welcher durch die 
Bindung \enquote{verschwindet} und nicht mehr als sichtbarer Teil der Seite 
vorhanden ist. Als Faustregel gilt, dass die erforderliche Bindekorrektur in 
etwa der halben Höhe des Buchblocks entsprechen sollte. Dessen Höhe wiederum 
ist abhängig von der Anzahl der Seiten sowie der Papierdichte.

Wird qualitativ höherwertiges Papier mit einer Dichte von \SI{100}{g/m^2} 
verwendet, so entsprechen 100~Blatt einer Buchblockhöhe von zirka~\SI{12}{mm}. 
Dementsprechend wäre in diesem Beispiel eine Bindekorrektur von~\SI{6}{mm} 
notwendig, die sich mit der Klassenoption \Option*{BCOR=6mm} einstellen ließe. 
Sollte die erwartete Seitenzahl davon abweichen, kann die einzustellende 
Bindekorrektur linear skaliert werden. Für differierende Papierdichten sollte 
die Buchblockhöhe für die geschätzte Gesamtseitenanzahl in Erfahrung gebracht 
werden.



\section{Umschlagseite und Titel}
Die Umschlagseite und der Titel sind sich in ihrer Gestalt sehr ähnlich. 
Allerdings gibt es ein paar kleine Unterschiede. Einerseits werden auf dem 
Cover weniger Informationen als auf der Titelseite ausgegeben. Andererseits 
wird der Titel immer im Satzspiegel des restlichen Dokumentes ausgegeben, 
wohingegen die Umschlagseite~-- ohne weitere Optionen~-- im asymmetrischen 
Layout des \CDs der \TnUD erscheint. Wie dieses Verhalten geändert werden kann, 
ist im Handbuch für \Macro{makecover}'manual' erläutert. Die resultierende 
Ausgabe des nachfolgenden Quelltextauszugs ist in \autoref{fig:title} zu sehen 
und stellt eine mögliche Ausprägung von Umschlagseite und Titel dar.
%
\begin{figure}
\IncludeStandalone{title}[1,2]
\caption{Umschlagseite und Titel}
\label{fig:title}
\end{figure}

\begin{Trunk!}{title}
\faculty{Juristische Fakultät}
\department{Fachrichtung Strafrecht}
\institute{Institut für Kriminologie}
\chair{Lehrstuhl für Kriminalprognose}
\title{%
  Entwicklung eines optimalen Verfahrens zur Eroberung des
  Geldspeichers in Entenhausen
}
\thesis{master}
\graduation[M.Sc.]{Master of Science}
\author{%
  Mickey Mouse%
  \matriculationnumber{12345678}%
  \dateofbirth{2.1.1990}%
  \placeofbirth{Dresden}%
  \course{Klinische Prognostik}%
  \discipline{Individualprognose}%
\and%
  Donald Duck%
  \matriculationnumber{87654321}%
  \dateofbirth{1.2.1990}%
  \placeofbirth{Berlin}%
  \course{Statistische Prognostik}%
  \discipline{Makrosoziologische Prognosen}%
}
\matriculationyear{2010}
\supervisor{Dagobert Duck \and Mac Moneysac}
\professor{Prof. Dr. Kater Karlo}
\date{10.09.2014}
\makecover
\maketitle

\end{Trunk!}



\section{Vor- und Nachspann}
In den folgenden Unterabschnitten werden Elemente vorgestellt, welche häufig 
als Bestandteil einer wissenschaftlichen (Abschluss"~)Arbeit gefordert werden. 
Die Platzierung oder Position der vorgestellten Elemente innerhalb der Arbeit 
ist nicht eindeutig durch eine Norm oder dergleichen festgelegt. Vielmehr gibt 
es meist eine Richtlinie vom verantwortlichen Prüfungsamt oder eine konkrete 
Vorgabe des betreuenden wissenschaftlichen Mitarbeiters oder Hochschullehrers.

Verwenden Sie für ihre Arbeit eine Buchklasse, so können die entsprechenden 
Abschnitte mit \Macro{frontmatter}(\Package{koma-script}) (Vorspann), 
\Macro{mainmatter}(\Package{koma-script}) (Hauptteil) und 
\Macro{backmatter}(\Package{koma-script}) (Nachspann) gekennzeichnet werden. 
Einen Anhang starten Sie mit \Macro{appendix}(\Package{koma-script}), womit 
unter anderem die Kapitelnummerierung auf Großbuchstaben umgeschaltet wird. 
Weitere Hinweise zu den genannten Befehlen sind im \scrguide nachzulesen.


\subsection{Aufgabenstellung}
\label{sec:task}%
%
Das Erstellen der Aufgabenstellung einer Abschlussarbeit im \CD der \TnUD muss 
das Paket \Package{tudscrsupervisor}'manual' geladen werden.
%
\begin{Preamble}
\usepackage{tudscrsupervisor}

\end{Preamble}
%
Dieses Paket stellt die Umgebung \Environment{task}'manual' sowie den Befehl 
\Macro{taskform}'manual' bereit. Bei beiden Varianten wird zu Beginn eine 
Tabelle mit Autoreninformationen erzeugt. Am Ende werden der oder die Betreuer 
der Arbeit sowie Professor und gegebenenfalls der Prüfungsausschussvorsitzende 
ausgegeben. Mit der Umgebung \Environment{task}'manual' kann zwischen Kopf und 
Fuß ein beliebiger Inhalt gesetzt werden. Der Befehl \Macro{taskform}'manual' 
hingegen erzeugt eine standardisierte Ausgabe, wobei der Inhalt des zweiten 
obligatorischen Arguments in einer 
\Environment{itemize}(\Package{koma-script},\Package{enumitem})-Umgebung 
verwendet wird und somit \Macro{item}(\Package{koma-script},\Package{enumitem})
zu nutzen ist. Das Resultat des folgenden Quelltextes ist in \autoref{fig:task} 
zu sehen. 
%
\begin{figure}
\IncludeStandalone{task}[1,2]
\caption{Aufgabenstellung in freier und standardisierter Form}
\label{fig:task}
\end{figure}

\begin{Hint!}{task}
\faculty{Juristische Fakultät}
\department{Fachrichtung Strafrecht}
\institute{Institut für Kriminologie}
\chair{Lehrstuhl für Kriminalprognose}
\title{%
  Entwicklung eines optimalen Verfahrens zur Eroberung des
  Geldspeichers in Entenhausen
}
\thesis{master}
\author{%
  Mickey Mouse%
  \matriculationnumber{12345678}%
  \course{Klinische Prognostik}%
  \discipline{Individualprognose}%
\and%
  Donald Duck%
  \matriculationnumber{87654321}%
  \course{Statistische Prognostik}%
  \discipline{Makrosoziologische Prognosen}%
}
\matriculationyear{2010}
\issuedate{1.2.2015}
\duedate{1.8.2015}
\supervisor{Dagobert Duck \and Mac Moneysac}
\chairman{Prof. Dr. Primus von Quack}
\professor{Prof. Dr. Kater Karlo}

\newcommand{\taskcontent}{%
  Momentan ist das besagte Thema in aller Munde. Insbesondere wird es
  gerade in vielen~-- wenn nicht sogar in allen~-- Medien diskutiert.
  Es ist momentan noch nicht abzusehen, ob und wann sich diese Situation 
  ändert. Eine kurzfristige Verlagerung aus dem Fokus der Öffentlichkeit
  wird nicht erwartet.
  
  Als Ziel dieser Arbeit soll identifiziert werden, warum das Thema
  gerade so omnipräsent ist und wie dieser Effekt abgeschwächt werden
  könnte. Zusätzlich sind Methoden zu entwickeln, mit denen sich ein 
  ähnlicher Vorgang zukünftig vermeiden lässt.
}

\begin{task}
\smallskip
\par\noindent
\taskcontent
\end{task}

\taskform[pagestyle=empty]{\taskcontent}{%
  \item Recherche
  \item Analyse
  \item Entwicklung eines Konzeptes
  \item Anwendung der entwickelten Methodik
  \item Dokumentation und grafische Aufbereitung der Ergebnisse
}

\end{Hint!}
\begin{Trunk+}
\newcommand{\taskcontent}{%
  Momentan ist das besagte Thema in aller Munde. Insbesondere wird es
  gerade in vielen~-- wenn nicht sogar in allen~-- Medien diskutiert.
  Es ist momentan noch nicht abzusehen, ob und wann sich diese Situation 
  ändert. Eine kurzfristige Verlagerung aus dem Fokus der Öffentlichkeit
  wird nicht erwartet.
  
  Als Ziel dieser Arbeit soll identifiziert werden, warum das Thema
  gerade so omnipräsent ist und wie dieser Effekt abgeschwächt werden
  könnte. Zusätzlich sind Methoden zu entwickeln, mit denen sich ein 
  ähnlicher Vorgang zukünftig vermeiden lässt.
}
\taskform[pagestyle=empty]{\taskcontent}{%
  \item Recherche
  \item Analyse
  \item Entwicklung eines Konzeptes
  \item Anwendung der entwickelten Methodik
  \item Dokumentation und grafische Aufbereitung der Ergebnisse
}

\end{Trunk+}


\subsection{Zusammenfassung}
Häufig wird zu Beginn einer wissenschaftliche Arbeit die Motivation und der 
Inhalt dieser zusammengefasst, um den Leser die Thematik der Abhandlung 
vorzustellen. in den meisten Fällen wird diese dabei in deutscher und 
englischer Sprache verfasst. Hierfür stellt \KOMAScript{} bereits die Umgebung 
\Environment{abstract}'manual' bereit. Vielfach wird der Wunsch geäußert, 
sowohl die deutsche als auch die englische Zusammenfassung auf derselben Seite 
zu setzen. Diese Variante kann mithilfe der \TUDScript-Klassen sehr einfach 
umgesetzt werden, wie der nachfolgende Quelltextauszug zeigt. Die resultierende 
Ausgabe ist in \autoref{fig:abstr} zu sehen.
%
\begin{figure}
\centering
\IncludeStandalone[width=.5\textwidth]{abstract}
\caption{Zusammenfassung in deutscher und englischer Sprache}
\label{fig:abstr}
\end{figure}

\begin{Trunk!}{abstract}
\TUDoption{abstract}{multiple,section}
\begin{abstract}
  Dies ist der deutschsprachige Teil der Zusammenfassung, in dem die
  Motivation sowie der Inhalt der nachfolgenden wissenschaftlichen
  Abhandlung kurz dargestellt werden.
\nextabstract[english]
  This is the english part of the summary, in which the motivation and
  the content of the following academic treatise are briefly presented.
\end{abstract}

\end{Trunk!}



\subsection{Selbstständigkeitserklärung und Sperrvermerk}
Für die meisten Abschlussarbeiten an der \TnUD wird vom Verfasser eine 
Selbstständigkeitserklärung verlangt. Für diese wird ein Standardtext 
bereitgestellt. Dieser kann mit dem Befehl \Macro{confirmation}'manual' 
ausgegeben werden. Wurde das Thema in Kooperation mit einem Unternehmen 
bearbeitet, so wird zumeist auch ein Sperrvermerk gefordert, welcher mit 
\Macro{blocking}'manual' erzeugt werden kann. Mit \Macro{declaration}'manual' 
lassen sich beide Erklärungen direkt nacheinander erzeugen. Die verwendete 
Überschrift und ein möglicher Eintrag in das Inhaltsverzeichnis können über die 
Option \Option{declaration}'manual' reguliert werden. Eine mögliche Ausprägung 
der Erklärungen ist in \autoref{fig:decl} abgebildet.
%
\begin{figure}
\centering
\IncludeStandalone[width=.5\textwidth]{declaration}
\caption{Selbstständigkeitserklärung und Sperrvermerk}
\label{fig:decl}
\end{figure}

\begin{Hint!}{declaration}
\title{%
  Entwicklung eines optimalen Verfahrens zur Eroberung des
  Geldspeichers in Entenhausen
}
\author{Mickey Mouse\and Donald Duck}
\declaration[company=FIRMA]
\end{Hint!}
\begin{Trunk+}
\declaration[company=FIRMA]

\end{Trunk+}

\subsection{Inhalts-, Abbildungs-, und Tabellenverzeichnis}
Das Inhaltsverzeichnis wird mit \Macro{tableofcontents}(\Package{koma-script}) 
erzeugt und führt die Gliederung des erstellten Dokumentes entsprechend der 
verwendeten Befehle%
\footnote{%
  \Macro{part}(\Package{koma-script}),
  \Macro{addpart}(\Package{koma-script}), 
  \Macro{chapter}(\Package{koma-script}), 
  \Macro{addchap}(\Package{koma-script}), 
  \Macro{section}(\Package{koma-script}),
  \Macro{addsec}(\Package{koma-script}) etc.%
}
auf. Wurde das Paket \Package{hyperref} geladen, so werden im 
Inhaltsverzeichnis PDF"~Hyperlinks auf die einzelnen Abschnitte erzeugt.

Sowohl Abbildungen als auch Tabellen werden in \hologo{LaTeX} normalerweise 
mit speziellen Umgebungen~-- \Environment{figure} und \Environment{table}~-- 
eingebunden. Innerhalb dieser sogenannten Gleitumgebungen kann der Befehl 
\Macro{caption}[%
  \OParameter{Verzeichniseintrag}\Parameter{Bezeichnung}%
](\Package{koma-script},\Package{caption}) genutzt werden, um diesen eine 
Bezeichnung hinzuzufügen. Mit \Macro{listoffigures}(\Package{koma-script}) 
beziehungsweise \Macro{listoftables}(\Package{koma-script}) lassen sich 
Verzeichnisse erstellen, in denen alle Gleitobjekte des jeweiligen Typs 
ausgegeben werden, falls diese denn eine Bezeichnung hinzugefügt wurde. 

Sollen Abbildungen oder Tabellen außerhalb ihrer angestammten Gleitumgebung 
\Environment{figure} beziehungsweise \Environment{table} genutzt und benannt 
werden, kann dies entweder mit dem Befehl 
\Macro{captionof}[%
  \Parameter{Typ}\OParameter{Verzeichniseintrag}\Parameter{Bezeichnung}%
](\Package{koma-script}) oder~-- falls die beiden Pakete \Package{caption} und 
\Package{hyperref} genutzt werden~-- auch wie gewohnt mit 
\Macro{caption}(\Package{koma-script},\Package{caption}) allerdings zusätzlich 
ergänzt mit einem zuvor aufgerufenem
\Macro{captionsetup}[\PParameter{type=figure/table}](\Package{caption})  
erfolgen. Weitere Informationen diesbezüglich sind dem \scrguide respektive der 
Anleitung zu \Package{caption} zu entnehmen. In \autoref{sec:floats} wird 
genauer auf Gleitumgebungen eingegangen.
%
\begin{Trunk}
\tableofcontents
\listoffigures
\listoftables

\end{Trunk}
\begin{Trunk+}
\printacronyms[style=acrotabu]
\printsymbols[style=symblongtabu]

\setchapterpreamble{%
  \renewcommand*{\dictumwidth}{.4\textwidth}%
  \dictum[Johann Wolfgang von Goethe]{%
    Es irrt der Mensch, solang er strebt.%
  }%
  \bigskip
}
\chapter{Einleitung}
Nachdem nun der Vorspann und~-- bis auf das Literaturverzeichnis am 
Ende des Dokumentes auf Seite~\pageref{sec:bibliography}~-- alle 
Verzeichnisse erfolgreich ausgegeben wurden, wird nun die Verwendung 
der weiteren Umgebungen und Befehle demonstriert, welche im Tutorial 
\texturn{treatise.pdf} vorgestellt wurden.

\end{Trunk+}


\subsection{Abkürzungs- und Symbolverzeichnis}
\label{sec:glossaries}%
%
Für die Auszeichnung von Abkürzungen gibt es zwei sehr gute Pakete, die dieses 
Unterfangen stark vereinfachen. Die einfachere~-- jedoch nicht so mächtige~-- 
der beiden Varianten ist die Nutzung des Paketes \Package{acro}. Sollen nur 
Abkürzungen und gegebenenfalls eine sortierte Liste dieser gesetzt werden, ist 
dieses allerdings absolut ausreichend. Für ein Symbolverzeichnis lässt sich in 
dieser Variante das Paket \Package{nomencl} nutzen. Dieses bietet meiner 
Meinung nach jedoch keine großen Vorteile, stattdessen kann auch einfach eine 
Tabelle händisch erzeugt werden. Das Paket \Package{acro} ist sehr gut und 
ausführlich dokumentiert. Deshalb wird hier auf eine exemplarische Erläuterung 
verzichtet und stattdessen auf dessen Dokumentationen verwiesen. Alternativ 
kann auch das Paket \Package{acronym} verwendet werden, wobei die alphabetische 
Sortierung der Abkürzungen durch den Anwender erfolgen muss, weshalb ich von 
diesem eher abrate.

Die andere Möglichkeit ist die Nutzung des Paketes \Package{glossaries}, das 
eine große Zahl an Einstellmöglichkeiten und Optionen besitzt, allerdings auch 
etwas Zeit für die Einarbeitung und Studium der Dokumentation benötigt. Der 
ursprüngliche Einsatzzweck dieses Paketes ist das Setzen eines fachsprachlichen 
oder technischen Glossars. Es bietet zusätzlich die Mittel zum Erzeugen eines 
Abkürzungs- sowie Symbolverzeichnis. Es soll folgend hier kurz erläutert 
werden, wie das Paket zu verwenden ist. Für weiterführende Beispiele sollte 
die Dokumentationen zu Rate gezogen werden, welche dahingehend keine Wünsche 
offen lässt. 

Das Paket \Package{glossaries} sollte immer \emph{nach} \Package{hyperref} 
geladen werden. Entweder Sie achten explizit darauf oder Sie verwenden den 
Befehl \Macro{AfterPackage*}(\Package{scrbase}) aus dem \KOMAScript-Bundle. Mit 
diesem können Sie die Quelltext nach dem Laden eines Paketes ausführen. 
\Attention{%
  Es ist allerdings darauf zu achten, dass der Quelltext nur ausgeführt wird, 
  wenn das avisierte Zielpaket auch tatsächlich geladen wird.
}
Falls Sie demnach auf \Package{hyperref} jedoch nicht auf \Package{glossaries} 
verzichten möchten, sollte der nachfolgende Quelltext am Ende der Präambel 
eingefügt werden.

Die für \Package{glossaries} verwendeten Optionen werden kurz erläutert.
%
\begin{Preamble+}
\AfterPackage*{hyperref}{%
\end{Preamble+}
\begin{Preamble}
\usepackage[%
\end{Preamble}
%
Die zwei Paketoptionen \Option{acronym}(\Package{glossaries}) sowie 
\Option{symbols}(\Package{glossaries}) erzeugen die beiden Verzeichnisse für 
Abkürzungen und Symbole. Die Option \Option{nomain}(\Package{glossaries}) wird 
immer dann verwendet, wenn im Dokument kein zusätzliches allgemeines oder 
technisches Glossar erzeugt werden soll.
%
\begin{Preamble}
  acronym,% Abkürzungen
  symbols,% Formelzeichen
  nomain,% kein Glossar
\end{Preamble}
%
Durch \Option{nogroupskip}(\Package{glossaries}) wird der automatische Abstand 
zwischen den Einträgen zur Gruppierung innerhalb eines Glossars entfernt, die 
Option \Option{toc}(\Package{glossaries}) fügt alle erzeugten Verzeichnisse 
dem Inhaltsverzeichnis hinzu, \Option{section}(\Package{glossaries}) bestimmt 
die Gliederungsebene der Überschrift.
%
\begin{Preamble}
  nogroupskip,%
  toc,%
  section=chapter,%
\end{Preamble}
%
Mit der Option \Option{nostyles}(\Package{glossaries}) kann man gegebenenfalls 
das Laden der von \Package{glossaries} vordefinierten Stilen verhindern. Dies 
ist insbesondere sinnvoll, wenn für die einzelnen Glossare und Verzeichnisse 
ohnehin eigene Stile erstellt werden, wie dies nachfolgend geschieht.
%
\begin{Preamble}
  nostyles,%
\end{Preamble}
%
Mit der Option \Option{translate}(\Package{glossaries}) werden die Überschriften
aller Glossare in der Dokumentsprache gesetzt, wobei \Option*{translate=babel} 
die Nutzung des Paketes \Package{babel} erzwingt. 
%
\begin{Preamble}
  translate=babel,%
\end{Preamble}
%
Das Paket \Package{glossaries} erstellt bei der Kompilierung des Dokumentes 
Hilfsdateien, anhand derer die Indexe erzeugt und sortiert werden können. Da 
\Engine{pdfLaTeX} diese Funktionalität nicht bereitstellt, ist hierfür der 
Aufruf einer passenden Anwendung notwendig. Die meisten Distributionen stellen
dafür \Application{makeindex} bereit. Dieses Programm kann prinzipiell zwar 
Einträge alphabetisch sortieren, bietet allerdings weder die Nutzung von 
Unicode-Symbolen noch eine Sprachunterstützung. Aufgrund der genannten 
Nachteile ist \Application{xindy} zum Sortieren der Einträge zu empfehlen, da 
dieses Programm sowohl eine Unterstützung von Unicode als auch die Möglichkeit, 
nach sprachabhängigen Regeln zu sortieren, bietet. Allein für die deutsche 
Sprache gibt es beispielsweise zwei Varianten der Sortierung~-- nach DIN und 
nach Duden. 

Ohne weiteres Zutun des Anwenders wird vom Paket \Package{glossaries} in der 
Grundeinstellung \Application{makeindex} genutzt. Soll stattdessen für die 
Sortierung der Glossar- und Verzeichniseinträge \Application{xindy} zum Einsatz 
kommen, muss die Option \Option{xindy}(\Package{glossaries}) aktiviert werden.
Dieser kann zusätzlich sowohl die zu nutzende Spracheinstellung zur Sortierung 
als auch die gewünschte Eingabekodierung an \Application{xindy} übergeben 
werden. Beispielsweise wird mit \Option*{xindy=\PParameter{language=german-din}}
die deutsche Sortierung nach~DIN aktiviert.
%
%
\begin{Hint}
% mit Tex Live einfach verwendbar
  xindy=§\{§language=german-din§\}§,
\end{Hint}
\begin{Preamble+}
% mit Tex Live einfach verwendbar
  xindy={language=german-din},
\end{Preamble+}
%
Das Erstellen aller Glossare~-- unabhängig davon, ob \Application{xindy} oder 
\Application{makeindex} für die Sortierung zum Einsatz kommt~-- sollte das 
Perl"~Skript \File{makeglossaries}(\Package{glossaries}) verwendet werden, 
welches alle notwendigen Optionen an die jeweilige Anwendung weiterleitet.

\Attention{%
  Für den Aufruf von \Application{xindy} selbst als auch für die Ausführung des 
  Skriptes \File{makeglossaries}(\Package{glossaries}) ist ein Perl"=Interpreter
  notwendig. Dieser wird lediglich von \Distribution{\hologo{TeX}~Live} nicht 
  jedoch von \Distribution{\hologo{MiKTeX}} direkt bereitgestellt. Wird die 
  letztgenannte Distribution verwendet, muss der Anwender gegebenenfalls diesen 
  zusätzlich installieren.
}

Kann das Skript \File{makeglossaries}(\Package{glossaries}) nicht genutzt 
werden, kann alternativ dazu die Option \Option{automake}(\Package{glossaries})
aktiviert werden. Diese sorgt dafür, dass am Ende des \Engine{pdfLaTeX}"~Laufs 
der Aufruf von \Application{xindy} oder \Application{makeindex} direkt erfolgt. 
Soll \Application{xindy} verwendet werden, sind dafür allerdings erweiterte 
Schreibrechte notwendig, weshalb für diese Variante \Engine{pdfLaTeX} mit der 
Option \Option{-{}-shell-escape}(\Engine{pdfLaTeX}) respektive 
\Option{-{}-enable-write18}(\Engine{pdfLaTeX}) aufgerufen werden muss. Für 
\Application{xindy} ist außerdem die Angabe der Sprache über die Option 
\Option*{xindy=\PParameter{language=\dots}} zwingend notwendig. 

Ist weder die Nutzung des Skriptes \File{makeglossaries}(\Package{glossaries}) 
noch der~-- im Zweifel sicherheitskritische~-- Aufruf von \Engine{pdfLaTeX} mit 
erweiterten Schreibrechten möglich, so muss die gewünschte Anwendung für die 
Sortierung explizit durch den Anwender aufgerufen werden, wobei auf die Angabe 
der richtigen Parameter zu achten ist. Genaueres hierzu sowie Lösungen für 
Probleme beim Erstellen der Glossare und Verzeichnisse sind in der Dokumentation
von \Package{glossaries} zu finden.
%
\begin{Hint}
  automake,%
\end{Hint}
%
Damit sind alle verwendeten Optionen erläutert. Schließlich sorgt der Befehl 
\Macro{makeglossaries}(\Package{glossaries}) für das Erstellen der 
optionsabhängigen Stildateien für \Application{makeindex} respektive 
\Application{xindy} sowie das Erzeugen der benötigten Hilfsdateien.
%
\begin{Preamble}
]{glossaries}
\makeglossaries
\end{Preamble}
\begin{Preamble+}
}% Ende von AfterPackage

\AfterPackage*{glossaries}{%
\end{Preamble+}
%
Damit wäre der erste Teil zur Initialisierung überstanden und wir können zum 
eigentlichen Problem kommen. Wie wird nun ein Abkürzungs- und/oder 
Symbolverzeichnis erstellt?

\subsubsection{Abkürzungsverzeichnis}
Das Paket \Package{glossaries} stellt für die Definition von Abkürzungen einen 
speziellen Befehl bereit. Mit \Macro{newacronym}[%
  \LParameter\Parameter{Label}\Parameter{Abkürzung}\Parameter{Wortgruppe}%
](\Package{glossaries}) wird eine Abkürzung definiert und kann später über 
\Parameter{Label} genutzt werden. Die möglichen optionalen Parameter können in 
der Dokumentation zu \Package{glossaries} nachgeschlagen werden. Für ein 
kleines Beispiel werden drei Abkürzungen erstellt\dots
%
\begin{Trunk+}
\section{Die Verwendung von Akronymen und Symbolen}
\end{Trunk+}
\begin{Trunk*}
\newacronym{apsp}{APSP}{All-Pairs Shortest Path}
\newacronym{spsp}{SPSP}{Single-Pair Shortest Path}
\newacronym{sssp}{SSSP}{Single-Source Shortest Path}

\end{Trunk*}
%
\dots und diese in einer kurzen Textpassage mit dem \Package{glossaries}"=Befehl
\Macro{gls}[\Parameter{Label}](\Package{glossaries}) verwendet.
%
\begin{Trunk*}
In der Graphentheorie wird häufig die Lösung des Problems des kürzesten
Pfades zwischen zwei Knoten gesucht. Dieses Problem wird häufig auch
mit \gls{spsp} bezeichnet. Es lässt sich auf die Variationen \gls{sssp}
und \gls{apsp} erweitern. Für die Lösung von \gls{spsp}, \gls{sssp} 
oder \gls{apsp} kommen unterschiedliche Algorithmen zum Einsatz.

\end{Trunk*}
%
Gut zu sehen ist, dass sich die Ausgabe der Abkürzung bei der ersten Verwendung 
mit \Macro{gls}(\Package{glossaries}) von der zweiten~-- und jeder weiteren~-- 
unterscheidet. Das Verhalten lässt sich über verschiedene Stile mit 
\Macro{setacronymstyle}(\Package{glossaries}) anpassen. Die Ausgabe einer Liste 
aller Abkürzungen erfolgt mit:
%
\begin{Hint}
\printacronyms
\end{Hint}
\newglossarystyle{simple}{%
  \renewenvironment{theglossary}{}{\par}%
  \renewcommand*{\glossentry}[2]{%
    \noindent%
    \textsbn{\glsentryitem{##1}\glstarget{##1}\glossentryname{##1}}\quad
    \glossentrydesc{##1}\glspostdescription\space ##2%
    \par%
  }%
}
\begin{quoting}[rightmargin=0pt]
\glsdisablehyper
\printacronyms[style=simple]
\end{quoting}
%
Dabei werden die Akronyme in einer \Environment*{description}"=Umgebung gesetzt,
was absolut ausreichend ist. Mir persönlich ist allerdings die Darstellung in 
einer quasi"~tabellarischen Form lieber. Dabei soll der Stil mit fettgedruckten 
Abkürzungen beibehalten werden. Das \Package{glossaries}"=Paket stellt zwar 
auch eine Vielzahl an Stilen in Tabellenform bereit, allerdings nicht in dem 
gewünschten. Deshalb wird nachfolgend gezeigt, wie sich ein eigener Stil in 
Tabellenform kreieren lässt.

Es bieten sich die Umgebungen \Environment{tabularx}(\Package{tabularx}) oder 
\Environment{tabu}(\Package{tabu}) an, bei denen die Spaltenbreite teilweise 
automatisch berechnet wird, um sich manuelle Formatierungsarbeiten zu sparen. 
Folgend werden beide Varianten vorgestellt. Falls Sie noch keine Erfahrungen 
mit dem Tabellensatz in \hologo{LaTeXe} haben, lohnt sich vorher ein Blick in 
\autoref{sec:tables}, um die verwendeten Befehle und Umgebungen zu verstehen. 
Wie Sie Stile definieren, die Seitenumbrüche in einer Tabelle zulassen, können 
Sie in diesem \autorefname[sec:glossaries] bei der Symbolverzeichniserstellung 
erfahren.

\minisec{Eigener Stil mit \Environment*{tabularx}}
Ein eigener Glossarstil kann mit \Macro{newglossarystyle}(\Package{glossaries}) 
definiert werden, wobei für den neu definierten Stil die Umgebung 
\Environment{theglossary}(\Package{glossaries}) umdefiniert wird. Als erstes 
kommt die Tabellenumgebung \Environment{tabularx}(\Package{tabularx}) aus dem 
gleichnamigen Paket, welches in \autoref{sec:tabularx} vorgestellt wird, zum 
Einsatz.

Es werden in der Tabelle drei Spalten definiert. Die erste und letzte Spalte 
sind schlicht linksbündige Standardspalten~(\PValue{l}). In diesen werden die 
Abkürzungen selbst sowie die Seitenzahlen eingetragen. Die Verwendung 
von~\PValue{@\PParameter{}} führt dazu, dass der normalerweise vor der ersten 
und nach der letzten Spalte eingefügte Abstand von \Length{tabcolsep} entfällt. 
Die Breite der Spalte vom Typ~\PValue{X} wird automatisch berechnet.

Aufgrund der Implementierung von \Environment{tabularx}(\Package{tabularx}) 
lässt sich diese nicht als verschachtelte Umgebung verwenden. Allerdings kann 
stattdessen die mit \Macro*{tabularx} sowie \Macro*{endtabularx} die 
Low"~Level"~Variante genutzt werden. Die Definition des neuen Stils 
\PValue{acrotabularx} wird nachfolgend ausgegeben, die weitergehende 
Erläuterung schließt sich daran an.
%
\CodeHook{\let\newglossarystyle\renewglossarystyle}
\begin{Hint*}
\newglossarystyle{acrotabularx}{%
  \renewenvironment{theglossary}{%
    \tabularx{\linewidth}{@{}lXl@{}}%
  }{%
    \endtabularx\par\bigskip%
  }%
  \renewcommand*{\glossaryheader}{}%
  \renewcommand*{\glsgroupheading}[1]{}%
  \renewcommand*{\glsgroupskip}{}%
  \renewcommand*{\glossentry}[2]{%
    \glsentryitem{##1}% Entry number if required
    \glstarget{##1}{\sffamily\bfseries\glossentryname{##1}} &
    \glsentrydesc{##1} &
    ##2\tabularnewline
  }
}

\end{Hint*}
%
Der Rest des Stils ist schnell erläutert. Zunächst wird auf Tabellenköpfe 
(\Macro{glossaryheader}(\Package{glossaries})) sowie Überschriften und Abstände 
bei Gruppierungen (\Macro{glsgroupheading}(\Package{glossaries}) sowie 
\Macro{glsgroupskip}(\Package{glossaries})) verzichtet. Der Befehl 
\Macro{glossentry}(\Package{glossaries}) ist verantwortlich für die 
Formatierung der Einträge im Abkürzungsverzeichnis. Dieser wird intern durch 
\Package{glossaries} mit zwei obligatorischen Argumenten aufgerufen. Das erste 
enthält das entsprechende Label, das zweite ein kommaseparierte Liste der 
Seitenzahlen. Dabei stehen verschiedene Makros zur Auswahl, um anhand des 
gegebenen Labels die gewünschten Informationen zu extrahieren.%
\footnote{%
  bspw. mit \Macro{glossentryname}(\Package{glossaries}) die Bezeichnung oder 
  mit \Macro{glsentrydesc}(\Package{glossaries}) die dazugehörige Beschreibung%
}

Der Befehl \Macro{glossentry}(\Package{glossaries}) wurde so definiert, dass 
für jeden Eintrag eine separate Zeile in der Tabelle erzeugt wird, wo in der 
ersten Spalte die Abkürzung selbst, in der zweiten die Langform und in der 
dritten Spalte schließlich die Liste der Seiten, auf welchen die jeweilige 
Abkürzung mit \Macro{gls}[\Parameter{Label}](\Package{glossaries}) verwendet 
wurde, ausgegeben wird. Zum Abschluss die resultierende Ausgabe des 
Abkürzungsverzeichnisses im neuen Stil.
%
\begin{Hint}
\printacronyms[style=acrotabularx]
\end{Hint}
\begin{quoting}[rightmargin=0pt]
\InputCode
\end{quoting}
%
Sollte der Platz für die Erläuterungen wie in diesem Beispiel in der mittleren 
Spalte mehr als ausreichend sein und kein Zeilenumbruch benötigt werden, kann 
auch einfach eine \Environment{tabular}"=Umgebung mit einer \PValue{l}"~Spalte 
anstelle von \PValue{X} verwendet werden.

\minisec{Eigener Stil mit \Environment*{tabu}}
Die sehr komfortabel zu nutzende Umgebung \Environment{tabu}(\Package{tabu}) 
wird durch das Paket \Package{tabu} bereitgestellt. Es wird später in 
\autoref{sec:tabu} vorgestellt, wobei die dortigen Anmerkungen \emph{unbedingt} 
zu beachten sind. Für die bereits verwendeten 
\Environment{tabularx}(\Package{tabularx})"=Tabellen muss generell eine feste 
Tabellenbreite angegeben werden. Die Breite der \PValue{X}"~Spalten wird anhand 
der angegebenen Gesamtbreite und dem für andere Spalten vom 
Typ~\PValue{l},~\PValue{r}~und~\PValue{c} benötigten Platz berechnet. 

Wie \Environment{tabularx}(\Package{tabularx}) bietet auch die Umgebung 
\Environment{tabu}(\Package{tabu}) einen \PValue{X}"~Spaltentyp. Für diese kann 
jedoch anstelle einer fest vorgegebenen Breite auch \PValue{spread~0pt} 
angegeben werden. Dadurch werden \PValue{X}"~Spalten anfänglich in ihrer 
natürlichen Breite gesetzt. Sobald jedoch die Gesamtbreite der Tabelle den zur 
Verfügung stehenden Platz bis zum Zeilenende überschreiten würde, werden die 
\PValue{X}"~Spalten automatisch umbrochen. 

In umbrochenen Spalten gibt es beim Paket \Package{tabu} jedoch ein kleineres 
Problem. In diesen setzt \Package{tabu} zu wenig vertikalen Leerraum am unteren 
Ende. Um dieses Manko zu beheben, wird am Schluss jeder \PValue{X}"~Spalte mit 
\PValue{X<\PParameter{\Macro{strut}}} einfach der Befehl \Macro{strut} 
angehängt, der vertikalen Leerraum ober- und unterhalb der aktuellen Grundlinie 
einfügt. In \autoref{sec:tables} wird ein Ansatz aufgezeigt, wie dies 
automatisiert über einen neuen Spaltenstil ausmerzt werden kann. Der Rest des 
Stils ist identisch zu \PValue{acrotabularx}.
%
\CodeHook{\let\newglossarystyle\renewglossarystyle}
\begin{Preamble*}
\newglossarystyle{acrotabu}{%
  \renewenvironment{theglossary}{%
    \begin{tabu}spread 0pt{@{}lX<{\strut}l@{}}%
  }{%
    \end{tabu}\par\bigskip%
  }%
  \renewcommand*{\glossaryheader}{}%
  \renewcommand*{\glsgroupheading}[1]{}%
  \renewcommand*{\glsgroupskip}{}%
  \renewcommand*{\glossentry}[2]{%
    \glsentryitem{##1}% Entry number if required
    \glstarget{##1}{\sffamily\bfseries\glossentryname{##1}} &
    \glsentrydesc{##1} &
    ##2\tabularnewline
  }
}

\end{Preamble*}
\begin{Hint}
\printacronyms[style=acrotabu]
\end{Hint}
\begin{quoting}[rightmargin=0pt]
\glsdisablehyper
\InputCode
\end{quoting}
%
Welche Variante gewählt wird, ist letztendlich eine Frage, was für spezifische 
Anforderungen an die Formatierung des Abkürzungsverzeichnis gestellt werden. 
Natürlich ist auch die Definition eines eigenen Stils möglich. Wenn Sie wissen 
wollen, wie Sie einen Stil in Tabellenform erstellen können, welcher einen 
Seitenumbruch zulässt, sollte Sie den nächsten Abschnitt lesen.


\subsubsection{Symbolverzeichnis}
Für das Erzeugen eines Symbolverzeichnisses kann ebenfalls \Package{glossaries} 
verwendet werden. Allerdings muss dazu ein wenig mehr Aufwand getrieben werden, 
da das Paket hierfür keine dedizierte Schnittstelle bereitstellt. Wurde die 
Paketoption \Option{symbols}(\Package{glossaries}) angegeben wird jedoch 
zumindest das notwendige Glossar erstellt.

Als erstes sollte ein gut nutzbarer Befehl zum Definieren eines neuen Symbols 
erstellt werden. In Anlehnung an den Befehl für Abkürzungen \Macro*{newacronym} 
wird er \Macro{newformulasymbol}|?| genannt. Dieser hat ein optionales und vier 
obligatorische Argumente, wobei das optionale Argument prinzipiell alle 
Schlüssel"~Wert"~Paare enthalten kann, die durch das Paket \Package{glossaries} 
akzeptiert werden. Welche davon letztlich auch Auswirkungen haben, hängt 
allerdings von der Gestaltung des Stils durch den Anwender ab. Der Befehl hat 
folgende Definition:
%
\begin{quoting}[leftmargin=\parindent]
\Macro{newformulasymbol}[%
  \LParameter\Parameter{Label}\Parameter{Name}%
  \Parameter{Symbol}\Parameter{Einheit}%
]|?|%
\end{quoting}
%
Mit \PName{Label} erfolgt die eindeutige Kennzeichnung des Symbols. Außerdem 
wird dies standardmäßig für die Sortierung verwendet, was unter Umständen etwas
problematisch sein könnte. In diesem Fall ist eine manuelle Festlegung für den 
dazugehörigen Schlüssel durch den Anwender über das optionale Argument mit 
\PValue{sort=}\PName{Sortierung} eventuell sinnvoll. Nach dem \PName{Label} 
schließt der \PName{Name} für das Formelzeichen an, gefolgt vom \PName{Symbol} 
sowie der dazugehörige physikalische \PName{Einheit}.
%
\CodeHook{\let\newcommand\renewcommand}
\begin{Preamble*}
\newcommand*{\newformulasymbol}[5][]{%
  \newglossaryentry{#2}{%
    type=symbols,%
    name={#3},%
    description={\nopostdesc},%
    symbol={\ensuremath{#4}},%
    user1={\ensuremath{\mathrm{#5}}},%
    sort={#2},%
    #1%
  }%
}

\end{Preamble*}
%
Da es sich zumeist um mathematische Symbole handelt, wird für das Symbol und 
die Einheit mit \Macro{ensuremath} Sorge getragen, dass diese auch im 
Textmodus ohne Probleme verwendet werden können. Für das aufrechte Setzen der 
Einheit wird für diese außerdem \Macro{mathrm} verwendet. Als Alternative dazu 
könnte der Befehl \Macro{si}(\Package{siunitx}) aus dem Paket \Package{siunitx} 
genutzt werden. Für ein kleines Beispiel werden folgend fünf Formelzeichen 
definiert. Damit die Darstellung der mathematischen Brüche auch für den 
Fließtext gut genutzt werden kann und ansehnlich ist, wird für diese der Befehl 
\Macro{sfrac}(\Package{xfrac}) aus dem Paket \Package{xfrac} genutzt.
%
\begin{Trunk*}
\newformulasymbol{l}{Länge}{l}{m}
\newformulasymbol{m}{Masse}{m}{kg}
\newformulasymbol{a}{Beschleunigung}{a}{\sfrac{m}{s^2}}
\newformulasymbol{t}{Zeit}{t}{s}
\newformulasymbol{f}{Frequenz}{f}{s^{-1}}
\newformulasymbol{F}{Kraft}{F}{m \cdot kg \cdot s^{-2} = \sfrac{J}{m}}

\end{Trunk*}
%
Die soeben definierten Symbole werden für ein kleines Beispiel mit dem Befehl 
\Macro{gls}[\Parameter{Label}](\Package{glossaries}) in einer kurzen 
Textpassage verwendet.
%
\begin{Hint*}
Die Einheiten für die \gls{f} sowie die \gls{F} werden aus den 
SI"=Einheiten der Basisgrößen \gls{l}, \gls{m} und \gls{t} abgeleitet.
Und dann gibt es noch die Grundgleichung der Mechanik, welche für den
Fall einer konstanten Kraftwirkung in die Bewegungsrichtung einer
Punktmasse lautet:
\[\gls{F} = \gls{m} \cdot \gls{a}\]
\end{Hint*}
%
Das Ergebnis ist nur bedingt befriedigend. Sowohl im Fließtext als auch im 
Mathematikmodus werden lediglich die Bezeichnungen jedoch nicht die Symbole 
selbst verwendet. Damit die Formelzeichen auch für den mathematischen Satz 
sinnvoll nutzbar sind, sollte das Erscheinungsbild der Einträge mit 
\Macro{defglsentryfmt}(\Package{glossaries}) angepasst werden.
%
\begin{Preamble*}
\defglsentryfmt[symbols]{%
  \ifmmode%
    \glssymbol{\glslabel}%
  \else%
    \glsgenentryfmt~\glsentrysymbol{\glslabel}%
  \fi%
}
\end{Preamble*}
%
Bei der Verwendung von \Macro{gls}[\Parameter{Label}](\Package{glossaries}) 
führt diese Definition dazu, dass im Mathematikmodus allein das Symbol 
verwendet wird. Im Fließtext wird diesem zusätzlich die Bezeichnung 
vorangestellt. Das nachfolgende Beispiel macht dies deutlich.
%
\begin{Trunk*}
Die Einheiten für die \gls{f} sowie die \gls{F} werden aus den
SI"=Einheiten der Basisgrößen \gls{l}, \gls{m} und \gls{t} abgeleitet.
Und dann gibt es noch die Grundgleichung der Mechanik, welche für den
Fall einer konstanten Kraftwirkung in die Bewegungsrichtung einer
Punktmasse lautet:
\[\gls{F} = \gls{m} \cdot \gls{a}\]

\end{Trunk*}
%
Als nächstes kümmern wir uns um die Ausgabe des Symbolverzeichnisses. Momentan 
erzeugt der Befehl \Macro{printsymbols}(\Package{glossaries}) jedenfalls kein 
sinnvolles Verzeichnis:
%
\begin{Hint}
\printsymbols
\end{Hint}
\begin{quoting}[rightmargin=0pt]
\glsdisablehyper
\printsymbols[style=simple]
\end{quoting}
%
Für dieses muss erst ein Stil definiert werden, was nachfolgend ähnlich zum 
Stil \PValue{acrotabularx} respektive \PValue{acrotabu} geschieht. Allerdings 
wird hier eine Variante gezeigt, mit der die Tabelle einen Seitenumbruch 
zulässt.

\minisec{Eigener Stil mit \Environment*{tabularx} und \Environment*{longtable}}
Soweit mir bekannt ist, lassen sich umbruchfähige Tabellen nicht direkt mit 
\Environment{tabularx}(\Package{tabularx}) setzen. Vielmehr ist für die 
Verwendung der Umgebung \Environment{longtable}(\Package{longtable}) das Paket 
\Package{ltxtable} notwendig. Dieses wiederum verlangt, dass die zu setzende 
Tabelle in einer separaten Datei abgelegt wird. Soll diese dennoch innerhalb 
des Hauptdokumentes erstellt werden, kann für dieses Unterfangen die Umgebung 
\Environment{filecontents}(\Package{filecontents}) aus dem gleichnamigen Paket 
genutzt werden, womit \emph{bereits existierende Dateien} überschrieben werden.

Leider habe ich keine Möglichkeit gefunden, diesen Prozess ohne wahnsinnig 
großen Aufwand für das Erstellen eines Glossars zu portieren. Für umbruchfähige 
Tabellen mit automatisch berechneten Spaltenbreiten kommt meines Wissens nach 
momentan nur die Umgebung \Environment{longtabu}(\Package{tabu}) aus dem Paket 
\Package{tabu} infrage.

\minisec{Eigener Stil mit \Environment*{longtabu}}
Das Paket \Package{tabu} definiert für umbruchfähige Tabellen die Umgebung 
\Environment{longtabu}(\Package{tabu}), welche wiederum auf der Umgebung 
\Environment{longtable}(\Package{longtable}) basiert und nachfolgend verwendet 
wird. Damit diese linksbündig gesetzt wird, muss vor dem obligatorischen 
Argument mit den Spaltendefinitionen noch das optionale Argument \OParameter{l} 
angegeben werden. Die im nachfolgend definierten Stil verwendeten Befehle  
\Macro{toprule}(\Package{booktabs}), \Macro{midrule}(\Package{booktabs}) sowie
\Macro{cmidrule}(\Package{booktabs}) und \Macro{bottomrule}(\Package{booktabs}) 
für unterschiedliche horizontale Linien stammen allesamt aus dem Paket 
\Package{booktabs}, welches für den Satz von hochwertigen Tabellen eine große 
Hilfe ist.
%
\CodeHook{\let\newglossarystyle\renewglossarystyle}
\begin{Preamble*}
\newglossarystyle{symblongtabu}{%
  \renewenvironment{theglossary}{%
    \begin{longtabu}spread 0pt[l]{ccX<{\strut}l}%
  }{%
    \end{longtabu}%
  }%
  \renewcommand*{\glsgroupheading}[1]{}%
  \renewcommand*{\glsgroupskip}{}%
  \renewcommand*{\glossaryheader}{%
    \toprule
    \bfseries Formelzeichen & \bfseries Einheit &
    \bfseries Bezeichnung & \bfseries Seite(n)
    \tabularnewline\midrule\endhead%
    \bottomrule\endfoot%
  }%
  \renewcommand*{\glossentry}[2]{%
    \glsentryitem{##1}% Entry number if required
    \glstarget{##1}{\glossentrysymbol{##1}} &
    \glsentryuseri{##1} &
    \glossentryname{##1} &
    ##2\tabularnewline%
  }%
}
\end{Preamble*}
%
Innerhalb von \Macro{newglossarystyle}(\Package{glossaries}) wird der Befehl 
\Macro{glossaryheader}(\Package{glossaries}) für einen Tabellenkopf definiert, 
wie er auch für eine \Environment{longtable}(\Package{longtable})"=Umgebung 
erscheinen würde. In hier vorgestellten Fall werden der Kopf mit dem Makro 
\Macro{endhead}(\Package{longtable}) beziehungsweise der Fuß durch 
\Macro{endfoot}(\Package{longtable}) terminiert. Diese werden beim 
einem möglichen Seitenumbruch zu Beginn und am Ende auf jeder Seite gesetzt.
%
\begin{Preamble+}
}% Ende von AfterPackage

\end{Preamble+}
%
Das Symbolverzeichnis kann sich nun durchaus sehen lassen.
%
\begin{Hint}
\printsymbols[style=symblongtabu]
\end{Hint}
\begin{quoting}[rightmargin=0pt]
\newglossarystyle{symbtabu}{%
  \setglossarystyle{symblongtabu}%
  \renewenvironment{theglossary}{%
    \begin{tabu}spread 0pt{ccX<{\strut}l}%
  }{%
    \bottomrule\end{tabu}%
  }%
  \renewcommand*{\glossaryheader}{%
    \toprule
    \bfseries Formelzeichen & \bfseries Einheit &
    \bfseries Bezeichnung & \bfseries Seite(n)
    \tabularnewline\midrule%
  }%
}
\printsymbols[style=symbtabu]
\end{quoting}

\minisec{Unterteilung nach griechischen und lateinischen Symbolen}
Oftmals wird es für Abschlussarbeiten verlangt, das Symbolverzeichnis nach 
griechischen und lateinischen Formelzeichen zu untergliedern. Für die Umsetzung 
dieser Anforderung ist die zuvor beschriebene Lösung prinzipiell nutzbar, 
allerdings muss diese etwas erweitert werden.

Realisiert wird das Ganze, indem die zwei Kategorien respektive Elterneinträge 
\PValue{greekletters} sowie \PValue{romanletters} definiert und sämtliche
Formelzeichen einem der beiden als Untereintrag zugeordnet werden.
%
\begin{Hint*}
\providecommand*\greeklettersname{Greek letters}
\providecommand*\romanlettersname{Roman letters}
\newglossaryentry{greekletters}{%
  type=symbols,%
  name={\greeklettersname},%
  description={\nopostdesc},%
  sort={a}%
}
\newglossaryentry{romanletters}{%
  type=symbols,%
  name={\romanlettersname},%
  description={\nopostdesc},%
  sort={b}%
}
\end{Hint*}
%
Die Bezeichnungen der beiden Kategorien könnten auch als Parameter von 
\Macro*{newglossaryentry} mit \PValue{name=\PName{Bezeichung}} direkt 
eingetragen werden. Um bei der Definition jedoch eine gewisse Flexibilität zu 
gewährleisten, wird dies über die zwei Befehle \Macro*{greeklettersname} sowie
\Macro*{romanlettersname} realisiert. Diese werden mit \KOMAScript-Mitteln 
als sprachabhängige Bezeichner definiert.
%
\begin{Hint*}
\providecaptionname{%
  american,australian,british,canadian,english,newzealand%
}{\greeklettersname}{Greek letters}
\providecaptionname{%
  german,ngerman,austrian,naustrian,swissgerman,nswissgerman%
}{\greeklettersname}{Griechische Symbole}
\providecaptionname{%
  american,australian,british,canadian,english,newzealand%
}{\romanlettersname}{Roman letters}
\providecaptionname{%
  german,ngerman,austrian,naustrian,swissgerman,nswissgerman%
}{\romanlettersname}{Lateinische Symbole}
\end{Hint*}
%
Für die Ausgabe wird der Glossarstil \PValue{symbsplitlongtabu} in ähnlicher 
Manier wie die Stile zuvor definiert. Dieser wird darauf ausgelegt, dass alle 
aufzulistenden Formelzeichen als Untereintrag von einem der beiden zuvor 
erstellten Elterneinträge angegeben werden. Allerdings ist bei diesem Stil 
aufgrund der Verwendung von \Environment{longtabu}(\Package{tabu}) etwas 
Aufwand zu betreiben, um die gewünschte Ausgabe zu erhalten. Das grundlegende 
Problem ist das Setzen von Überschriften und dem folgenden Tabellenkopf 
innerhalb der Tabelle. Die Erläuterung des Quelltextauszugs erfolgt nach dessen 
Ausgabe.
%
\CodeHook{\let\newglossarystyle\renewglossarystyle}
\begin{Hint*}
\newglossarystyle{symbsplitlongtabu}{%
  \newcommand*\symbollevel{-1}%
  \renewenvironment{theglossary}{%
    \begin{longtabu}spread 0pt[l]{ccX<{\strut}l}%
  }{%
    \end{longtabu}%
  }%
  \renewcommand*{\glsgroupheading}[1]{}%
  \renewcommand*{\glsgroupskip}{}%
  \newcommand*\symbolhead{%
    \toprule
    \bfseries Formelzeichen & \bfseries Einheit &
    \bfseries Bezeichnung & \bfseries Seite(n)
    \tabularnewline\midrule
  }%
  \renewcommand*{\glossaryheader}{%
    \endfirsthead%
    \symbolhead\endhead%
    \bottomrule\endfoot%
    \gdef\symbollevel{-1}%
  }%
  \renewcommand*{\glossentry}[2]{%
    \ifglshaschildren{##1}{%
      \ifnum\symbollevel>0\relax%
        \tabularnewline\bottomrule\tabularnewline[\smallskipamount]%
      \fi
      \gdef\symbollevel{0}%
      \tabularnewline[%
        \arraystretch\dimexpr-\ht\strutbox-\dp\strutbox\relax%
      ]%
      \multicolumn{4}{@{}l@{}}{\minisec{\glsentryname{##1}}}%
    }{%
      \GlossariesWarning{%
        There are no childrens for entry ##1.\MessageBreak
        Nothing will be printed. Maybe you should\MessageBreak
        specify ##1 as a child entry.
      }%
    }%
  }%
  \renewcommand*{\subglossentry}[3]{%
    \ifnum\symbollevel=0\relax%
      \tabularnewline[\medskipamount]\symbolhead%
    \else%
      \tabularnewline
    \fi%
    \gdef\symbollevel{##1}%
    \glsentryitem{##2}% Entry number if required
    \glstarget{##2}{\glossentrysymbol{##2}} &
    \glsentryuseri{##2} &
    \glossentryname{##2} &
    ##3%
  }%
}
\end{Hint*}
%
Der neu definierte Befehl \Macro*{symbollevel} wird benötigt, um bei der 
Ausgabe zu überprüfen, um es sich beim zuletzt gesetzten Eintrag um einen 
Elterneintrag (\Macro*{glossentry}) oder Untereintrag (\Macro*{subglossentry}) 
gehandelt hat. Im Makro \Macro*{symbolhead} wird die Definition für den zu 
setzenden Tabellenkopf gespeichert. Der Inhalt von \Macro*{glossaryheader} wird 
direkt zu Beginn der Umgebung \Environment{theglossary}(\Package{glossaries}) 
ausgeführt. Dieser wird so umdefiniert, dass die Tabelle keinen Kopf vor der 
ersten Überschrift erhält, jedoch nach einem Seitenumbruch sehr wohl der 
Tabellenkopf gesetzt wird. Abgeschlossen wird die Tabelle mit einer Fußlinie.

Danach erfolgt die Formatierung der Einträge. Der Befehl \Macro*{glossentry} 
setzt \enquote{\greeklettersname} und \enquote{\romanlettersname} sowie 
gegebenenfalls weitere Elterneinträge als Überschriften. Bevor dies passieren 
kann wird eine vorher begonnene Tabelle mit einer Fußlinie beendet. Um die 
Überschrift innerhalb von \Environment{longtabu}(\Package{tabu}) linksbündig zu 
setzen, wird mit \Macro*{multicolumn} gearbeitet. Dieser Befehl muss in der 
ersten Zelle zwingend als aller erstes genutzt werden. Um dies sicherzustellen, 
wird zuvor eine neue Tabellenspalte eingefügt. Damit jedoch kein zusätzlicher 
Leerraum entsteht, wird diese über das optionale Argument nach oben verschoben.
Die Einträge für Formelzeichen selber erfolgen mit \Macro*{subglossentry}. Im 
Gegensatz zum vorher beschriebenen Stil \PValue{symblongtabu} wird zusätzlich 
der notwendige Tabellenkopf nach der Überschrift gesetzt, falls es sich um den 
\emph{ersten} Untereintrags handelt.

Jetzt könnte man den Glossarstil \PValue{symbsplitlongtabu} bereits verwenden 
und jedem Formelzeichen bei der Definition mit 
\begin{quoting}[leftmargin=\parindent]
\Macro{newformulasymbol}[%
  \POParameter{parent={greekletters/romanletters}}%
  \PParameter{\dots}\PParameter{\dots}\PParameter{\dots}\PParameter{\dots}%
]|?|%
\end{quoting}
eine der beiden Elterneinträge zuweisen. Für wenige Formelzeichen mag dies 
ausreichen. Komfortabler wäre es natürlich, wenn die Zuweisung zu einer der 
beiden Kategorien automatisch erfolgen würde. Ein Ansatz für dieses Unterfangen 
wird folgend vorgestellt. 

\Attention{%
  Es kann nicht garantiert werden, dass dieser Ansatz in jedem Fall und für 
  sämtliche Anwendungsszenarien funktioniert. Tatsächlich ist die Idee auf eine
  \hrfn{http://latex.wcms-file3.tu-dresden.de/phpBB3/viewtopic.php?f=11&t=427}{%
    Frage aus dem \Forum*%
  }
  heraus entstanden. Für auftretende Probleme bei der Verwendung ist dort der 
  beste Anlaufpunkt.
}

Die Grundidee ist, das zu definierende Formelzeichen mit einer Liste der Makros 
für griechische Buchstaben abzugleichen und so zu entscheiden, in welche der 
beiden Kategorien das Symbol einzuordnen ist. Hierfür werden nachfolgend eine 
Liste \Macro*{greeksymbollist} aller griechischen Buchstaben sowie der Befehl 
\Macro*{ifisgreeksymbol}[\Parameter{Symbol}\Parameter{Dann}\Parameter{Sonst}] 
zur Fallunterscheidung definiert. Problematisch bei der Umsetzung ist, dass ein 
als Argument übergebenes Formelzeichen auch mit diakritischen Zeichen und 
Indizes verwendet werden kann~-- beispielsweise $\dot{\varphi}_{\mathrm{max}}$
(%
  \Macro*{dot}[\PParameter{\Macro*{varphi}}]\PValue{\textunderscore}%
  \PParameter{\Macro*{mathrm}[\PParameter{max}]}%
). Deshalb werden bei der Ausführung von \Macro*{ifisgreeksymbol} temporär alle 
diakritische Zeichen sowie \Macro*{boldsymbol} für den Mathematikmodus 
\enquote{unschädlich} gemacht und das Argument \Parameter{Symbol} expandiert. 
Anschließend werden alle Anteile für Hoch- sowie Tiefstellung abgeschnitten und 
das Resultat mit der Liste \Macro*{greeksymbollist} abgeglichen.
%
\CodeHook{\let\newcommand\renewcommand}
\begin{Hint*}
\makeatletter
\newcommand*\greeksymbollist{}
\def\@tempa#1{\ifdefvoid{#1}{}{\listadd\greeksymbollist{#1}}}
\forcsvlist{\@tempa}{%
  \alpha,\beta,\varbeta,\gamma,\delta,\epsilon,\varepsilon,\zeta,%
  \eta,\theta,\vartheta,\iota,\kappa,\varkappa,\lambda,\mu,\nu,%
  \xi,\omicron,\pi,\varpi,\rho,\varrho,\sigma,\varsigma,\tau,%
  \upsilon,\phi,\varphi,\chi,\psi,\omega,%
  \Alpha,\Beta,\Gamma,\Delta,\Epsilon,\Zeta,\Eta,\Theta,\Iota,%
  \Kappa,\Lambda,\Mu,\Nu,\Xi,\Omicron,\Pi,\Rho,\Sigma,\Tau,\Upsilon,%
  \Phi,\Chi,\Psi,\Omega%
}
\newcommand*\ifisgreeksymbol[1]{%
  \begingroup%
    \def\@tempa##1{\let##1\@firstofone}%
    \forcsvlist{\@tempa}{%
      \acute,\bar,\breve,\check,\dot,\ddot,\dddot,\ddddot,%
      \hat,\widehat,\grave,\tilde,\widetilde,\vec,%
      \Acute,\Bar,\Breve,\Check,\Dot,\Ddot,\Hat,\Grave,\Tilde,\Vec%
    }%
    \def\boldsymbol{}%
    \protected@edef\@tempa{#1}%
    \def\@tempb##1_##2\relax{\def\@tempa{##1}}%
    \expandafter\@tempb\@tempa_\relax\relax%
    \def\@tempb##1^##2\relax{\def\@tempa{##1}}%
    \expandafter\@tempb\@tempa^\relax\relax%
    \expandafter\ifinlist\expandafter{\@tempa}{\greeksymbollist}{%
      \aftergroup\@firstoftwo%
    }{%
      \aftergroup\@secondoftwo%
    }%
  \endgroup%
}
\makeatother
\end{Hint*}
%
Mit \Macro*{ifisgreeksymbol} kann jetzt auf die Art des Symbols getestet. Dies 
wird innerhalb von \Macro{newformulasymbol}|?| genutzt, um den passenden 
Elterneintrag auszuwählen. Über das optionale Argument kann jederzeit mit 
\POParameter{parent={greekletters/romanletters}} der automatische ermittelte 
Elterneintrag überschrieben werden.
%
\begin{Hint}
\newcommand*\symbollettergroup{}
\newcommand*{\newformulasymbol}[5][]{%
  \ifisgreeksymbol{#4}{%
    \renewcommand*{\symbollettergroup}{greekletters}%
  }{%
    \renewcommand*{\symbollettergroup}{romanletters}%
  }%
  \newglossaryentry{#2}{%
    type=symbols,%
    name={#3},%
    description={\nopostdesc},%
    symbol={\ensuremath{#4}},%
    user1={\ensuremath{\mathrm{#5}}},%
    sort={#2},%
    parent={\symbollettergroup},%
    #1%
  }%
}
\end{Hint}


\subsection{Literaturverzeichnis}
\label{sec:biblatex}%
%
Für das Erstellen eines Literaturverzeichnisses wurde in der Vergangenheit fast 
ausschließlich \hologo{BibTeX} verwendet. Leider wird auch heute immer noch 
darauf verwiesen, obwohl es seit einigen Jahren das Paket \Package{biblatex} 
gibt, welches insbesondere für neue Dokumente den Vorzug erhalten sollte. Auch 
die Umstellung älterer \hologo{BibTeX}"=Datenbanken ist mit wenigen Handgriffen 
realisierbar. Für \Package{biblatex} existieren eine Menge unterschiedlicher, 
vordefinierter Zitierstile, welche sich im Vergleich zu \hologo{BibTeX} auch 
wesentlich leichter an die individuellen Bedürfnisse anpassen lassen.

Ein weiterer Vorteil ist die Unterstützung von Datenbanken, welche eine 
UTF"~8"~Kodierung nutzen, wenn \Application{biber} zur Sortierung der Einträge 
verwendet wird. Zitierstil und Backend zur Sortierung lassen sich durch das 
optionale Argument beim Laden des Paketes festlegen. Damit die Zitierstile das 
optimale Ergebnis erzielen wird das Laden von \Package{csquotes} sehr empfohlen.
%
\begin{Preamble}
\usepackage{csquotes}
\usepackage[backend=biber,style=alphabetic]{biblatex}

\end{Preamble}
%
Die Erstellung einer Literaturdatenbank kann entweder von Hand oder mithilfe 
einer externen Anwendung erfolgen. Für die letztgenannte Variante sind die 
Programme \Application{Citavi} respektive \Application{JabRef} empfehlenswert. 
Auf eine Einführung in diese Anwendungen wird jedoch verzichtet. 

Die \Environment{filecontents}(\Package{filecontents})"=Umgebung kann verwendet 
werden, um innerhalb eines \hologo{LaTeX}"=Dokumentes externe Dateien direkt 
beim Kompilieren zu erstellen. Damit wird nachfolgend für dieses Tutorial eine 
Literaturdatenbank \File{\jobname-temp.bib} mit drei Einträgen manuell erzeugt. 
Die Umgebung gehört standardmäßig zu den Bordmitteln von \hologo{LaTeXe}. Das 
Paket \Package{filecontents} erweitert die Umgebung dahingehend, dass 
\emph{bereits existierende Dateien} überschrieben werden. Hier ist folglich 
Vorsicht geboten. Der große Vorteil ist jedoch, dass die erweiterte 
\Environment{filecontents}(\Package{filecontents})"=Umgebung~-- im Gegensatz 
zur Standardversion~-- die Dateien in der gleichen Eingabekodierung erzeugt, 
wie das verwendete Dokument. Diese Funktionalität wird für dieses Tutorial 
benötigt, weshalb auf das Laden des Paketes \Package{filecontents} nicht 
verzichtet werden kann. 
%
\begin{Preamble}
\usepackage{filecontents}
\end{Preamble}
%
Folgend wird eine Literaturdatenbank mit drei Einträge definiert. Jeder Eintrag 
einer \File*{.bib}"~Datei beginnt mit \PValue{@}\PName{Eintragstyp}. Direkt 
danach ist für jeden Eintrag ein \emph{eindeutiges} \PParameter{Label} 
festzulegen. Anschließend können für unterschiedliche Felder die dazugehörigen 
Werte eingetragen werden. Die verwendbaren Eintragstypen sowie die für den 
jeweiligen Typ obligatorischen und optionalen Felder sind in der Dokumentation 
von \Package{biblatex} zu finden.
%
\begin{Preamble*}
\begin{filecontents}{\jobname-temp.bib}
@book{goossens94,
  author    = {Goossens, Michel and Mittelbach, Frank
               and Samarin, Alexander},
  title     = {The LaTeX Companion},
  date      = {1994},
  publisher = {Addison-Wesley},
  location  = {Reading, Massachusetts},
  language  = {english},
}
@book{knuth84,
  author    = {Knuth, Donald E.},
  title     = {The \TeX book},
  date      = {1984},
  maintitle = {Computers \& Typesetting},
  volume    = {A},
  publisher = {Addison-Wesley},
  location  = {Reading, Massachusetts},
  language  = {english},
}
@manual{hanisch14,
  author    = {Hanisch, Falk},
  title     = {Ein \LaTeX"=Bundle für Dokumente im neuen Corporate 
               Design der Technischen Universität Dresden},
  date      = {2014},
  subtitle  = {Benutzerhandbuch},
  location  = {Dresden},
  language  = {german},
}
\end{filecontents}
\end{Preamble*}
%
Nachdem die Literaturdatenbank erstellt wurde, muss diese auch noch eingebunden 
werden:
%
\begin{Preamble}
\addbibresource{\jobname-temp.bib}

\end{Preamble}
%
Im einfachsten Fall werden die gewünschten Einträge der Literaturdatenbank im 
Dokument mit \Macro{cite}[\Parameter{macro}](\Package{biblatex}) referenziert, 
von \Package{biblatex} werden zusätzliche Möglichkeiten angeboten.
%
\begin{refsection}[\jobname-temp.bib]
\begin{Trunk+}
\section{Referenzen und das Literaturverzeichnis}
Das Literaturverzeichnis wird auf Basis der nachfolgend verwendeten 
Zitate erstellt und ist auf Seite~\pageref{sec:bibliography} zu finden.
\end{Trunk+}
\begin{Trunk*}
In diesem Textabschnitt werden die zwei bekannten \LaTeX-Bücher
\cite{knuth84} und \cite{goossens94} sowie das Anwenderhandbuch
\cite{hanisch14} zitiert.

\end{Trunk*}
%
Das Literaturverzeichnis wird mit \Macro{printbibliography}(\Package{biblatex}) 
ausgegeben, wobei nicht alle Einträge der Literaturdatenbank sondern lediglich 
die tatsächlich referenzierten verwendet werden. Der Parameter 
\PValue{heading=bibintoc} führt zu einem Eintrag in das Inhaltsverzeichnis.
%
\begin{Hint}
\printbibliography[heading=bibintoc]
\end{Hint}
\begin{quoting}[rightmargin=0pt]
\makeatletter
\let\markboth\@gobbletwo
\let\markright\@gobble
\makeatother
\printbibliography
\end{quoting}
\end{refsection}



\section{Anfangszitat oder Schlauer Spruch}
\renewcommand*\dictumwidth{.4\textwidth}%
\dictum[Johann Wolfgang von Goethe]{Es irrt der Mensch, solang er strebt.}%
\bigskip\noindent
Oftmals möchte der Autor einer wissenschaftlichen Arbeit für das erste oder 
auch jedes Kapitel ein Zitat oder ähnliches voranstellen. Dies kann mit dem 
Befehl \Macro{dictum}[\OParameter{Autor}\Parameter{Text}](\Package{koma-script})
erfolgen. Damit wird der im obligatorischen Argument angegeben Ausspruch in 
einer \Macro{parbox} ausgegeben. Das optionale Argument kann für die Angabe des 
Autors verwendet werden. Soll das Ganze für einen Teil oder ein Kapitel 
erfolgen, sollte der Befehl \Macro{dictum}(\Package{koma-script}) innerhalb von 
\Macro{setpartpreamble}(\Package{koma-script}) beziehungsweise 
\Macro{setchapterpreamble}(\Package{koma-script}) verwendet werden. Genaueres 
hierzu und zu den weiteren Möglichkeiten, die Gestalt eines Zitats zu 
beeinflussen, ist in der \scrguide[Anleitung von \KOMAScript] zu finden. Es 
folgt ein Beispiel zur Verwendung.
%
\begin{Hint}
\setchapterpreamble{%
  \dictum[Johann Wolfgang von Goethe]{%
    Es irrt der Mensch, solang er strebt.%
  }%
  \bigskip
}
\chapter{Einleitung}

\end{Hint}



\section{Gleitumgebungen für Abbildungen und Tabellen}
\label{sec:floats}%
%
Die Positionierung von Abbildungen mit \hologo{LaTeX} kann zu Beginn für viele 
Anfänger durchaus frustrierend sein. Das liegt häufig am Missverständnis der 
beiden Standard"=Gleitobjektumgebungen für Tabellen (\Environment{table}) und 
Abbildungen (\Environment{figure}). Diese sind in erster Linie zur Ergänzung 
des Fließtextes gedacht und sollten für das prinzipielle Verständnis des 
Geschriebenen nicht notwendig sein. Das oft geforderte Verhalten, ein 
Gleitobjekt an einer ganz bestimmten und explizit festgelegten Position im Text 
zu setzen, ist nicht erforderlich, insbesondere weil dadurch der Lesefluss 
unnötig unterbrochen wird. 

Vielmehr ist es sinnvoll, Gleitobjekte entweder am Anfang oder Ende einer Seite 
zu platzieren, wo sie den Lesefluss deutlich weniger stören. Allerdings sollte 
auf jedes Gleitobjekt im Fließtext über eine Referenz~-- beispielsweise mit dem 
Befehl \Macro{autoref}(\Package{hyperref}) aus dem Paket \Package{hyperref}~-- 
Bezug genommen und gegebenenfalls eine kurze Erläuterung gegeben werden. 
Zusätzliche Erläuterungen zum Thema \emph{Querverweise} sind unter 
\autoref{sec:references} zu finden.

Ein weitere Grund, \hologo{LaTeX} die Platzierung von Tabellen und Abbildungen 
vollständig zu überlassen, ist die Ungewissheit über den vorhandenen Platz auf 
der momentan erzeugten Seite. In nicht wenigen Fällen kann es passieren, dass 
das einzufügende Objekt zu groß für die aktuelle Seite ist, was einen wahrlich 
schlechten Seitenumbruch mit einer schlecht gefüllten Seite zur Folge hätte.
Die Verwendung einer Gleitumgebung für eine Abbildung wird im nachfolgenden 
Quelltextauszug exemplarisch gezeigt, das Ergebnis ist in \autoref{fig:example} 
zu sehen. Bei Gleitobjekten sollte in jedem Fall darauf geachtet werden, dass 
der Befehl \Macro{label} immer erst \emph{nach} 
\Macro{caption}(\Package{koma-script},\Package{caption}) verwendet wird, da 
der erzeugte Anker sich sonst nicht auf das Objekt bezieht.
%
\begin{Trunk+}
\section{Grafiken und Tabellen in Gleitumgebungen}
Es folgt die Demonstration von Gleitumgebungen, welche sowohl für 
Grafiken als auch Tabellen verwendet werden sollten. Im vorliegenden 
Beispiel kann unter Umständen der Eindruck entstehen, dass diese Seite 
etwas zu überladen mit Gleitobjekten ist. Dies liegt nicht an der 
Verwendung der Gleitobjekte sondern vielmehr am zu geringen Textvolumen 
und den eingeschränkten Möglichkeiten von \LaTeX{}, diese an geeigneten 
Stellen zu platzieren. 

\subsection{Abbildungen als Gleitobjekte und das Einbinden von Grafiken}
In \autoref{fig:example} wird dargestellt, wie eine Grafik im PDF"~Format 
in ein Dokument eingebunden und auf diese verwiesen werden kann. Ein 
Querverweis auf ein Gleitobjekt sollte im Fließtext am besten mit Befehl 
\texttt{\textbackslash autoref\{\emph{<Label>}\}} erstellt werden. 
Hierfür ist ein entsprechender Anker am zu referenzierenden Objekt nötig, 
welcher mit dem Makro \texttt{\textbackslash label} erzeugt wird. Dabei 
ist entscheidend, dass dieser Anker erst \emph{nach} der Beschriftung des 
Objektes, welche mit \texttt{\textbackslash caption} zu erstellen ist, 
definiert werden sollte.

\end{Trunk+}
\begin{Trunk}
\begin{figure}
\centering
\includegraphics{TUD-black}
\caption{Beispielgrafik}\label{fig:example}
\end{figure}

\end{Trunk}
\InputCode\noindent
%
In der Konsequenz ist dies hier ein Plädoyer, bei der Platzierung von Tabellen 
und Abbildungen vollständig auf \hologo{LaTeX} zu vertrauen. Anfangs kann der 
Anwender dies als Kontrollverlust empfinden. Sobald Sie jedoch Änderungen am 
Dokument vornehmen und eventuell in einem Kapitel einen Absatz ergänzen oder 
gar einen ganzen Abschnitt hinzufügen werden sie dankbar sein, sich in der 
Folge nicht mit der Neupositionierung sämtlicher Objekte herumschlagen zu 
müssen. Leidgeprüfte Anwender einschlägiger Textverarbeitungsprogramme können 
gewiss ein Lied darüber singen.



\subsection{Gleitobjektlayout}
\label{sec:floatlayout}%
%
Es wurde bis jetzt das prinzipielle Vorgehen bei der Nutzung von Gleitobjekten 
beschrieben. Allerdings gibt es auch noch einige typografische Aspekte, welche 
zu beachten sind. Zum einen sollte beachtet werden, dass die Beschriftung einer 
Abbildung immer unterhalb dieser erfolgen sollte, bei Tabellen hingegen eine 
Überschrift gesetzt wird. Hierfür muss der Anwender ohne die Verwendung eines 
zusätzlichen Paketes selber Sorge tragen, indem er den Befehl 
\Macro{caption}(\Package{koma-script},\Package{caption}) entweder vor oder 
nach dem eigentlichen Objekt in der Gleitumgebung verwendet. Der zweite Punkt 
ist die verwendete Schrift innerhalb der Gleitumgebungen. 

Damit sich der Inhalt dieser besser vom restlichen Fließtext abhebt und vom 
Leser direkt als nicht dazugehörig erkannt werden kann ist es ratsam, diesen in 
serifenloser Schrift zu setzen~-- die Verwendung einer Serifenschrift für den 
Fließtext vorausgesetzt. Das Paket \Package{floatrow} bietet die Möglichkeiten, 
diese beiden Punkte automatisiert umzusetzen. Außerdem wird zur Formatierung 
der Beschriftungen das Paket \Package{caption} benötigt. Sollten Sie noch keine 
Erfahrung mit dem Setzen von Tabellen mit \hologo{LaTeX} haben, so wäre zuvor 
ein Blick in \autoref{sec:tables} sehr sinnvoll. 

\newcommand*\tableexample[1][]{%
  \begin{tabularx}{.75\textwidth}{@{}XXX@{}}
  \toprule
  \textbf{Erste Spalte} & \textbf{Zweite Spalte} & 
  \textbf{Dritte Spalte} \tabularnewline\midrule
  Etwas Blindtext für die erste Spalte &
  Etwas Blindtext für die zweite Spalte &
  Etwas Blindtext für die dritte Spalte
  \tabularnewline\bottomrule
  \end{tabularx}
  #1%
}
\begin{table}
\tableexample[%
  \caption{Eine Tabelle in einer Gleitumgebung}\label{tab:tabular}%
]
\end{table}

Normalerweise wird eine Tabelle sowie deren Beschriftung innerhalb einer 
\Environment{table}"=Gleitumgebung in der selben Schrift wie der Fließtext 
gesetzt, die korrekte Platzierung der Bezeichnungen bleibt dem Anwender 
überlassen. Zu sehen ist dies in \autoref{tab:tabular}. Im Folgenden wird 
gezeigt, wie sich dies automatisiert für alle Gleitobjekte im Dokument ändern 
lässt. Die \emph{Formatierung} der Bezeichnungen von Gleitobjekten kann 
mithilfe des Paketes \Package{caption} angepasst werden.
%
\begin{Preamble}
\usepackage{caption}
\end{Preamble}
%
Diese stellt den Befehl \Macro{captionsetup}(\Package{caption}) bereit. Mit 
diesem werden die Beschriftungen in Serifenlosen und das Label zur besseren 
Unterscheidung im fetter Schriftstärke gesetzt.
%
\begin{Preamble*}
\captionsetup{font=sf,labelfont=bf,labelsep=space}
\end{Preamble*}
%
Um die \emph{Platzierung} der Beschriftungen beeinflussen sowie den Inhalt 
einer Gleitumgebung formatieren zu können, wird \Package{floatrow} geladen. 
\begin{Preamble}
\usepackage{floatrow}
\end{Preamble}
%
Der Inhalt von Gleitobjekten wird~-- entgegen des normalen Verhaltens~-- durch 
\Package{floatrow} automatisch zentriert gesetzt. Zusätzlich soll der Inhalt 
noch die serifenlose Schriftfamilie verwenden. Dies lässt sich mit der Nutzung 
des Befehls \Macro{floatsetup}(\Package{floatrow}) realisieren.
%
\begin{Preamble*}
\floatsetup{font=sf}
\end{Preamble*}
%
Ohne weitere Maßnahmen werden durch \Package{floatrow} alle Beschriftungen 
unterhalb der Gleitobjekte ausgegeben. Da das Ziel jedoch die Verwendung von 
Tabellenüberschriften und Abbildungsunterschriften ist, wird dies für Tabellen
angepasst.
%
\begin{Preamble*}
\floatsetup[table]{style=plaintop}
\end{Preamble*}
%
\begin{table}
\tableexample[%
  \caption{%
    Eine Tabelle in einer Gleitumgebung mit einer angepassten Formatierung%
  }%
  \label{tab:tabularnew}%
]
\end{table}%
%
Damit wären alle Einstellungen soweit erfolgt, in \autoref{tab:tabularnew} ist 
das Ergebnis zu sehen. Zumeist soll die Beschriftung für Gleitobjekte nicht 
zentriert sondern linksbündig gesetzt werden. Dies lässt sich ebenfalls mit den 
Mitteln des Paketes \Package{caption} und der passenden Paketoption 
\Option{justification}(\Package{caption}) erreichen. Damit dies auch für 
einzeilige Beschriftungen angewendet wird, muss zusätzlich die Einstellung 
\Option{singlelinecheck}(\Package{caption}) deaktiviert werden. Mit der 
Paketoption \Option{format}(\Package{caption}) kann außerdem beeinflusst 
werden, ob eine mehrzeilige Beschriftung nach einem Zeilenumbruch direkt unter 
dem Label oder mit Einzug fortgesetzt werden soll.
%
\begin{Preamble*}
\captionsetup{singlelinecheck=off,format=hang,justification=raggedright}
\end{Preamble*}
%
In \autoref{tab:tabularnew} fällt die Beschriftung auf, welche die Breite der 
Tabelle überragt. Falls linksbündige Gleitumgebungsbezeichnungen verwendet 
werden, wie dies gerade eingestellt wurde, besteht das Problem quasi bei jeder 
Beschriftung, da diese immer am linken Seitenrand beginnen. Auch hierfür stellt 
das Paket \Package{floatrow} die Befehle \Macro{ttabbox}(\Package{floatrow}) 
für Tabellen sowie \Macro{ffigbox}(\Package{floatrow}) für Abbildungen bereit. 

Beide Makros erwarten im ersten obligatorischen Argument den Befehl 
\Macro{caption}(\Package{koma-script},\Package{caption}) für die Beschriftung 
gegebenenfalls gefolgt von einem mit \Macro{label} gesetzten Textanker. Im 
zweiten Argument wird das Objekt selbst~-- sprich Tabelle oder Abbildung~-- 
angegeben. Zu verwenden sind diese folgendermaßen:
%
\begin{Hint}
\begin{table}
\ttabbox
  §\Parameter{Tabellendefinition}§
  {\caption§\Parameter{Beschriftung}§\label§\Parameter{Label}§}
\end{table}
\begin{figure}
\ffigbox
  §\Parameter{Abbildungsdefinition}§
  {\caption§\Parameter{Beschriftung}§\label§\Parameter{Label}§}
\end{figure}
\end{Hint}
%
Diese beiden Befehle sorgen dafür, dass die Beschriftungen in der jeweiligen 
Breite des Objektes gesetzt werden. Entspricht das zu setzende Objekt in seiner 
Breite dem Wert \Length{textwidth}~-- es ist genauso Breit wie die Laufweite 
des Fließtextes~--, so ist deren Gebrauch nicht notwendig. Das Ergebnis der 
Verwendung von des Befehls \Macro{ttabbox}(\Package{floatrow}) ist in 
\autoref{tab:tabularbox} zu sehen.

Bei der Nutzung der beiden Befehle sollte darauf geachtet werden, dass zwischen 
dem Ende des Objektes und der schließender Klammer des zweiten Argumentes kein 
ungewolltes Leerzeichen gesetzt wird. Dies lässt sich leicht vermeiden, indem 
direkt nach dem Ende des Objektes mit \Macro*{end}[\PParameter{tabularx}], 
\Macro*{end}[\PParameter{tabu}], \Macro*{end}[\PParameter{tikz}] etc. die 
schließende Klammer des zweiten Argumentes oder das Zeichen für einen Kommentar 
\PValue{\%} folgt.

\begin{table}
\ttabbox{\tableexample}{%
  \caption[%
    Eine mit \Macro{ttabbox}(\Package{floatrow}) gesetzte Tabelle in einer 
    Gleitumgebung%
  ]{%
    Eine in einer Gleitumgebung gesetzte Tabelle in Verbindung mit dem Befehl 
    \Macro{ttabbox}(\Package{floatrow}), der durch das Paket \Package{floatrow} 
    für Beschriftungen in Objektbreite zur Verfügung gestellt wird%
  }%
  \label{tab:tabularbox}%
}
\end{table}


\subsection{Untergleitobjekte}
\label{sec:subfloats}%
%
Das Paket \Package{floatrow} stellt allerhand Möglichkeiten zur Erstellung von 
nebeneinanderliegenden Gleitobjekten sowie \enquote{Untergleitobjekten} bereit 
und ist insbesondere in Verbindung mit den Fähigkeiten von \Package{caption} 
respektive \Package{subcaption} sehr gut für dieses Unterfangen nutzbar. Dafür 
werden die Umgebungen \Environment{floatrow}(\Package{floatrow}) sowie 
\Environment{subfloatrow}(\Package{floatrow}) angeboten, wobei die letztere für 
ein Beispiel verwendet wird. Als erstes muss der Typ der zu erzeugenden Labels 
für Abbildungen und Tabellen definiert werden. Dies geschieht mit dem Befehl 
\Macro{DeclareCaptionSubType}(\Package{subcaption}) aus dem Paket 
\Package{subcaption}.
%
\begin{Preamble}
\DeclareCaptionSubType[alph]{figure}
\DeclareCaptionSubType[alph]{table}
\end{Preamble}
%
Anschließend werden die Untergleitobjekte formatiert. Dabei wird zum einen nach 
dem alphanumerischen Label eine schließende Klammer gesetzt. Zum andern wird 
ein Eintragen der Untergleitobjekte in das Abbildungsverzeichnis unterdrückt. 
Lediglich die Bildunterschrift der Hauptabbildung wird in dieses eingetragen.
%
\begin{Preamble*}
\captionsetup[subfloat]{labelformat=brace,list=off}

\end{Preamble*}
%
Mit diesen Einstellungen wird eine gleitende Abbildung erstellt, welche in 
\autoref{fig:logos} zu sehen ist. Wurde der entsprechende Anker gesetzt, kann 
zusätzlich auf die untergeordnete \autoref{fig:tud} sowie \autoref{fig:ddc} 
verwiesen werden.
%
\begin{Trunk+}
\subsection{Untergleitobjekte}
Nachdem nun schon eine gleitende Abbildung und zwei gleitende Tabellen 
erstellt wurden, folgt jetzt noch eine gleitende Abbildung mit zwei 
Unterabbildungen. Durch die drei gesetzten Anker kann im Fließtext 
sowohl auf \autoref{fig:logos} als auch auf \autoref{fig:tud} sowie 
\autoref{fig:ddc} verwiesen werden.

\end{Trunk+}
\begin{Trunk}
\begin{figure}
\ffigbox[\FBwidth]%
  {\begin{subfloatrow}%
    \ffigbox[\FBwidth]%
      {\fbox{\includegraphics[height=2cm]{TUD-black}}}%
      {\caption{Eine Abbildung}\label{fig:tud}}%
    \ffigbox[\FBwidth]%
      {\fbox{\includegraphics[height=2cm]{DDC-21}}}%
      {\caption{Eine weitere Abbildung}\label{fig:ddc}}%
  \end{subfloatrow}}%
  {\caption{Eine Gleitumgebung mit zwei Abbildungen}\label{fig:logos}}%
\end{figure}

\end{Trunk}
\InputCode


\subsection{Beeinflussung des Gleitprozesses}
Sobald das Dokument inhaltlich den finalen Zustand erreicht hat, haben Sie sich 
zu diesem Zeitpunkt schon mit Sicherheit an das standardmäßige Vorgehen von 
\hologo{LaTeX} gewöhnt und müssen respektive wollen gegebenenfalls nur noch bei 
wenigen Gleitobjekten~-- über das optionale Argument der Gleitumgebungen zur
individuellen Empfehlung für die Platzierung~-- nachjustieren. Mögliche Werte 
sind:
%
\begin{description}[labelindent=\parindent,leftmargin=*,style=nextline]
\item[\POParameter{h} (here)]
  An der Stelle, wo es im Quelltext angegeben wurde~-- falls genügend Platz 
  vorhanden ist
\item[\POParameter{t} (top)]
  Am oberen Ende der aktuellen oder der folgenden Seite
\item[\POParameter{b} (bottom)]
  Am unteren Ende der aktuellen Seite
\item[\POParameter{p} (page)]
  Auf einer separaten Seite für mindestens ein Gleitobjekt
\end{description}
%
Sie können eine, mehrere oder alle Optionen angeben, wobei die Reihenfolge 
keine Rolle spielt. Der Algorithmus arbeitet alle ihm zur Verfügung gestellten 
Optionen immer in der zuvor aufgezählten Reihenfolge ab, wobei diese nur als 
Empfehlung und nicht als Verpflichtung angesehen werden. Es gibt folglich keine 
Garantie, dass Ihr Vorschlag akzeptiert wird. Fügen sie dem optionalen Argument 
ein~\PValue{!} an, so verhindern Sie, dass \hologo{LaTeX} weitere Optionen 
evaluiert. Das Gleitobjekt wird anhand der optionalen Parameter positioniert, 
selbst wenn dabei ein unschönes Seitenlayout entsteht. Kann das Gleitobjekt 
technisch unmöglich auf die angegebene Weise positioniert werden, wird dieses 
sowie alle folgenden aufgeschoben und am Ende des Abschnitts oder Kapitels 
angehängt. Dieser Effekt ist wohl in den seltensten Fällen gewollt. 

Außerdem können die Pakete \Package{flafter} sowie \Package{placeins} genutzt 
werden. Das erstgenannte verhindert das Auftreten von Gleitobjekten im Dokument 
vor ihrer Definition im Quelltext. In der Konsequent bedeutet dies, dass die 
Option \POParameter{t} Gleitobjekte nur am oberen Ende der nächsten Seite 
jedoch nicht auf der aktuellen zulässt. Mit dem zweiten Paket können Barrieren 
definiert werden, an welchen die Ausgabe aller noch in der Warteschlange 
befindlichen Gleitobjekte forciert wird~-- beispielsweise vor bestimmten 
Gliederungsüberschriften. Im Handbuch zu \TUDScript findet sich außerdem zum 
Thema \enquote{\tudhyperref{sec:tips:floats}{Platzierung von Gleitobjekten}} 
ein eigener Abschnitt mit weiterführenden Informationen.


\subsection{Abstellen des Gleitprozesses}
Trotz der zuvor genannten Vorteile von Gleitobjekten wird oftmals verlangt, den 
Gleitprozess vollständig abzustellen. Dies bringt jedoch in den meisten Fällen 
einige Probleme mit sich. Die Folge sind ein sehr unruhiges Erscheinungsbild 
des Satzspiegels aufgrund schlecht gefüllter Seiten, außerdem im Zweifelsfall 
viel Handarbeit mit hart kodierten Seitenumbrüchen sowie daraus resultierend 
ein gegenüber von Umbruchänderungen äußerst anfälliges Dokument. Sollte es in 
\emph{Einzelfällen} dennoch erforderlich sein, dass eine Abbildung oder Tabelle 
nicht gleitet, ist die einzig logische Konsequenz, auf die Verwendung einer 
Gleitumgebung gänzlich zu verzichten. 

Hierfür eignet sich beispielsweise die Umgebung \Environment{center}, welche 
zum einen etwas Abstand vor und nach dem Objekt zum umgebenden Text erzeugt 
und zum anderen die Grafik respektive Tabelle und die dazugehörige Bezeichnung 
zusammenhält. \KOMAScript{} stellt für diesen Anwendungsfall das Makro 
\Macro{captionof}(\Package{koma-script}) für die Beschriftung zur Verfügung. 
Werden die Pakete \Package{caption} und \Package{hyperref} geladen, sollte 
weiterhin \Macro{caption}(\Package{koma-script},\Package{caption}) ergänzt um 
ein zuvor aufgerufenes  
\Macro{captionsetup}[\PParameter{type=figure/table}](\Package{caption})
genutzt werden. Damit lassen sich Bildunter- sowie Tabellenüberschriften auch 
ohne die bekannten Gleitumgebungen \Environment{figure} respektive 
\Environment{table} erzeugen. Das Vorgehen wird folgend demonstriert, in 
\autoref{fig:nonfloating-A} ist das Ergebnis zu sehen.
%
\begin{Hint}
\begin{center}
\captionsetup{type=figure}
\includegraphics{TUD-black}
\caption{Eine nichtgleitende Grafik in einer \texttt{center}"=Umgebung}
\label{fig:nonfloating-A}
\end{center}

\end{Hint}
\InputCode\noindent
%
Alternativ zu diesem Ansatz kann das Paket \Package{float} zum Einsatz kommen, 
welches für Gleitumgebungen den Platzierungsparameter \POParameter{H} 
bereitstellt. Das Resultat ist identisch zu \autoref{fig:nonfloating-A}.
%
\begin{Hint}
\begin{figure}[H]
\includegraphics{TUD-black}
\caption{Eine nichtgleitende Grafik in einer \texttt{figure}"=Umgebung}
\label{fig:nonfloating-B}
\end{figure}

\end{Hint}
%
Mit den Mitteln des Paketes \Package{floatrow} lässt sich eine etwas bessere 
Variante entwickeln, wobei für eine mehrmalige Anwendung ein eigenes Makro mit 
den passenden Argumenten definiert werden sollte. Zum besseren Verständnis des 
Quelltextauszuges ist die Dokumentation von \Package{floatrow} zu beachten.
%
\begin{Hint}
\floatbox[\captionsetup{type=figure}]{figure}[\FBwidth][%
  \dimexpr\FBheight+\abovedisplayskip\relax%
][b]{\includegraphics{TUD-black}}{%
  \caption{%
    Eine nichtgleitende Grafik mit \texttt{\textbackslash floatbox}%
  }%
  \label{fig:nonfloating-C}%
}
\end{Hint}
\InputCode\noindent
%
Es wird hier trotz der vorhergehenden Ausführungen abermals ausdrücklich 
empfohlen, Tabellen und Abbildungen in Gleitumgebungen zu setzen und die 
korrekte Platzierung dem \hologo{LaTeX}"=Algorithmus zu überlassen.




\section{Tabellensatz}
\label{sec:tables}%
%
Zum Thema \enquote{Tabellensatz mit \hologo{LaTeX}} sind bereits zahlreiche 
\hrfn{http://userpage.fu-berlin.de/latex/Materialien/tabsatz.pdf}{Leitfäden} 
\cite{reichert2012} im Internet zu finden. Deshalb werde ich meine Ausführungen 
zu diesem Punkt relativ kurz halten. Zwei Regeln sollten beim Satz von Tabellen 
in jedem Fall beachtet werden:
%
\begin{enumerate}[itemindent=0pt,labelwidth=*,labelsep=1em,label=\Roman*.,noitemsep]
\item keine vertikalen Linien
\item keine doppelten Linien
\end{enumerate}
%
Das Paket \Package{booktabs} ist für den Satz von hochwertigen Tabellen eine 
große Hilfe und stellt die Befehle \Macro{toprule}(\Package{booktabs}), 
\Macro{midrule}(\Package{booktabs}) sowie \Macro{cmidrule}(\Package{booktabs}) 
und \Macro{bottomrule}(\Package{booktabs}) für unterschiedliche horizontale 
Linien bereit.
%
\begin{Preamble}
\usepackage{booktabs}
\end{Preamble}
%
Außerdem existiert das Paket \Package{array}, welches mit dem Befehl 
\Macro{newcolumntype}(\Package{array}) die Definition eigener Spaltentypen 
sowie die Verwendung von \enquote{Hooks} vor (\PValue{>\PParameter{\dots}}) und 
nach (\PValue{<\PParameter{\dots}}) Einträgen innerhalb einer Spalte 
(\PValue{>\PParameter{\dots}}\PName{Spaltentyp}\PValue{<\PParameter{\dots}}) 
ermöglicht.
%
\begin{Preamble}
\usepackage{array}
\end{Preamble}
%
Für alle im Folgenden vorgestellten Umgebungen zum Setzen von Tabellen gilt, 
dass die Inhalte zeilenweise angegeben werden, wobei die Einträge für die 
einzelnen Spalten mit~\PValue{\&} voneinander zu trennen sind. Das Beenden 
einer Tabellenzeile und der Wechsel zur nächsten erfolgt normalerweise mit 
\PValue{\textbackslash\textbackslash}. Da jedoch einige \hologo{LaTeX}"=Pakete 
diesen Befehl innerhalb von Tabellen lokal ändern ist es zur Vermeidung 
unnötiger Fehler wesentlich sicherer, das Zeilenende ausschließlich mit 
\Macro{tabularnewline} zu setzen.

Im Folgenden wird in einigen Tabellen bei der Definition der Spalten vor der 
ersten und nach der letzten das Konstrukt~\PValue{@\PParameter{}} verwendet, 
was das \tudhyperref{sec:tips:table}{Einrücken der Tabellenspalten verhindert} 
und im Handbuch genauer erläutert wird. 


\subsection{Die Standardumgebung \Environment*{tabular}}
Normalerweise gibt es für \Environment{tabular} vier unterschiedliche 
Spaltentypen. Die drei Spaltentypen~\PValue{l},~\PValue{c}~und~\PValue{r} 
stehen für linksbündige, zentrierte und rechtsbündige Spalten zu Verfügung, 
welche allerdings keinen Zeilenumbruch erlauben. Der Inhalt wird quasi wie in 
einer \Macro{mbox} gesetzt, wodurch die Spalten sehr breit werden und über den 
Seitenrand hinausragen können.

Der Spaltentyp~\PValue{p}\Parameter{Breite} hingegen legt die Spaltenbreite 
fest und setzt die Zellinhalte in einer \Macro{parbox}, wobei diese mit ihrer 
obersten Zeile an der Grundlinie ausgerichtet wird. Das Paket \Package{array} 
stellt außerdem die Spaltentypen~\PValue{m}\Parameter{Breite} 
und~\PValue{b}\Parameter{Breite} bereit, welche ebenfalls mit \Macro{parbox} 
gesetzt werden, die Ausrichtung an der Grundlinie jedoch zentriert respektive 
an der unteren Zeile erfolgt. Es folgt ein Beispiel zur Nutzung der 
\Environment{tabular}"=Umgebung.
%
\begingroup
\pcolumnfuzz{70pt}
\begin{Hint*}
\begin{tabular}{lcrp{33mm}}
\toprule
\textbf{Linksbündig} & \textbf{Zentriert} & 
\textbf{Rechtsbündig} & \textbf{Blocksatz} \tabularnewline\midrule
a   & b   & c   & Dieser Text wird im Blocksatz gesetzt\tabularnewline
aa  & bb  & cc  & Auch Zeilenumbrüche sind vorhanden\tabularnewline
aaa & bbb & ccc & Worttrennungsmusterkontrolle\tabularnewline
\bottomrule
\end{tabular}
\end{Hint*}
\endgroup
%
Die Tabellenbreite ergibt sich aus der Breite der einzelnen Spalten. Bei dieser 
Umgebung liegt es allein beim Anwender, auf die korrekte Breite der Tabelle zu 
achten, damit diese nicht über die Seitenränder hinausragt. Das kann auf Dauer 
recht aufwändig werden. Das Festlegen der Gesamtbreite einer Tabelle und das 
automatische Berechnen einiger oder aller Spaltenbreiten ist sicher die 
angenehmere Variante. Wie sich dies sehr komfortabel bewerkstelligen lässt, 
wird~-- ebenso wie die Lösung des Problems eines nicht umbrochenen Eintrags wie
in der letzten Spalte der dritten Zeile der obigen Tabelle~-- im nachfolgenden
Abschnitt demonstriert.


\subsection{Tabellen mit variabler Spaltenbreite}
Die Pakete \Package{tabularx}, \Package{tabulary} oder auch \Package{tabu} 
stehen für das Setzen von Tabellen mit dynamisch berechneter Spaltenbreite bei 
fest vorgegebener Gesamtbreite zur Verfügung, welche nun kurz vorgestellt 
werden sollen. Die beiden erstgenannte Pakete \Package{tabularx} sowie 
\Package{tabulary} stammen vom gleichen Paketautor und haben sich als sehr 
stabil erwiesen. Das Paket \Package{tabu} ist relativ neu und versucht, viele 
Funktionalitäten ganz unterschiedlicher Pakete für den Tabellensatz in sich zu 
vereinen und glänzt insbesondere durch seine Vielseitigkeit. 
%
\begin{Preamble}
\usepackage{tabularx}
\usepackage{tabulary}
\usepackage{tabu}
\usepackage{longtable}

\end{Preamble}
%
Jedes der Pakete stellt einen oder mehrere neue Spaltentypen zur Verfügung. 
Dies bedeutet jedoch nicht, dass in den dazugehörigen Tabellenumgebungen 
lediglich diese genutzt werden können. Vielmehr werden diese zusätzlich 
angeboten, was bedeutet, dass auch in den neuen Tabellenumgebungen weiterhin 
die gewohnten Spaltentypen~\PValue{l},~\PValue{c},~\PValue{r}~und~\PValue{p} 
verfügbar sind.

\subsubsection{Die Tabellenumgebung \Environment*{tabularx}}
\label{sec:tabularx}%
%
Das Paket \Package{tabularx} stellt den Spaltentyp~\PValue{X} bereit, welcher 
prinzipiell dem Spaltentyp~\PValue{p} entspricht. Auch für diesen wird eine 
\Macro{parbox} verwendet, allerdings wird deren Breite \emph{automatisch} 
berechnet. Die Umgebung \Environment{tabularx}(\Package{tabularx}) erwartet vor 
der Spaltentypenangabe als obligatorisches Argument die gewünschte Breite der 
Tabelle. Spalten vom Typ~\PValue{l},~\PValue{c}~und~\PValue{r} werden jedoch 
weiterhin \emph{ohne} Zeilenumbruch gesetzt werden. Nur für Spalten vom 
Typ~\PValue{X} und deren Derivate wird aus dem verbliebenen Platz die Breite 
berechnet. Für die \Environment{tabularx}(\Package{tabularx})"~Umgebung wird 
das vorherige Beispiel wiederholt.
%
\begingroup
\pcolumnfuzz{70pt}
\begin{Hint*}
\begin{tabularx}{11.7cm}{lcrX}
\toprule
\textbf{Linksbündig} & \textbf{Zentriert} & 
\textbf{Rechtsbündig} & \textbf{Blocksatz} \tabularnewline\midrule
a    & b    & c    & Dieser Text wird im Blocksatz gesetzt\tabularnewline
aa   & bb   & cc   & Auch Zeilenumbrüche sind vorhanden\tabularnewline
aaa  & bbb  & ccc  & Worttrennungsmusterkontrolle\tabularnewline
aaaa & bbbb & cccc & \hspace{0pt}Worttrennungsmusterkontrolle
\tabularnewline\bottomrule
\end{tabularx}
\end{Hint*}
\endgroup
%
Die Breite der letzten Spalte wurde dabei aus der Angabe der Gesamtbreite mit 
\PValue{11.7cm} berechnet. Des Weiteren ist zu sehen, wie das Problem des 
Zeilenumbruchs behandelt werden kann. Normalerweise wird das erste Wort in 
einem Absatz von \hologo{LaTeX} \emph{nie} umbrochen. Dies kann durch das 
Einfügen eines breitenlosen Leerzeichens mit \Macro{hspace}[\PParameter{0pt}] 
umgangen werden. Die gefundene Lösung ist allerdings alles andere als elegant.

Mit den Möglichkeiten des Paketes \Package{array} ist das Problem relativ 
schnell gelöst. Es wird mit \Macro{newcolumntype}(\Package{array}) ein neuer 
Spaltentyp definiert. Das erste Argument legt den Namen des Spaltentyps fest. 
Mit \PValue{>\Parameter{Definitionen}}\Parameter{Typ} wird im zweiten Argument 
das Ausführen von \PName{Definitionen} vor dem Setzen des eigentlichen Inhaltes 
in einer \Parameter{Typ}"~Spalte definiert. Es wird ein neuer, auf der 
\PValue{X}"~Spalte basierender Typ~\PValue{Y} definiert, welcher zu Beginn der 
Spalte den Phantomabstand automatisch einfügt. Darauf basierend werden drei 
Spaltentypen für den links- und rechtsbündigen sowie zentrierten Textsatz mit 
einem möglichen Zeilenumbruch erstellt.
%
\begin{Trunk+}
\subsection{Tabellen als Gleitobjekte}
Tabellen sollten in der \texttt{table}"=Gleitumgebung gesetzt werden. 
Welche Umgebung für die Tabelle selbst dabei genutzt wird ist dabei
nicht relevant. Es können sowohl die normale \texttt{tabular}"=Umgebung 
als auch die Umgebungen \texttt{tabularx}, \texttt{tabulary} sowie 
\texttt{tabu} für variable Spaltenbreiten bei einer fest vorgegebenen 
Tabellenbreite oder jede andere Tabellenumgebung genutzt werden. 
Nachfolgend wird dies an mehreren Beispielen demonstriert.

\subsubsection{Eine gleitende tabularx-Tabelle}
Es wird eine Tabelle mithilfe der \texttt{tabularx}"=Umgebung erstellt. 
Zu sehen ist diese in \autoref{tab:tabularx}. Für diese werden zuvor 
neue Spaltentypen definiert.

\end{Trunk+}
\CodeHook{\renewcommand*{\newcolumntype}[2]{}}
\begin{Trunk*}
\newcolumntype{Y}{>{\hspace{0pt}}X}
\newcolumntype{D}{>{\raggedright}Y}
\newcolumntype{E}{>{\centering}Y}
\newcolumntype{F}{>{\raggedleft}Y}

\end{Trunk*}
%
Nachfolgend wird die Tabelle gesetzt mit den neuen Spaltentypen gesetzt.
%
\begin{Trunk}
\begin{table}
\begin{tabularx}{\textwidth}{@{}DEFY@{}}
\toprule
\textbf{Linksbündig} & \textbf{Zentriert} & 
\textbf{Rechtsbündig} & \textbf{Blocksatz} \tabularnewline
\midrule
Ein linksbündiger Blindtext zur Demonstration einer S"~Spalte &
Ein zentrierter Blindtext zur Demonstration einer T"~Spalte &
Ein rechtsbündiger Blindtext zur Demonstration einer U"~Spalte &
Ein längerer Blindtext im Blocksatz zur besseren Demonstration
einer Y"~Spalte\tabularnewline
\bottomrule
\end{tabularx}
\caption{Eine \texttt{tabularx}"=Tabelle}\label{tab:tabularx}
\end{table}

\end{Trunk}
\InputCode\noindent
%
Die Breite der einzelnen \PValue{X}"~Spalten und deren Derivaten ist dabei 
identisch. Ein Gewichten der Spaltenbreiten untereinander ist prinzipiell 
möglich, allerdings mit einigen Einschränkungen verbunden. Genaueres hierzu ist 
der Dokumentation des Paketes \Package{tabularx} zu entnehmen.

\subsubsection{Die Tabellenumgebung \Environment*{tabulary}}
\label{sec:tabulary}%
%
Das Paket \Package{tabulary} verfolgt im Vergleich zum zuvor beschrieben Paket 
\Package{tabularx} einen etwas anderen Ansatz. Anstatt die verfügbare Breite 
über alle Spalten gleichmäßig zu verteilen, werden die Spaltenbreite anhand des 
jeweils darin befindlichen Inhaltes zueinander gewichtet. Dafür werden die vier 
Spaltentypen~\PValue{L},~\PValue{C},~\PValue{R}~sowie~\PValue{J} zur Verfügung 
gestellt. Diese entsprechen prinzipiell dem Spaltentyp~\PValue{p}, allerdings 
wird deren Breite \emph{automatisch} berechnet, wobei Spalten mit mehr Inhalt 
breiter gesetzt werden. Für die automatische Berechnung der Spaltenbreiten muss 
verständlicherweise die gewünschte Gesamtbreite der Tabelle angegeben werden, 
was bei der \Environment{tabulary}(\Package{tabulary})"~Umgebung mit dem ersten 
obligatorischen Argument vor der Angabe der Spaltendefinition erfolgt. Werden 
eine oder mehrere Spalten vom Typ~\PValue{l},~\PValue{c}~und~\PValue{r} 
angegeben, so werden diese weiterhin in ihrer natürlichen Breite und ohne 
Zeilenumbruch gesetzt. 

Die Gewichtung der Spalten vom Typ~\PValue{LCRJ} und deren Derivate wird aus 
dem verbliebenen Platz die Breite berechnet. Die Berechnung der Spaltenbreite 
kann durch den Anwender außerdem über zwei Längen beeinflusst werden. Mit 
\Length{tymin}(\Package{tabulary}) wird die Mindestgröße für Spalten 
festgelegt. Sollten Sie wissen, dass eine Spalte sehr schmal wird, können Sie 
allerdings auch einfach eine der Standardspalten~\PValue{lcrp} nutzen. Mit der 
Länge \Length{tymax}(\Package{tabulary}) kann die maximale Ausdehnung einer der 
Spalten vom Typ~\PValue{LCRJ} festgelegt werden. 

Für die \Environment{tabulary}(\Package{tabulary})"~Umgebung wird das vorherige 
Beispiel wiederholt. Zu beachten ist, dass die Spaltentypen~\PValue{LCRJ} mit 
dem Makro \Macro{tyformat}(\Package{tabulary}) bereits ein breitenloses 
Leerzeichen einsetzen, womit sich das manuelle Einfügen von selbigem durch den 
Anwender erübrigt.
%
\begin{Trunk+}
\subsubsection{Eine gleitende tabulary-Tabelle}
Es wird eine Tabelle mithilfe der \texttt{tabulary}"=Umgebung erstellt. 
Zu sehen ist diese in \autoref{tab:tabulary}. 

\end{Trunk+}
\begin{Trunk}
\begin{table}
\begin{tabulary}{\textwidth}{@{}LCRJ@{}}
\toprule
\textbf{Linksbündig} & \textbf{Zentriert} & 
\textbf{Rechtsbündig} & \textbf{Blocksatz} \tabularnewline\midrule
Ein linksbündiger Blindtext zur Demonstration einer L"~Spalte &
Ein zentrierter Blindtext zur Demonstration einer C"~Spalte &
Ein rechtsbündiger Blindtext zur Demonstration einer R"~Spalte &
Ein wesentlich längerer und absolut inhaltsleerer Blindtext im 
Blocksatz für eine um einiges bessere Demonstration einer J"~Spalte
\tabularnewline\bottomrule
\end{tabulary}
\caption{Eine \texttt{tabulary}"=Tabelle}\label{tab:tabulary}
\end{table}

\end{Trunk}
\InputCode


\subsubsection{Die Tabellenumgebung \Environment*{tabu}}
\label{sec:tabu}%
%
Das Paket \Package{tabu} bietet eine mächtige und komfortable Alternative zu 
\Package{tabularx}. Es kam in diesem Tutorial bereits für die Verzeichnisse von 
Abkürzungen und Symbolen in \autoref{sec:glossaries} zum Einsatz. Leider ist 
das Paket in er aktuellen Version~v2.8 mit Vorsicht zu genießen. Zum einen
wären für das Paket in der aktuellen Version seit geraumer Zeit ein paar 
kleinere Bugfixes notwendig, mehr dazu sehen Sie später. Außerdem wird sich die 
\hrfn{https://groups.google.com/d/topic/comp.text.tex/xRGJTC74uCI}{%
  Benutzerschnittstelle in einer zukünftigen Version%
}
sehr stark ändern. Sie sollten sich bewusst sein, dass mit der Version~v2.8
gesetzte Dokumente gegebenenfalls später angepasst werden müssen.

Nichtsdestotrotz soll hier folgend die individuelle Verwendung einer 
\Environment{tabu}(\Package{tabu})"~Umgebung gezeigt werden, insbesondere weil 
für das Setzen umbruchfähiger Tabellen mit automatisch berechneten 
Spaltenbreiten momentan (fast) keine Alternative zu der Umgebung 
\Environment{longtabu}(\Package{tabu}) existiert. 

Dieses Paket definiert ebenso einen Spaltentyp~\PValue{X}, welchem allerdings 
zusätzlich ein optionales Argument angehangen werden kann. Mit diesem lässt 
sich sowohl die Gewichtung der automatisch berechneten Spalten untereinander 
als auch die Positionierung zur Grundlinie sowie die Ausrichtung des Inhaltes 
ändern.
%
\begingroup
\pcolumnfuzz{70pt}
\begin{Hint*}
\begin{tabu} to 11.7cm {lcrX}
\toprule
\textbf{Linksbündig} & \textbf{Zentriert} & 
\textbf{Rechtsbündig} & \textbf{Blocksatz} \tabularnewline\midrule
a    & b    & c    & Dieser Text wird im Blocksatz gesetzt\tabularnewline
aa   & bb   & cc   & Auch Zeilenumbrüche sind vorhanden\tabularnewline
aaaa & bbbb & cccc & Worttrennungsmus\-terkontrolle\tabularnewline
aaa  & bbb  & ccc  & Worttrennungsmusterkontrolle
\tabularnewline\bottomrule
\end{tabu}
\end{Hint*}
\endgroup
%
Das Beispiel zeigt auch gleich das momentan meiner Meinung nach größte Problem 
des Paketes: Bei mehrzeiligen Zellen fehlt vertikale Zwischenraum zur nächsten 
Tabellenzeile. Nun besteht entweder die Möglichkeit, bei jeder Angabe einer 
Spalte im Tabellenkopf, welche eine \Macro{parbox} verwendet, mit dem Ausdruck 
\PValue{>\PParameter{\Macro{strut}}} die fehlenden Unterlänge auszugleichen 
oder es wird abermals ein entsprechender Spaltentyp definiert. Das ist für 
die \PValue{X}"~Spalten der Umgebung \Environment{tabu}(\Package{tabu}) etwas 
schwieriger, da \Macro{newcolumntype}(\Package{array}) normalerweise die 
Definition eines optionalen Argumentes nicht vorgesehen ist. Mit ein wenig 
Trickserei ist das dennoch möglich:
%
\begin{Trunk+}
\subsubsection{Eine gleitende tabu-Tabelle}
In \autoref{tab:tabu} ist eine weitere Tabelle mit variabler Breite der 
Spalten und festgelegter Gesamtbreite zu sehen, welche in der Umgebung 
\texttt{tabu} gesetzt wurde. Auch für diese wird zuerst ein neuer 
Spaltentyp definiert, der die Unzulänglichkeiten der Umgebung reduziert. 
Mit \texttt{\textbackslash ttabbox} aus dem Paket \texttt{floatrow} wird 
die Beschriftung auf die Breite der Tabelle begrenzt.

\end{Trunk+}
\CodeHook{\renewcommand*{\newcolumntype}[2]{}}
\begin{Trunk*}
\makeatletter
\newcolumntype{Z}{}
\renewcommand*{\NC@rewrite@Z}[1][]{%
  \NC@find>{\hspace{0pt}}X[#1]<{\@finalstrut\@arstrutbox}%
}
\makeatother

\end{Trunk*}
%
Anschließend kann der \PValue{Z}"~Spaltentyp äquivalent zu \PValue{X} genutzt 
werden. Zu beachten ist dabei insbesondere die Möglichkeit, die Spalten in 
ihrer resultierenden Breite sehr einfach zueinander gewichten zu können~-- im 
Gegensatz zur Umgebung \Environment*{tabularx}, wo dieses Unterfangen etwas 
schwieriger ist. Das Ergebnis ist in \autoref{tab:tabu} zu sehen. Zu beachten 
ist die Verwendung von \Macro{ttabbox}(\Package{floatrow}) aus dem Paket 
\Package{floatrow}, um die Beschriftung an die Breite der Tabelle anzupassen. 
Genaueres dazu in \autoref{sec:floatlayout}.
%
\begin{Trunk}
\begin{table}
\ttabbox{%
  \begin{tabu} to .8\textwidth {@{}Z[3,l]Z[3,c]Z[3,r]Z[2,j]@{}}
    \toprule
    \textbf{Linksbündig} & \textbf{Zentriert} & 
    \textbf{Rechtsbündig} & \textbf{Blocksatz} \tabularnewline\midrule
    Ein linksbündiger Blindtext zur Demonstration einer Z[l]"~Spalte &
    Ein zentrierter Blindtext zur Demonstration einer Z[c]"~Spalte &
    Ein rechtsbündiger Blindtext zur Demonstration einer Z[r]"~Spalte &
    Ein Blindtext im Blocksatz innerhalb einer Z"~Spalte
    \tabularnewline\bottomrule
  \end{tabu}%
}{%
  \caption[Eine \texttt{tabu}"=Tabelle]{%
    Eine \texttt{tabu}"=Tabelle in Verbindung mit dem Befehl 
    \texttt{\textbackslash ttabbox}, welcher vom Paket \texttt{floatrow} 
    für Beschriftungen in Objektbreite bereitgestellt wird%
  }%
  \label{tab:tabu}%
}
\end{table}

\end{Trunk}
\InputCode



\section{Querverweise}
\label{sec:references}%
%
Damit alle möglichen Querverweise in einem PDF"~Dokument automatisch verlinkt 
werden, sollte das Paket \Package{hyperref} geladen werden. Um die erzeugten 
Links verträglich aussehen zu lassen, werden die beiden Paketoptionen 
\Option{colorlinks}(\Package{hyperref}) sowie \Option*{linkcolor=blue} 
verwendet. Da \Package{hyperref} allerhand Veränderungen an vielen 
Standardbefehlen vornimmt, sollte dieses als letztes in der Präambel 
eingebunden werden. Nur Pakete, bei denen in der Dokumentation explizit darauf 
hingewiesen wird, dass diese nach \Package{hyperref} zu laden sind, sollten 
auch danach folgen. Eines dieser wenigen Pakete ist das in diesem Tutorial 
verwendete \Package{glossaries}. 
%
\begin{Hint}
\usepackage[colorlinks,linkcolor=blue]{hyperref}

\end{Hint}
%
Um auf ein bestimmtes Objekt wie beispielsweise eine Tabelle, eine Abbildung 
oder einen bestimmten Anschnitt im Dokument referenzieren zu können, muss mit 
\Macro{label}[\Parameter{label}] ein Anker für selbiges an der entsprechenden 
Stelle gesetzt werden. Bei Überschriften sollte dies direkt nach der Nutzung 
des Gliederungsbefehls%
\footnote{%
  \Macro{chapter}(\Package{koma-script}), 
  \Macro{section}(\Package{koma-script}) etc.%
}
erfolgen. Für Abbildungen, Tabellen oder andere Objekte in Gleitumgebungen muss 
der Anker unmittelbar nach der Verwendung von 
\Macro{caption}(\Package{koma-script},\Package{caption}) respektive 
\Macro{captionof}(\Package{koma-script}) erstellt werden. Um nun auf diesen 
Anker zu referenzieren, sollte nicht \Macro{ref} sondern 
\Macro{autoref}(\Package{hyperref}) im Fließtext genutzt werden. Letztgenannter 
Befehl erzeugt nicht nur einen Link mit der dazugehörigen Nummerierung sondern 
stellt dieser die Bezeichnung des referenzierten Objektes%
\footnote{\enquote{Abbildung}, \enquote{Tabelle}, \dots}
voran. 

Mit dem Befehl \Macro{pageref} lässt sich auf die Seite eines Ankers verweisen. 
Das Paket \Package{varioref} kann dabei unterscheiden, ob der Anker sich auf 
der aktuellen Seite, auf einer Seite direkt davor beziehungsweise danach oder 
auf einer entfernteren Seite befindet und dementsprechend einen Hinweis darauf 
ausgeben. Hierfür werden die Befehle \Macro{vref}(\Package{varioref}) sowie 
\Macro{vpageref}(\Package{varioref}) zur Verfügung gestellt. Das Paket 
\Package{cleveref} vereint die Vorzüge von \Macro{autoref}(\Package{hyperref}) 
in Verbindung mit \Macro{vref}(\Package{varioref}) durch den Befehl 
\Macro{cref}(\Package{cleverref}).



\section{Zitate}
Für das wörtliche Zitieren kann das Paket \Package{csquotes} verwendet werden. 
Dieses stellt den Befehl \Macro{enquote}(\Package{csquotes}) bereit, der die 
kennzeichnenden Anführungszeichen automatisch in Abhängigkeit der verwendeten 
Dokumentsprache setzt. 
%
\begin{Hint}
\usepackage{csquotes}
\end{Hint}
%
Da für die unterschiedlichen Fachdisziplinen teils sehr unterschiedliche 
Zitierrichtlinien existieren, kann hier keine allgemeingültige Empfehlung zu 
Stilfragen gegeben werden. Zu beachten ist auf jeden Fall, dass zu wörtlichen 
Zitaten eine Quellenangabe erfolgen \emph{muss}. Dabei sollte darauf geachtet 
werden, dass nicht nur die Quelle selbst sondern zumindest auch die Seitenzahl 
des Zitats angegeben wird.
%
\begin{refsection}[\jobname-temp.bib]
\begin{Trunk+}
\section{Zitate}
Bei der Verwendung von wörtlichen Zitaten sollten diese als solche 
gekennzeichnet werden.
\end{Trunk+}
\begin{Trunk*}
\enquote{Dies ist ein zugegebenermaßen nicht sehr sinnvolles Zitat.}
\cite[58]{hanisch14}
\end{Trunk*}
\begin{Trunk+}
Für eine möglichst gut nachvollziehbare Referenz sollte nicht nur 
das Werk selber sondern zumindest die Seitenzahl und gegebenenfalls 
der Absatz der originalen Textstelle angegeben werden. 
\end{Trunk+}
%
Längere Zitate sollten abgesetzt vom Fließtext ausgegeben werden. Hierfür 
bieten sowohl die Standard"~ als auch die \KOMAScript-Klassen standardmäßig 
die Umgebungen \Environment{quote}(\Package{koma-script}) und 
\Environment{quotation}(\Package{koma-script}) an. Allerdings verwenden beide 
fest eingestellte Absatzauszeichnungen. Sollen jedoch die mit der Option 
\Option{parskip}(\Package{koma-script}) vorgegebenen Einstellungen für Absätze 
beachtet werden, so bietet sich das Paket \Package{quoting} an, welches die 
gleichnamige Umgebung \Environment{quoting}(\Package{quoting}) definiert und 
die Absätze wie im restlichen Dokument auszeichnet.
%
\begin{Preamble}
\usepackage{quoting}

\end{Preamble}
%
\begin{Trunk*}
\begin{quoting}
\enquote{%
  Dies ist ein noch sinnloseres Zitat. Allerdings wird zumindest die
  Wirkung der Umgebung \texttt{quoting} bei der Absatzauszeichnung 
  deutlich.
  
  Wie zu sehen ist, wird der zweite Absatz~-- wie jeder weitere~-- 
  aufgrund der Option \texttt{parskip=false} eingezogen.
}
\cite[sinngemäß nach][\pno{} 12, zweiter Absatz]{hanisch14}
\end{quoting}
\end{Trunk*}
\begin{Trunk+}
Ebenfalls sollten sinngemäße Zitate mit einer möglichst genauen Referenz 
angegeben werden. Dies kann im Laufe der Arbeit auch für einen selbst von 
Vorteil sein, wenn beispielsweise die originale Textpassage noch einmal 
analysiert werden soll.

\end{Trunk+}
\begin{quoting}[rightmargin=0pt]
\makeatletter
\let\markboth\@gobbletwo
\let\markright\@gobble
\makeatother
\printbibliography
\end{quoting}
\end{refsection}



\section{Worttrennungen}
\label{sec:hyphenation}%
%
Bei der Verwendung von \hologo{LaTeXe} gibt es ab und an das Problem, dass 
bestimmte Wörter am Zeilenende falsch getrennt werden. Dies hat historische 
Gründe. Als \hologo{TeX} in den 1980er"~Jahren entwickelt wurde, war es 
aufgrund der zu dieser Zeit zur Verfügung stehenden Speichergrößen schier 
unmöglich, die Worttrennungen mit einem Wörterbuchansatz umzusetzen. 
Stattdessen wurde~-- um den benötigten Speicherbedarf möglichst gering zu 
halten~-- ein Algorithmus entworfen, der mögliche Trennstellen in einem Wort 
mit einer Mustererkennung ermittelt. Die verwendeten Muster basieren jedoch 
auf der englischen Sprache, weshalb in nicht"~englischsprachigen Texten auch 
heute noch teilweise falsche Trennstellen gefunden werden.

In \autoref{sec:introduction} wurde bereits auf das Paket 
\Package{hyphsubst} hingewiesen, welches dieses Manko~-- aufgrund der nicht 
mehr so stark gegebenen Restriktionen bezüglich des Speicherbedarfes~-- mit 
einem Wörterbuch behebt. Dieses Paket muss für \Engine{pdfLaTeX} vom Anwender 
eingebunden werden, \Engine{LuaLaTeX} und \Engine{XeLaTeX} verwenden die 
Trennmuster~-- ein geladenes Sprachpaket wie \Package{polyglossia} oder 
\Package{babel} vorausgesetzt~-- automatisch. Damit auch Wörter mit Umlauten 
richtig getrennt werden, ist zudem das Paket \Package{fontenc} mit der 
\PValue{T1}"~Schriftkodierung respektive \Package{fontspec} notwendig. Der 
Beginn einer Dokumentpräambel könnte folgendermaßen aussehen:
%
\begin{Hint}
\documentclass[ngerman,§\PName{Klassenoptionen}§]§\Parameter{Dokumentklasse}§
\usepackage[T1]{fontenc}
\ifpdftex{\usepackage[ngerman=ngerman-x-latest]{hyphsubst}}{}
\usepackage{babel}
§\dots§
\end{Hint}
%
Die Option \Option{ngerman}(\Package{hyphsubst}) lädt dabei die Trennmuster für 
die neue deutsche Rechtschreibung, wobei der Wert \PValue{ngerman-x-latest} die 
neuesten aller verfügbaren Trennmuster lädt. Für die alte Orthografie ist 
stattdessen die Option \Option{german}(\Package{hyphsubst}) mit dem Wert 
\PValue{german-x-lastest} zu verwenden.


\subsection{Einmalige und globale Worttrennungskorrektur}
Sollte es dennoch passieren, dass ein Wort nicht korrekt getrennt wird, lässt
sich dies manuell korrigieren. Dafür stellt das Paket \Package{babel} mehrere 
Kurzbefehle zur Verfügung. Mit dem Makro \Macro{hyphenation} lassen sich für 
ein bestimmtes Wort außerdem alle potenziellen Trennstellen angeben. Zum Thema 
\tudhyperref{sec:tips:hyphenation}{Worttrennungen in deutschsprachigen Texten} 
ist im \TUDScript-Handbuch unter dem Kapitel Tipps~\&~Tricks mehr zu finden. 
Dort wird auch erläutert, wie das Problem der alleinigen Trennung von Wörtern 
mit Bindestrich an selbigem umgangen werden kann.


\subsection{Worttrennung im Flattersatz}
Normalerweise ist bei \hologo{LaTeXe} die Worttrennung für den Flattersatz%
\footnote{\Macro{raggedright}, \Macro{raggedleft}, \Macro{centering}} 
deaktiviert. Mit dem Paket \Package{ragged2e} kann diese aktiviert werden. Nach 
dem Laden stehen die Befehle \Macro{Centering}(\Package{ragged2e}), 
\Macro{RaggedLeft}(\Package{ragged2e}) sowie 
\Macro{RaggedRight}(\Package{ragged2e}) für den Anwender zur Verfügung, um 
Worttrennungen im Flattersatz zu verwenden. Die Option 
\Option{newcommands}(\Package{ragged2e}) überschreibt die originalen Befehle 
\Macro{centering}, \Macro{raggedleft} und \Macro{raggedright} mit den 
korrelierten Befehlen aus dem Paket. Genaueres dazu ist der Dokumentation des 
Paketes zu entnehmen.



\section{Mikrotypografische Feinheiten}
Unter Mikrotypografie oder Detailtypografie wird die Gestaltung folgender 
Feinheiten des Schriftsatzes verstanden:
\begin{itemize}[noitemsep]
\item korrekte Anwendung der Satzzeichen
\item Schriftart, Kapitälchen und Ligaturen
\item Wortabstände sowie Laufweite (Buchstaben- und Zeichenabstände)
\item optischer Randausgleich
\end{itemize}

\noindent Die letzten beiden Punkte werden sensationell gut durch das Paket 
\Package{microtype} bearbeitet. Dieses muss lediglich in der Präambel geladen 
werden, wobei es nicht mit der klassischen \hologo{TeX}"~Engine, wohl jedoch 
mit \Engine{pdfTeX} als auch \Engine{LuaTeX} sowie \Engine{XeTeX} funktioniert.
%
\begin{Preamble}
\usepackage[babel]{microtype}

\end{Preamble}
%
Die Option \Option{babel}(\Package{microtype}) ist nur notwendig, wenn einzelne 
Textpassagen oder das gesamte Dokument in Englisch oder Französisch verfasst 
werden und hierfür die genutzte Sprache mit den entsprechenden Möglichkeiten 
von \Package{babel} eingestellt wird.

Das korrekte Anwenden von Satzzeichen wird für deutschsprachige Texte bereits 
in mehreren Artikeln \cite{neubauer1996,neubauer1997,struckmann2007,bier2009} 
sehr gut erläutert, weshalb ein Hinweis auf diese Arbeiten für weitergehende 
Informationen zu dieser Thematik absolut ausreichend ist. Nachfolgend werden 
Tipps gegeben sowie einige \hologo{LaTeXe}"=Pakete vorgestellt, die bei der 
Umsetzung der in den genannten Artikeln erläuterten Empfehlungen helfen.


\subsection{Abkürzungen}
In \cite{struckmann2007} wird gleich zu Beginn darauf hingewiesen, wie häufig 
verwendete Abkürzungen sehr einfach im Dokument verwendet werden können und 
dabei typografisch richtig gesetzt werden. Beispielsweise gelingt dies für die 
Abkürzung \enquote{zum~Beispiel~(z.\,B.)} mit dieser Definition:
%
\begin{Hint}
\newcommand{\zB}{\mbox{z.\,B.}\xspace}
\end{Hint}
%
Wenn ohnehin das Paket \Package{glossaries} für Akronyme verwendet wird, kann 
alternativ dazu auch dessen Funktionalität genutzt werden. Nach dem Einbinden 
muss in der Präambel lediglich Folgendes eingefügt werden:
%
\begin{Hint}
  \newignoredglossary{abbreviation}
  \newcommand*{\newabbreviation}[4][]{%
    \newacronym[type=abbreviation,#1]{#2}{\mbox{#3\xspace}}{#4}%
  }%
\end{Hint}
%
Mit \Macro{newignoredglossary}(\Package{glossaries}) wird dabei ein Glossar 
erstellt, welche bei der Ausgabe aller Glossare mit 
\Macro{printglossaries}(\Package{glossaries}) ignoriert wird. Im Dokument kann 
die gewünschte Abkürzung mit 
%
\begin{Hint*}
\newabbreviation{zB}{z.\,B.}{zum Beispiel}
\end{Hint*}
%
definiert und über die gewohnten Befehlen aus \Package{glossaries} verwendet 
werden.
%
\begin{Hint*}
Die Ausgabe der soeben definierten Abkürzung \enquote{\acrshort{zB}}
erfolgt mit dem Befehl \texttt{\textbackslash acrshort}. Wurde das
Paket \texttt{glossaries} mit der Option \texttt{shortcut} geladen, 
kann auch das Makro \texttt{\textbackslash acs} genutzt werden.
\end{Hint*}


\subsection{Mathematiksatz}
Von \hologo{LaTeXe} werden einige Umgebungen für das Setzen mathematischer 
Formeln und Ausdrücke bereitgestellt, welche meist jedoch nicht \emph{alle} 
Anforderungen eines Nutzers an den Mathematiksatz erfüllen können. Das Paket
\Package{amsmath} bietet deshalb eine Reihe weiterführender Umgebungen für das 
Setzen von einzeiligen und mehrzeiligen sowie ausgerichteter Formeln zur 
Verfügung. Auf eine Einführung in das Paket wird~-- aufgrund der bereits in 
großer Zahl vorhandenen Einführungen~-- in diesem Tutorial verzichtet. Außerdem 
stehen mit \Tutorial{mathtype} und \Tutorial{mathswap} zwei weitere Tutorials 
zur Verfügung, die sich explizit mit der Thematik eines guten mathematischen 
Satz beschäftigen.

Das Paket \Package{amsmath} stellt für \hologo{LaTeXe} den De"~facto"~Standard 
für den mathematischen Satz dar. Allerdings wird es~-- insbesondere aus diesem 
Grund~-- sehr \enquote{vorsichtig} weiterentwickelt. Für das Setzen neuer 
Dokumente sollte deshalb das Paket \Package{mathtools} verwendet werden, das
auf \Package{amsmath} basiert und dieses um zusätzliche Funktionalitäten 
ergänzt sowie Bugfixes bereithält.


\subsection{Einheiten}
Mit dem Paket \Package{siunitx} lassen sich sowohl physikalische Einheiten als 
auch Zahlen typografisch korrekt setzen. Sollen die Funktionalitäten in einem 
deutschsprachigen Dokument problemlos nutzbar sein, so sind die Hinweise zum 
\tudhyperref{sec:tips:siunitx}{Setzen von Einheiten mit \Package{siunitx}} im 
\TUDScript-Handbuch zu beachten. Das Paket \Package{units} ist nicht ganz so 
mächtig wie \Package{siunitx}, kann allerdings als pragmatische Alternative 
gesehen und genutzt werden.


\subsection{Darstellung von mathematischen Brüchen}
Für ansehnliche und insbesondere im Fließtext gut lesbare mathematische Brüche 
stellt das Paket \Package{xfrac} den Befehl \Macro{sfrac}(\Package{xfrac}) zur 
Verfügung.
%
\begin{Preamble}
\usepackage{xfrac}

\end{Preamble}


\subsection{Ligaturen}
Die meisten \hologo{LaTeX}"=Schriftfamilien enthalten Ligaturen, die für einen 
typografisch sauberen Satz bei zusammengesetzten Wörtern, wie sie gerade im 
Deutschen relativ häufig vorkommen, aufgetrennt werden müssten. Mit der 
Anwendung \Application{DeLig}<delig> kann dies automatisiert geschehen. Wird 
\Engine{LuaLaTeX} als Textsatzsystem genutzt, kann alternativ für dieses 
Unterfangen das Paket \Package{selnolig} verwendet werden.


\subsection{Listen}
\label{sec:lists}%
%
Für Auflistungen aller Art bietet \hologo{LaTeX} die drei Standardumgebungen 
\Environment{itemize}(\Package{koma-script},\Package{enumitem}),
\Environment{enumerate}(\Package{koma-script},\Package{enumitem}) und
\Environment{description}(\Package{koma-script},\Package{enumitem}). Das 
häufigste Anliegen bei der Verwendung dieser Umgebungen ist das Reduzieren 
der Abstände zwischen den einzelnen, mit 
\Macro{item}(\Package{koma-script},\Package{enumitem}) gesetzten Punkten, 
welche sehr häufig als zu groß empfunden werden. Eine Aufzählung erscheint ohne 
weitere Anpassungen wie folgt:
%
\CodePreamble{Aufzählung mit vertikalen Standardabständen}
\begin{Hint*}
\begin{itemize}
\item erster Punkt
\item zweiter Punkt
\item dritter Punkt
\end{itemize}
\end{Hint*}
%
Dabei wirken die Abstände zwischen den Punkten tatsächlich etwas 
überdimensioniert. Mit \Package{enumitem} können alle Listenumgebungen an die 
individuellen Bedürfnisse angepasst werden. 
%
\begin{Preamble}
\usepackage{enumitem}
\end{Preamble}
%
Um diese ungewollten Abstände zwischen den Listenpunkten zu entfernen, kann der 
Schlüssel \Option{noitemsep}(\Package{enumitem}) verwendet werden. Mit dem 
Befehl \Macro{setlist}(\Package{enumitem}) werden~-- ohne zusätzliche Angaben 
im optionalen Argument~-- alle Umgebungen für Aufzählungen global geändert. 
Hier soll dies lediglich für die Umgebung \Environment*{itemize} geschehen:
\begin{Preamble*}
\setlist[itemize]{noitemsep}

\end{Preamble*}
\CodePreamble{Aufzählung ohne vertikale Abstände}
\begin{Hint*}
\begin{itemize}
\item erster Punkt
\item zweiter Punkt
\item dritter Punkt
\end{itemize}

\end{Hint*}
%
Alternativ zur globalen Änderung einer speziellen oder aller Aufzählungslisten 
kann der Schlüssel \PValue{noitemsep} auch nur gezielt an einzelne Umgebung 
über das optionale Argument weitergereicht werden.

Natürlich erlaubt das Paket noch weitergehende Einstellungen für die Umgebungen 
\Environment{itemize}(\Package{koma-script},\Package{enumitem}),
\Environment{enumerate}(\Package{koma-script},\Package{enumitem}) und
\Environment{description}(\Package{koma-script},\Package{enumitem}). 
Zusätzlich können diese für eigene Aufzählungsumgebungen geklont und den 
eigenen Bedürfnissen angepasst werden. Genaueres hierzu ist der Dokumentation 
des Paketes \Package{enumitem} entnehmen.


\subsection{Datumsformatierung}
Das Paket \Package{isodate} stellt \Macro{printdate}(\Package{isodate}) bereit. 
Dieser Befehl kann für die Formatierung von Datumsangaben verwendet werden. 
Wird das Paket mit einer der Dokumentklassen aus dem \TUDScript-Bundle geladen, 
wird der genannte Befehl automatisch für alle bereitgestellten Datumsfelder 
verwendet. Alternativ dazu kann das Paket \Package{datetime2} genutzt werden, 
dessen Befehl \Macro{DTMDate}(\Package{datetime2}) wird ebenfalls unterstützt.
%
\begin{Preamble}
\usepackage{isodate}

\end{Preamble}


\subsection{Auslassungspunkte}
Das typografisch richtige Setzen von Auslassungspunkten mit den Befehlen 
\Macro{dots} respektive \Macro{textellipsis} wird im Handbuch unter 
\tudhyperref{sec:tips:dots}{Das Setzen von Auslassungspunkten} genauer 
beschrieben, weshalb hier nur kurz darauf eingegangen wird. Es wird das Paket 
\Package{ellipsis} benötigt.
%
\begin{Preamble}
\usepackage{ellipsis}
\end{Preamble}
%
Für deutschsprachige Texte sollten außerdem noch 
%
\begin{Preamble}
\let\ellipsispunctuation\relax

\end{Preamble}
%
ausgeführt werden.



\section{Erstellen von Abbildungen}
\label{sec:figures}%
%
Die einfachste Möglichkeit, in einer wissenschaftlichen Abhandlung Grafiken zu 
verwenden, ist sicherlich das Einbinden von externen Abbildungen mit dem Befehl 
\Macro{includegraphics}(\Package{graphicx}) aus dem Paket \Package{graphicx}. 
Allerdings ist für eine wissenschaftliche Arbeit in höchster Qualität meiner 
Meinung nach das bloße Einfügen einer gescannten oder heruntergeladenen Grafik 
unzureichend. Vielmehr sollte diese zum einen als skalierbare Vektorgrafik 
vorliegen, um sowohl die höchste Druckqualität als auch die 
Wiederverwendbarkeit sicherzustellen, und zum anderen in ihrer Gestaltung dem 
verwendeten Layout des verfassten Dokumentes entsprechen. Dies betrifft sowohl 
die eingesetzten Farben und Schriften als auch die Stärke der verwendeten 
Linien.

Zum Zeichnen von Grafiken sind die freien Programme \Application{LaTeXDraw} und 
besonders \Application{Inkscape} empfehlenswert, da diese die Verwendung der 
Dokumentschriften innerhalb der erstellten Grafiken ermöglichen. Im Handbuch zu 
\TUDScript wird dies für \tudhyperref{sec:tips:svg}{\Application{Inkscape}} 
genauer erläutert. Eine andere Variante ist das \enquote{Programmieren} von 
Grafiken. Die zwei bekanntesten \hologo{LaTeX}"=Pakete sind \Package{tikz}<pgf> 
und \Package{pstricks}<pstricks-base>. Für diese beiden wird im Folgenden eine 
Abbildung exemplarisch erstellt, ohne dabei tiefer ins Detail gehen zu wollen. 
Ebenfalls wird darauf verzichtet, diese im Kopierbeispiel einzubinden. Für 
beide Pakete gibt es neben der umfangreichen Dokumentation zahlreiche Beispiele 
und Erläuterungen im Internet.


\subsection{Das Paket \Package{tikz}}
Zur Verwendung von \Package{tikz}<pgf> ist nicht viel zu sagen. Einfach in der 
Dokumentpräambel einbinden und es kann losgehen. Für das nachfolgende Beispiel 
sind außerdem noch zwei weitere Programmbibliotheken für \Package{tikz}<pgf> 
notwendig.
%
\begin{Hint}
\usepackage{tikz}
\usetikzlibrary{chains}
\usetikzlibrary{decorations.markings}
\tikzset{on grid}
\end{Hint}
%
Ein kurze Anmerkungen zur Skalierung der Grafiken möchte ich mir erlauben. 
Normalerweise verwendet \Package{tikz}<pgf> Koordinaten im Zentimeter"=Raster. 
Ich persönlich bevorzuge der relativierte Skalierung auf die Textbreite des 
Dokumentes, wobei eine Einheit genau \SI{1}{\percent} dieser entspricht. Das 
kann mit folgendem Quelltext erreicht werden:
%
\CodeHook{\renewcommand\newlength[1]{}}
\begin{Hint*}
\newlength{\tikzunit}
\setlength{\tikzunit}{.01\textwidth}
\tikzset{x=\tikzunit,y=\tikzunit}
\end{Hint*}
%
Nachfolgend wird eine Grafik innerhalb einer Gleitumgebung exemplarisch 
erstellt. Das Ergebnis ist in \autoref{fig:tikz} zu sehen.

\begin{Hint?}
\begin{figure}
\begin{tikzpicture}
  \tikzstyle{inner box}=[%
    text width=17\tikzunit,
    align=center,
    rectangle,
    inner sep=.5\tikzunit,
    minimum height=8\tikzunit,
    font=\hspace{0pt},
    draw
  ]
  \tikzstyle{inner label}=[align=center, font=\scriptsize]
  \tikzstyle{inner box chain}=[every node/.style={on chain}]
  \tikzstyle{inner box chain below}=[%
    inner box chain, node distance=8\tikzunit,continue chain=going below
  ]
  \tikzstyle{inner box chain right}=[%
    inner box chain,node distance=35\tikzunit,continue chain=going right
  ]
  \tikzstyle{inner box chain above}=[%
    inner box chain,node distance=16\tikzunit,continue chain=going above
  ]
  \tikzstyle{pstarrow->}=[%
    decoration={markings,
      mark=at position 1 with {\arrow[xscale=1.5]{stealth}};
    },
    postaction={decorate},
    shorten >=0.7pt
  ]
  \newcommand{\tikzparbox}[2][9]{%
    \parbox{#1\tikzunit}{\centering\hspace{0pt}#2}%
  }
  \begin{scope}[start chain]
    \begin{scope}[inner box chain below]
      \node(NE)[inner box]{Navigations\-ebene};
      \node(NB)[inner label]{gewählte Fahrtroute\\ zeitlicher Ablauf};
      \node(BE)[inner box]{{Bahnführungs\-ebene}};
      \node(BS)[inner label]{%
        gewählte Führungsgrößen:\\ Sollspur, Sollgeschwindigkeit%
      };
      \node(SE)[inner box]{Stabilisierungs\-ebene};
    \end{scope}
    \begin{scope}[inner box chain right]
      \node(LQ)[inner box]{Längs- und Querdynamik};
      \node(FO)[inner box]{Fahrbahn\-oberfläche};
    \end{scope}
    \begin{scope}[inner box chain above]
      \node(FR)[inner box]{Fahrraum\\ \smallskip{\scriptsize Straße 
      und\\ \vspace{-1.5ex}Verkehrssituation}};
      \node(SN)[inner box]{Straßennetz};
    \end{scope}
  \end{scope}
  \begin{scope}[inner label,minimum size=0pt]
    \draw [pstarrow->] (FO) -| ++(13.5,-12) to
      node [above]{Istspur, Istgeschwindigkeit} ++(-97,0) |- (SE);
    \draw [pstarrow->] (FR) -| ++(14  ,-32) to 
      node [above]{Bereich sicherer Führungsgrößen} ++(-98,0) |- (BE);
    \draw [pstarrow->] (SN) -| ++(14.5  ,-52) to 
      node [above]{mögliche Fahrtroute} ++(-99,0) |- (NE);
  \end{scope}
  \begin{scope}[inner label]
    \draw              (NE) to (NB);
    \draw [pstarrow->] (NB) to (BE);
    \draw              (BE) to (BS);
    \draw [pstarrow->] (BS) to (SE);
    \draw [pstarrow->] (SE) to
      node[above] {\tikzparbox{Stell\-größen}}
      node[below] {\tikzparbox{Lenken Gasgeben Bremsen}}
    (LQ);
    \draw [pstarrow->] (LQ) to
      node[above]{\tikzparbox{Regel\-größen}}
      node[below]{\tikzparbox{Fahrzeugbewegung}}
    (FO);
    \draw [pstarrow->] (LQ)+(24,0) |- (FR);
    \draw [pstarrow->] (LQ)+(24,0) |- (SN);
  \end{scope}
  \begin{scope}[very thick,rounded corners=5\tikzunit]
    \draw (-12.5,-40) rectangle (12.5,14);
    \draw ( 22.5,-40) rectangle (47.5,-18);
    \draw ( 57.5,-40) rectangle (82.5,14);
  \end{scope}
  \begin{scope}[font=\bfseries]
    \node at (0,9) {Fahrer};
    \node at (35,-23) {Fahrzeug};
    \node at (70,9) {Umwelt};
  \end{scope}
\end{tikzpicture}
\caption{Eine mit TikZ erstellte Grafik}\label{fig:tikz}
\end{figure}
\end{Hint?}



\subsection{Das Paket \Package{pstricks}}
Die zuvor mit \Package{tikz}<pgf> erstellte Grafik wird nachfolgend mit dem 
Paket \Package{pstricks} erstellt. Mit diesem lassen sich PostScript"=Grafiken 
erstellen. Ähnlich wie bei \Package{tikz}<pgf> gibt es ein Kernpaket und 
zusätzliche Ergänzungen. Diese werden jedoch nicht als Programmbibliotheken 
sondern ganz normal als \hologo{LaTeX}"=Pakete eingebunden.
%
\begin{Hint}
\usepackage{pstricks,pst-node}
\end{Hint}
%
Die Verwendung von PostScript hat zahlreiche Vorteile, macht die direkte 
Nutzung von \Engine{pdfLaTeX} etwas schwierig. Das Paket \Package{auto-pst-pdf} 
schafft hier jedoch Abhilfe. Wird das Textsatzsystem \Engine{pdfLaTeX} über die 
zusätzliche Option \Option{-{}-shell-escape}(\Engine{pdfLaTeX}) beziehungsweise 
\Option{-{}-enable-write18}(\Engine{pdfLaTeX}) mit erweiterten Schreibrechten 
aufgerufen, so werden die Abbildungen automatisch im PDF"~Format erzeugt und 
eingebunden.
%
\begin{Hint}
\usepackage{auto-pst-pdf}
\end{Hint}
%
Bei einigen Pakete kommt es in der Kombination mit \Package{auto-pst-pdf} zu 
kleineren Problemen, die sich jedoch meist relativ leicht beheben lassen. Im 
Handbuch wird die \tudhyperref{sec:tips:auto-pst-pdf}{gleichzeitige Verwendung} 
von \Package{auto-pst-pdf} in Kombination mit den Paketen \Package{tikz}<pgf>, 
\Package{todonotes} sowie \Package{floatrow}~-- bei der Nutzung des Befehls 
\Macro{ffigbox}(\Package{floatrow})~-- erklärt. Diese Hinweise sollten Sie 
gegebenenfalls beachten.

Ähnlich wie zuvor bei der \Package{tikz}<pgf>"~Grafik wird die nachfolgende 
\Package{pstricks}"=Abbildung auf die Textbreite des Dokumentes skaliert, wobei 
eine Einheit genau \SI{1}{\percent} dieser entspricht. Außerdem werden noch ein 
paar weitere Einstellungen gemacht. Dies geschieht mit:
%
\begin{Hint*}
\psset{%
  unit=.01\textwidth,%
  cornersize=absolute,%
  labelsep=.8ex,%
  linewidth=.4pt,%
  arrowscale=1.5,%
}
\end{Hint*}
%
Die innerhalb einer Gleitumgebung erzeugte \Package{pstricks}"=Grafik ist in 
\autoref{fig:pstricks} dargestellt.
%
\begin{Hint?}
\begin{figure}
\begin{pspicture}(0,-2)(100,64)
\newcommand{\fnodetext}{}
\def\fnodetext(#1)#2#3{%
  \fnode[framesize=18 8](#1){#2}%
  \rput(#1){\parbox{17\psunit}{\centering\hspace{0pt}#3}}%
}
\newcommand{\scriptbox}[2][24]{%
  \parbox{#1\psunit}{\scriptsize\centering\hspace{0pt}#2}%
}
\rput(15,10){%
  \rput(0,49){\textbf{Fahrer}}
  \fnodetext(0,40){NE}{Navigations\-ebene}
  \fnodetext(0,24){BE}{Bahnführungs\-ebene}
  \fnodetext(0,08){SE}{Stabilisierungs\-ebene}
  \ncline{->}{NE}{BE}
  \ncput*{\scriptbox{gewählte Fahrtroute\\ zeitlicher Ablauf}}
  \ncline{->}{BE}{SE}
  \ncput*{%
    \scriptbox{gewählte Führungsgrößen:\\ Sollspur,~Sollgeschwindigkeit}%
  }
  \psframe[dimen=middle,linewidth=1.2pt,linearc=5](-12.5,0)(12.5,54)
}
\rput(50,10){%
  \rput(0,17){\textbf{Fahrzeug}}
  \fnodetext(0,8){FZ}{Längs- und\\ Querdynamik}
  \psframe[dimen=middle,linewidth=1.2pt,linearc=5](-12.5,0)(12.5,22)
}
\rput(85,10){%
  \rput(0,49){\textbf{Umwelt}}
  \fnodetext(0,40){SN}{Straßennetz}
  \fnodetext(0,24){FR}{%
    Fahrraum\\ \smallskip\scriptsize{Straße und Verkehrssituation}%
  }
  \fnodetext(0,08){FO}{Fahrbahn\-oberfläche}
  \psframe[dimen=middle,linewidth=1.2pt,linearc=5](-12.5,0)(12.5,54)
}

\ncline{->}{SE}{FZ}
\naput{\scriptbox[9]{Stell\-größen}}
\nbput{\scriptbox[9]{Lenken Gasgeben Bremsen}}
\ncline{->}{FZ}{FO}
\naput{\scriptbox[9]{Regel\-größen}}
\nbput{\scriptbox[9]{Fahrzeugbewegung}}

\psset{armA=15,armB=0,angleA=0,angleB=180}
\ncangles{->}{FZ}{FR}
\ncangles{->}{FZ}{SN}

\psset{angleA=180,angleB=0}
\ncloop[loopsize=12,arm=4.5]{<-}{SE}{FO}
\naput{\scriptbox{Istspur, Istgeschwindigkeit}}
\ncloop[loopsize=32,arm=5]{<-}{BE}{FR}
\naput{\scriptbox[30]{Bereich sicherer Führungsgrößen}}
\ncloop[loopsize=52,arm=5.5]{<-}{NE}{SN}
\naput{\scriptbox{mögliche Fahrtroute}}
\end{pspicture}
\caption{Eine mit pstricks erstellte Grafik}\label{fig:pstricks}
\end{figure}
\end{Hint?}


\subsection{Auslagern von Grafiken in separate Dateien}
Damit \enquote{programmierte} Grafiken zum einen das Dokument respektive dessen 
Struktur nicht unnötig unübersichtlich machen und zum anderen nur nach einer 
Änderung abermals kompiliert werden, um die Kompilierungsdauer möglichst gering 
zu halten, können diese in separate Dateien ausgelagert werden. Hierfür bieten 
sich die Pakete \Package{subfiles}'manual' und \Package{standalone}'manual' an.



\section{Dokumentation von Quelltexten}
Sollen in einer Dokument Quelltextauszüge beschrieben werden oder ganze 
Quelltextdateien eingebunden werden, so empfiehlt sich für dieses Vorhaben das 
Paket \Package{listings}. Neben der bloßen Einbindung der Quelltexte bietet es 
zusätzlich die Möglichkeit, die Syntax in Abhängigkeit der verwendeten 
Programmiersprache hervorzuheben. Zusätzlich lässt sich ein Verzeichnis mit 
allen eingebundenen sowie direkt im Dokument angegebenen Quelltextauszügen 
erstellen.
%
\begin{Hint}
\usepackage{listings}
\end{Hint}
%
Wird das Paket in einem Dokument mit UTF"~8"~Kodierung verwendet, muss dies dem 
Paket in der Präambel mitgeteilt werden:
%
\begin{Hint}
\lstset{%
  inputencoding=utf8,extendedchars=true,
  literate=%
    {ä}{{\"a}}1 {ö}{{\"o}}1 {ü}{{\"u}}1
    {Ä}{{\"A}}1 {Ö}{{\"O}}1 {Ü}{{\"U}}1
    {~}{{\textasciitilde}}1 {ß}{{\ss}}1
}
\end{Hint}
%
Nachdem nun die Grundeinstellungen erfolgt sind, kann ein eigener Stil für eine 
gewünschte Programmiersprache definiert werden. Im folgenden Beispiel wird dies 
für ein MATLAB"~Skript demonstriert. Als erstes wird mit dem Befehl 
\Macro{lstdefinestyle}(\Package{listings}) ein eigener \Package{listings}"~Stil 
erstellt:
%
\begin{Hint*}
\lstdefinestyle{matlab}{%
  language=Matlab,%
  basicstyle=\ttfamily,%
  commentstyle={\color{green!50!black}},
  keywordstyle={\color{blue!75!black}},%
  columns=flexible,keepspaces,%
}
\end{Hint*}
%
Dabei wird die verwendete Programmiersprache definiert sowie die Formatierung 
für die Quelltextauszüge festgelegt. Für weitere Einstellungsmöglichkeiten 
sollte die Dokumentation zum Paket zu Rate gezogen werden. Der so definierte 
Stil lässt sich anschließend folgendermaßen verwenden.
%
\begin{Hint}
\begin{lstlisting}[style=matlab]
§\dots§
\end{lstlisting}
\end{Hint}
%
Soll innerhalb eines Quelltextauszuges ein Abschnitt nicht ausgegeben sondern 
als \hologo{LaTeX}"~Code interpretiert und ausgeführt werden, lässt sich mit 
\Macro{lstset}[%
  \PParameter{\PValue{escapechar=\PName{Zeichen}}}%
](\Package{listings}) ein spezielles Escape-Zeichen definieren. Alle innerhalb 
einer \Package{listings}"=Umgebung zwischen zwei Escape-Zeichen eingefügten 
Angaben in der Form \PName{Zeichen}\PValue{\dots}\PName{Zeichen} werden als 
\hologo{LaTeX}"~Code gelesen und ausgeführt. Außerdem ist es mit dem Befehl 
\Macro{lstnewenvironment}(\Package{listings}) möglich, eine individuelle 
\Package{listings}"=Umgebung zu definieren.

Das Einbinden und Darstellen von externen Quelldateien in ein Dokument ist über 
den Befehl
\Macro{lstinputlisting}[\LParameter\Parameter{Dateiname}](\Package{listings}) 
möglich, wobei dabei im optionalen Argument ebenfalls der definierte Stil 
angegeben werden sollte. Es folgt die Ausgabe eines kurzen Beispiels mit dem 
zuvor definierten Stil~\PValue{matlab}.
%
\begin{lstlisting}[style=matlab]
function perfect = isperfect(test_value)
  % This function checks to see if the given number is "perfect";
  %   a "perfect" number is an integer whose factors 
  %   (excluding itself) add up to it.
  %
  % Example:  6 is a perfect number: 1 + 2 + 3          =  6 ==  6
  %           8 is not:              1 + 2 + 4          =  7 ~=  8
  %          28 is a perfect number: 1 + 2 + 4 + 7 + 14 = 28 == 28
  %
  % Arguments:
  %
  %       test_value        (input)     check to see if this value 
  %                                     is perfect or not
  %       perfect           (output)    if yes, set this to 1
  %                                     if no, set this to 0
  
  % this will keep track of the sum of all integer divisors
  tempsum = 0;
  
  % we check all possible divisors
  for (divisor = 1 : test_value - 1)
    div_result = test_value / divisor;
    
    if (div_result == floor(div_result)) 
      % this is an integer divisor, so add it to the sum
      tempsum = tempsum + divisor;
    end
  end
  
  % now, does the sum equal the test_value?
  if (tempsum == test_value) 
    perfect = 1;
  else
    perfect = 0;
  end
end
\end{lstlisting}




\section{Und ganz zum Schluss \dots}
Mit Sicherheit werden Sie im Laufe der Zeit auf die eine oder andere Frage zu 
\hologo{LaTeXe} stoßen, die in diesem Tutorial nicht behandelt wurde. Sollten 
Sie an diesen Punkt gelangen, so empfehle zunächst einen Blick in die Kapitel 
\tudhyperref{sec:packages}{Benötigte, unterstützte und empfehlenswerte Pakete} 
sowie \tudhyperref{sec:tips}{Praktische Tipps \& Tricks} im Handbuch zu 
\TUDScript. Sollte Ihre Frage trotzdem unbeantwortet bleiben, so wäre der 
Besuch eines \hologo{LaTeX}"=Forums der nächste Schritt. Für allgemeine 
Anfragen kann ich unter anderem folgende Foren empfehlen:
%
\begin{itemize}
\item \hrfn{http://mrunix.de/forums/forumdisplay.php?38-LaTeX-Forum}{mrunix}
\item \hrfn{http://golatex.de/}{golatex}
\item \hrfn{http://tex.stackexchange.com/}{\hologo{LaTeX} Stack Exchange}
\item \hrfn{http://www.latex-community.org/forum/}{\hologo{LaTeX} Community}
\end{itemize} 
%
Insbesondere alle \TUDScript betreffende Fragen sollten natürlich direkt im 
dazugehörigen \Forum gestellt werden.

\begin{Trunk+}
\printbibliography[heading=bibintoc]\label{sec:bibliography}%
\end{Trunk+}
\printbibliography[heading=bibintoc]

\begin{Preamble+}
\usepackage[colorlinks,linkcolor=blue]{hyperref}

\end{Preamble+}

\addsec{Index}
Da das Tutorial etwas umfangreicher ist, wird für alle erläuterten Optionen, 
Umgebungen und Befehle ein Index erstellt. Bei Elementen, die weder \TUDScript 
noch \KOMAScript{} zuzuordnen sind, wird das dazugehörige Paket mit angeführt.
\PrintIndex

\FinishTutorial[%
  Um das im kopierten Beispiel erstellte Literaturverzeichnis in das Dokument 
  einbinden zu können, bedarf es dem einmaligen Aufruf von \Application{biber} 
  frühestens nach dem ersten Durchlauf von \Engine{pdfLaTeX}. Dies erfolgt mit 
  dem Aufruf \Path{biber }\PName{Dateiname}. Für das Erzeugen von Abkürzungs- 
  und Symbolverzeichnis ist das Ausführen des Perl"=Skriptes  
  \Path{makeglossaries }\PName{Dateiname} ausreichend. Danach ist ein weiteres 
  mal die Verwendung von \Path{pdflatex }\PName{Dateiname} notwendig, um die 
  Verzeichnisse in das Dokument einzubinden. Gegebenenfalls ist für korrekte 
  Seitenangaben in Abkürzungs- und Symbolverzeichnis ein abermaliges Ausführen 
  des Skriptes notwendig.
  \Attention{%
    Nutzer von \Distribution{\hologo{MiKTeX}} müssen für die Verwendung des 
    Perl"~Skriptes \File{makeglossaries}(\Package{glossaries}) unter Umständen 
    einen separaten Perl"=Interpreter installieren. In \autoref{sec:glossaries} 
    ist dazu mehr zu finden.%
  }%
]
\ListOfToDo
\end{document}
