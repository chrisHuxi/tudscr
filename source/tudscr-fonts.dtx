% \CheckSum{2259}
% \iffalse meta-comment
%
%  TUD-Script -- Corporate Design of Technische Universität Dresden
% ----------------------------------------------------------------------------
%
%  Copyright (C) Falk Hanisch <hanisch.latex@outlook.com>, 2012-2018
%
% ----------------------------------------------------------------------------
%
%  This work may be distributed and/or modified under the conditions of the
%  LaTeX Project Public License, version 1.3c of the license. The latest
%  version of this license is in http://www.latex-project.org/lppl.txt and
%  version 1.3c or later is part of all distributions of LaTeX 2005/12/01
%  or later and of this work. This work has the LPPL maintenance status
%  "author-maintained". The current maintainer and author of this work
%  is Falk Hanisch.
%
% ----------------------------------------------------------------------------
%
%  Dieses Werk darf nach den Bedingungen der LaTeX Project Public Lizenz
%  in der Version 1.3c, verteilt und/oder verändert werden. Die aktuelle
%  Version dieser Lizenz ist http://www.latex-project.org/lppl.txt und
%  Version 1.3c oder später ist Teil aller Verteilungen von LaTeX 2005/12/01
%  oder später und dieses Werks. Dieses Werk hat den LPPL-Verwaltungs-Status
%  "author-maintained", wird somit allein durch den Autor verwaltet. Der
%  aktuelle Verwalter und Autor dieses Werkes ist Falk Hanisch.
%
% ----------------------------------------------------------------------------
%
% \fi
%
% \CharacterTable
%  {Upper-case    \A\B\C\D\E\F\G\H\I\J\K\L\M\N\O\P\Q\R\S\T\U\V\W\X\Y\Z
%   Lower-case    \a\b\c\d\e\f\g\h\i\j\k\l\m\n\o\p\q\r\s\t\u\v\w\x\y\z
%   Digits        \0\1\2\3\4\5\6\7\8\9
%   Exclamation   \!     Double quote  \"     Hash (number) \#
%   Dollar        \$     Percent       \%     Ampersand     \&
%   Acute accent  \'     Left paren    \(     Right paren   \)
%   Asterisk      \*     Plus          \+     Comma         \,
%   Minus         \-     Point         \.     Solidus       \/
%   Colon         \:     Semicolon     \;     Less than     \<
%   Equals        \=     Greater than  \>     Question mark \?
%   Commercial at \@     Left bracket  \[     Backslash     \\
%   Right bracket \]     Circumflex    \^     Underscore    \_
%   Grave accent  \`     Left brace    \{     Vertical bar  \|
%   Right brace   \}     Tilde         \~}
%
% \iffalse
%%% From File: tudscr-fonts.dtx
%<*driver>
\ifx\ProvidesFile\@undefined\def\ProvidesFile#1[#2]{}\fi
\ProvidesFile{tudscr-fonts.dtx}[%
  2018/07/24 v2.06 TUD-Script\space%
%</driver>
%<package&identify>\ProvidesPackage{tudscrfonts}[%
%<*driver|package&identify>
%!TUD@Version
%<package>  package
  (corporate design fonts)%
]
%</driver|package&identify>
%<*driver>
\documentclass[english,ngerman,xindy]{tudscrdoc}
\usepackage[T1]{fontenc}
\ifpdftex{\usepackage[ngerman=ngerman-x-latest]{hyphsubst}}{}
\usepackage{babel}
\usepackage{tudscrfonts}
\KOMAoptions{parskip=half-}
\usepackage{bookmark}
\usepackage[babel]{microtype}

\CodelineIndex
\RecordChanges
\GetFileInfo{tudscr-fonts.dtx}
\title{\file{\filename}}
\author{Falk Hanisch\qquad\expandafter\mailto\expandafter{\tudscrmail}}
\date{\fileversion\nobreakspace(\filedate)}

\begin{document}
  \maketitle
  \tableofcontents
  \DocInput{\filename}
\end{document}
%</driver>
% \fi
%
% \selectlanguage{ngerman}
%
% \changes{v2.02}{2014/06/23}{Paket \pkg{titlepage} nicht weiter unterstützt}^^A
% \changes{v2.02}{2014/07/08}{\cs{FamilyKeyState} wird von Optionen genutzt}^^A
% \changes{v2.02}{2014/08/29}{Verwendung von \pkg{mweights} unnötig}^^A
% \changes{v2.02}{2014/08/29}{Unterstützung für \pkg{fontspec}}^^A
% \changes{v2.04}{2015/02/27}{Interaktion mit Paketen insbesondere für 
%   den Mathematiksatz verbessert}^^A
% \changes{v2.04c}{2015/11/29}{Bugfix für \pkg{fontspec} bei fehlenden 
%   Schriften}^^A
%
% \section{Die Hausschriften der \TnUD}
%
% Für das \TUDCD wird die Schrift \OpenSans benötigt. Die Type1-Variante wird 
% mit diesen Klassen mitgeliefert, die OTF-Variante muss entweder durch das 
% Paket \pkg{opensans} oder über das System bereitgestellt werden. Hier werden
% die Optionen und Befehle zur Verwendung der Schriften im Dokument definiert.
%
% Die Schriften des \CDs können nicht nur mit den \TUDScript-Klassen sondern 
% auch mit anderen \LaTeX-Dokumentklassen verwendet werden. Hierfür wird das
% Paket \pkg{tudscrfonts} bereitgestellt. Damit kann auf die Installtion der 
% Schriftfamilien für die alten Klassen verzichtet werden. Im Vergleich zu 
% dieser Installation werden dabei fehlende Glyphen ergänzt und Probleme beim
% Kerning behoben.
%
% \StopEventually{\PrintIndex\PrintChanges\PrintToDos}
%
% \iffalse
%<*class&option>
% \fi
%
% \subsection{%
%   Verwendung durch die \TUDScript-Klassen und das Paket \pkg{tudscrfonts}%
% }
%
% Die Schriften des \TUDCDs werden standardmäßig durch die \TUDScript-Klassen 
% verwendet. Für Dokumentklassen, welche nicht zum \TUDScript-Bundle gehören,
% können diese über das Paket \pkg{tudscrfonts} geladen werden.
%    \begin{macrocode}
\PreventPackageFromLoading[%
  \ClassWarning{\TUD@Class@Name}{%
    The package `tudscrfonts' must not be used with\MessageBreak%
    a TUD-Script class%
  }%
]{tudscrfonts}
%    \end{macrocode}
%
% \iffalse
%</class&option>
%<*package&identify>
% \fi
%
% Sollte keine \KOMAScript-Klasse genutzt werden, wird zusätzlich das Paket
% \pkg{scrextend} benötigt. 
%    \begin{macrocode}
\ifcsname KOMAClassName\endcsname\else
  \RequirePackage{scrextend}[2013/12/19]
\fi
%    \end{macrocode}
% Wird das Paket \pkg{tudscrfonts} nach einer der Klassen verwendet, welche die 
% veralteten Schriftfamilien für \Univers und \DIN definieren, so wird eine 
% Warnung mit dem Hinweis auf das Paket \pkg{fix-tudscrfonts} ausgegeben.
%    \begin{macrocode}
\@tempswafalse%
\@for\@tempa:={%
  tudbook,tudbeamer,tudfax,tudform,tudhaus,tudletter,tudposter,tudmathposter%
}\do{%
  \@ifclassloaded{\@tempa}{%
    \if@tempswa\else\@tempswatrue\let\@tempb\@tempa\fi%
  }{}%
}%
\if@tempswa%
  \@ifpackageloaded{fix-tudscrfonts}{}{%
    \PackageWarning{tudscrfonts}{%
      As you loaded class `\@tempb', you should\MessageBreak%
      load `fix-tudscrfonts' before the class\MessageBreak%
      with `\string\RequirePackage'%
    }%
  }%
\fi%
%    \end{macrocode}
%
% \iffalse
%</package&identify>
%<*body>
% \fi
%
% \subsection{Auswahl des Fontformates}
%
% Das \TUDScript-Bundle unterstützt sowohl die Schriftauswahl im klassischen 
% New Font Selection Scheme (NFSS) als auch das Einbinden der Schriften im 
% OpenType"=Format aus dem System über das Paket \pkg{fontspec} für Lua\LaTeX 
% bzw. Xe\LaTeX, wobei die letztere Variante nicht ideal ist. Zum einen 
% funktioniert das Kerning nicht sonderlich gut und zum anderen sind keine 
% Schriften für den mathematischen Satz vorhanden.
%
% \begin{macro}{\if@tud@x@fontspec@enabled}
% \changes{v2.02}{2014/08/29}{neu}^^A
% \begin{macro}{\if@tud@x@fontspec@math@enabled}
% \changes{v2.06}{2018/07/02}{neu}^^A
% Wird das Paket \pkg{fontspec} durch den Nutzer geladen, wird der passende 
% Schlüssel gesetzt und die Schrifteinstellungen erfolgen mit den dazugehörigen 
% Befehlen. Dies wurde in früheren Versionen mit der Option \opt{fontspec} 
% realisiert, die auch weiterhin bereitgehalten wird. Deshalb muss der Schalter 
% bereits vor der Abarbeitung der Optionen definiert werden. 
%    \begin{macrocode}
\newif\if@tud@x@fontspec@enabled
\newif\if@tud@x@fontspec@math@enabled
\PassOptionsToPackage{no-math}{fontspec}
\AfterPackage*{fontspec}{%
  \@ifpackagelater{fontspec}{2017/01/24}{%
    \@tud@x@fontspec@enabledtrue%
    \@ifpackagewith{fontspec}{no-math}{}{%
      \@tud@x@fontspec@math@enabledtrue%
    }%
  }{%
    \@tud@x@fontspec@enabledfalse%
%<*class>
    \ClassWarningNoLine{\TUD@Class@Name}%
%</class>
%<*package>
    \PackageWarningNoLine{tudscrfonts}%
%</package>
    {%
      Version `2017/01/24' of package `fontspec' is needed,\MessageBreak%
      but only version\MessageBreak%
      `\csname ver@fontspec.sty\endcsname'\MessageBreak%
      is available, so a fallback layout is used%
    }%
  }%
}
%    \end{macrocode}
% \end{macro}^^A \if@tud@x@fontspec@math@enabled
% \end{macro}^^A \if@tud@x@fontspec@enabled
% \begin{macro}{\if@tud@x@mweights@enabled}
% \changes{v2.04}{2015/04/24}{Erkennung des Paketes \pkg{mweights}}^^A
% Das Paket \pkg{mweights} wird geladen, um unterschiedliche Schriftschnitte
% für beispielsweise die TypeWriter-Schriften nutzen zu können.
%    \begin{macrocode}
\TUD@RecommendPackage{mweights}[2017/03/30]%
\newif\if@tud@x@mweights@enabled
\AfterPackage*{mweights}{\@tud@x@mweights@enabledtrue}
%    \end{macrocode}
% \end{macro}^^A \if@tud@x@mweights@enabled
%
% \subsection{Prüfen der Schriftverfügbarkeit}
%
% Nachfolgend werden Makros definiert, mit denen geprüft werden kann, ob die 
% notwendigen Schriften im passenden Fontformat auch installiert sind. Seit 
% 2018 wird \OpenSans als die Hausschrift verwendet. Um weiterhin noch alte 
% Dokumente mit den Schriften \Univers und \DIN nutzen zu können, werden die 
% entsprechenden Befehle ebenfalls vorgehalten.
%
% \begin{macro}{\if@tud@cdfont@fam@exist}
% \changes{v2.06}{2018/07/12}{neu}^^A
% \begin{macro}{\if@tud@cdfont@db@exist}
% \changes{v2.06}{2018/07/12}{neu}^^A
% \begin{macro}{\tud@cdfont@fam}
% \changes{v2.06}{2018/07/12}{neu}^^A
% \begin{macro}{\tud@cdfont@fam@lf}
% \changes{v2.06}{2018/07/12}{neu}^^A
% \begin{macro}{\tud@cdfont@fam@osf}
% \changes{v2.06}{2018/07/12}{neu}^^A
% \begin{macro}{\tud@cdfont@db}
% Zu Beginn werden Schalter und Hilfsmakros sowohl für den Fließtext\dots
%    \begin{macrocode}
\newif\if@tud@cdfont@fam@exist
\newif\if@tud@cdfont@db@exist
\newcommand*\tud@cdfont@fam{fdd}
\newcommand*\tud@cdfont@fam@lf{fdd}
\newcommand*\tud@cdfont@fam@osf{fddj}
\if@tud@cdoldfont@active
  \renewcommand*\tud@cdfont@fam{lun}
  \renewcommand*\tud@cdfont@fam@lf{lun}
  \renewcommand*\tud@cdfont@fam@osf{lun}
  \newcommand*\tud@cdfont@db{0m6}
\fi
%    \end{macrocode}
% \end{macro}^^A \tud@cdfont@db
% \end{macro}^^A \tud@cdfont@fam@osf
% \end{macro}^^A \tud@cdfont@fam@lf
% \end{macro}^^A \tud@cdfont@fam
% \end{macro}^^A \if@tud@cdfont@db@exist
% \end{macro}^^A \if@tud@cdfont@fam@exist
% \begin{macro}{\tud@ttfont@fam}
% \changes{v2.06}{2018/07/12}{neu}^^A
% \begin{macro}{\tud@ttfont@rb}
% \changes{v2.06}{2018/07/12}{neu}^^A
% \begin{macro}{\tud@ttfont@lm}
% \changes{v2.06}{2018/07/12}{neu}^^A
% \dots als auch für die TypeWriter-Schriften definiert.
%    \begin{macrocode}
\newcommand*\tud@ttfont@fam{fddt}
\newcommand*\tud@ttfont@rb{fddt}
\newcommand*\tud@ttfont@lm{lmtt}
%    \end{macrocode}
% \end{macro}^^A \tud@ttfont@lm
% \end{macro}^^A \tud@ttfont@rb
% \end{macro}^^A \tud@ttfont@fam
% \begin{macro}{\if@tud@ttfont@rb@exist}
% \changes{v2.06}{2018/07/12}{neu}^^A
% Die TypeWriter-Schrift aus \pkg{roboto-mono} kann nur genutzt werden, wenn
% diese tatsächlich auch vorhanden ist.
%    \begin{macrocode}
\newif\if@tud@ttfont@rb@exist
\IfFileExists{roboto-mono.sty}{%
  \@tud@ttfont@rb@existtrue%
}{}
%    \end{macrocode}
% \end{macro}^^A \if@tud@ttfont@rb@exist
% \begin{macro}{\tud@cdfont@check}
% \changes{v2.04}{2015/03/10}{neu}^^A
% \changes{v2.06}{2018/07/12}{zur Auswahl der Schriftfamilie angepasst}^^A
% \begin{macro}{\tud@cdfont@missing}
% \changes{v2.05l}{2017/03/29}{neu}^^A
% \begin{macro}{\tud@cdfont@missing@wrn}
% \changes{v2.06}{2018/07/12}{neu}^^A
% Mit \cs{tud@cdfont@check} wird geprüft, ob die Schrift \OpenSans respektive 
% die veralteten Schriften \Univers und \DIN im OTF-Format installiert sind. 
% Die Schrift \OpenSans im PostScript-Format wird mit dem Bundle mitgeliefert, 
% die veralteten müssen aus lizenrechtlichen Gründen installiert werden. Beim
% Nichtvorhandensein der Schriften wird eine Warnung ausgegeben und eine
% Rückfallebene definiert.
% 
% Im optionalen Argument wird der Präfix für die OTF-Schriften angegeben, im
% ersten obligatorischen Argument die einzelnen Suffixe für die verschiedenen
% Schnitte. Im zweiten Argument die zu prüfenden Postscript-Schriftfamilien und
% im letzten der zu setzende Schalter.
%    \begin{macrocode}
\newcommand*\tud@cdfont@check[4][]{%
  \@tud@res@swatrue%
  \def\tud@res@a##1{%
    \@tud@res@swafalse%
    \tud@cdfont@missing{##1}%
  }%
%    \end{macrocode}
% Für jede der OTF-Schriften wird eine Warnung ausgegeben, falls diese nicht 
% installiert ist. Um sicher zu gehen, dass das Layout komplett genutzt werden 
% kann, ist es hinreichend, wenn eine der benötigten Schriften nicht existiert,
% um die Rückfallebene zu aktivieren.
%    \begin{macrocode}
  \if@tud@x@fontspec@enabled%
    \def\tud@res@b##1{\IfFontExistsTF{##1}{}{\tud@res@a{##1}}}%
    \@for\tud@res@c:={#2}\do{\tud@res@b{#1\tud@res@c}}%
  \else%
%    \end{macrocode}
% Ist \pkg{fontspec} nicht aktiv, wird auf die Type1-Schriften für den
% Fließtext für die aktuelle Kodierung im gleichen Maße geprüft.
%    \begin{macrocode}
    \def\tud@res@b##1##2{%
      \begingroup%
%    \end{macrocode}
% Der zu prüfenden Schriftname wird mit \cs{lowercase} in Minuskeln in 
% \cs{tud@res@c} expandiert.
%    \begin{macrocode}
        \edef\tud@res@c{\lowercase{\def\noexpand\tud@res@c{##1##2.fd}}}%
        \tud@res@c%
%    \end{macrocode}
% Existiert die Schrift nicht, wird die entsprechende Warnung ausgegeben.
%    \begin{macrocode}
        \edef\tud@res@a{%
          \endgroup%
          \noexpand\IfFileExists{\tud@res@c}{}{\noexpand\tud@res@a{\tud@res@c}}%
        }%
      \tud@res@a%
    }%
    \@for\tud@res@c:={#3}\do{\tud@res@b{\encodingdefault}{\tud@res@c}}%
  \fi%
%    \end{macrocode}
% Zuletzt wird der angegebene Schalter gesetzt.
%    \begin{macrocode}
  \providebool{#4}%
  \if@tud@res@swa%
    \booltrue{#4}%
  \else%
    \boolfalse{#4}%
  \fi%
}
\newcommand*\tud@cdfont@missing@wrn{}
\newcommand*\tud@cdfont@missing[1]{%
  \begingroup%
    \def\tud@res@a{%
      `#1' font not found\MessageBreak%
      for `\encodingdefault' encoding.\MessageBreak%
    }%
    \if@tud@x@fontspec@enabled%
      \appto\tud@res@a{%
        It seems, you haven't installed the required\MessageBreak%
        OTF-font on your system. You should provide\MessageBreak%
        the fonts manually to your OS%
        \if@tud@cdoldfont@active\else%
        \MessageBreak or install package `opensans'%
        \fi%
      }%
    \else%
      \if@tud@cdoldfont@active%
        \def\tud@res@b{T1,OT1}%
      \else%
        \def\tud@res@b{T1,LY1,OT1,T2A,T2B,T2C,X2,OT2}%
      \fi%
      \@tud@res@swafalse%
      \@for\tud@res@c:=\tud@res@b\do{%
        \ifstr{\tud@res@c}{\encodingdefault}{\@tud@res@swatrue}{}%
      }%
%    \end{macrocode}
% Wird eine unterstützte Kodierung verwendet, müssen die Schriften installiert 
% werden.
%    \begin{macrocode}
      \if@tud@res@swa%
        \appto\tud@res@a{%
          You should try to install the necessary fonts\MessageBreak%
          \if@tud@cdoldfont@active%
            with a installation script listed in the manual%
          \else%
            by updating your distribution%
          \fi%
        }%
      \else%
        \ifstr{\encodingdefault}{TU}{%
          \appto\tud@res@a{%
            You should either load package `fontspec'\MessageBreak%
            or %
          }%
        }{%
          \appto\tud@res@a{Please }%
        }%
        \appto\tud@res@a{%
          choose a possible encoding with package\MessageBreak%
          `fontenc' (\tud@res@b)%
        }%
      \fi%
    \fi%
    \appto\tud@res@a{.\MessageBreak A fallback layout is used}%
    \edef\tud@res@c{%
%<*class>
      \noexpand\ClassWarning{\TUD@Class@Name}{\tud@res@a}%
%</class>
%<*package>
      \noexpand\PackageWarning{tudscrfonts}{\tud@res@a}%
%</package>
    }%
    \xappto\tud@cdfont@missing@wrn{\expandonce\tud@res@c}%
  \endgroup%
}
%    \end{macrocode}
% \end{macro}^^A \tud@cdfont@missing@wrn
% \end{macro}^^A \tud@cdfont@missing
% \end{macro}^^A \tud@cdfont@check
%
% \subsection{Kompatibilität der Schriften}
% \subsubsection{Majuskel-ß für Unicode-Prozessoren}
%
% Es wird für die Majuskelvariante der Glyphe \enquote*{ß} eine Rückfallebene
% (Substitution mit \enquote*{SS}) definiert, da diese für Unicode-Engines 
% standardmäßig nicht bereitgestellt wird. Um diese zu erkennen, wird auf die 
% gleiche Methodik wie im Paket \pkg{newunicodechar} zurückgegriffen.
%    \begin{macrocode}
\begingroup
\catcode`\^=7
\catcode30=12
\catcode`\!=12
\edef\tud@reserved{\@gobble^^^^0021}
\expandafter\endgroup
%    \end{macrocode}
% Wurde eine Unicode-Engine erkannt (|^^^^0021| wird als ein Token gelesen), 
% dann werden die Kodierungen für kleines und großes \enquote*{ß} dahingehend 
% angepasst, dass diese \enquote*{ss} bzw. \enquote*{SS} verwenden, wenn die 
% entsprechende Glyphe nicht vorhanden ist.
%    \begin{macrocode}
\ifx\tud@reserved\@empty\relax
  \RequirePackage{newunicodechar}[2012/11/12]%
  \uccode"00DF="1E9E%
  \lccode"1E9E="00DF%
  \newunicodechar{^^^^00df}{\iffontchar\font"00DF \symbol{"00DF}\else ss\fi}%
  \newunicodechar{^^^^1e9e}{\iffontchar\font"1E9E \symbol{"1E9E}\else SS\fi}%
\fi
%    \end{macrocode}
%
% \subsubsection{Ausrichtung von Überschriften und das Paket \pkg{ragged2e}}
%
% \begin{macro}{\tud@raggedright}
% \begin{macro}{\tud@RaggedRight}
% Die Überschriften sollen laut \CD linksbündig und ohne Silbentrennung gesetzt
% werden. Normalerweise ist das dies beim linksbündigen Satz bei \LaTeX{} der
% Fall. Allerdings ist es mit dem Paket \pkg{ragged2e} möglich, das Verhalten
% für den Flattersatz zu ändern und die Silbentrennung zu aktivieren. Mit der
% Option \opt{newcommands} werden dabei die originalen Befehle überschrieben,
% vorher jedoch in \cs{LaTeX\dots} gesichert. Diese Makros dienen dazu, die
% Überschriften in jedem Fall~-- auch bei der Verwendung von \pkg{ragged2e} mit
% der Option \opt{newcommands}~-- ohne Trennungen zu setzen.
%    \begin{macrocode}
\newcommand*\tud@raggedright{\raggedright}
\newcommand*\tud@RaggedRight{\raggedright}
\AfterPackage*{ragged2e}{%
  \ifundef{\LaTeXraggedright}{}{%
    \renewcommand*\tud@raggedright{\LaTeXraggedright}%
  }%
  \renewcommand*\tud@RaggedRight{\RaggedRight}%
}
%    \end{macrocode}
% \end{macro}^^A \tud@RaggedRight
% \end{macro}^^A \tud@raggedright
%
% \subsubsection{Anpassungen für das Paket \pkg{siunitx}}
%
% \begin{macro}{\lseries}
% \begin{macro}{\if@tud@x@siunitx@enabled}
% \changes{v2.05m}{2017/05/29}{neu}
% Das Paket \pkg{siunitx} nutzt den Befehl \cs{lseries} für den Fall, dass für
% den Fließtext eine Schrift die Serie \val{l} verwendet. Dem wird hier 
% Rechnung getragen. Außerdem muss bei der Auswahl der Schriften evtl. auf das 
% Paket reagiert werden, weil es sich bei der Definition der Schriften für den 
% Mathematikmodus auf \cs{familydefault} bzw. \cs{rmfamily} verlässt.
%    \begin{macrocode}
\newif\if@tud@x@siunitx@enabled
\AfterPackage*{siunitx}{%
  \@tud@x@siunitx@enabledtrue%
  \providecommand*\lseries{\fontseries{l}\selectfont}%
}
%    \end{macrocode}
% \end{macro}^^A \if@tud@x@siunitx@enabled
% \end{macro}^^A \lseries
%
% \subsubsection{Anpassungen für die Klasse \cls{beamer}}
%
% Die \cls{beamer}-Klasse lädt standardmäßig serifenlose Mathematikschriften. 
% Dies soll verhindert werden, damit die Auswahl über die Option \opt{cdmath} 
% erfolgen kann.
%    \begin{macrocode}
%<*package>
\AfterClass*{beamer}{\usefonttheme[onlymath]{serif}}
%</package>
%    \end{macrocode}
%
% \subsubsection{Mathematikschriften in Verbindung mit dem Paket \pkg{bm}}
%
% \begin{macro}{\if@tud@x@bm@requested}
% Das Paket \pkg{bm} stellt den Befehl \cs{bm} für fette und kursive Symbole im 
% Mathematiksatz bereit. Damit dies funktioniert, muss das Laden des Paketes 
% auf das Ende der Präambel verzögert werden, um zuvor alle Einstellungen für 
% die mathematischen Symbole vornehmen zu können.
%    \begin{macrocode}
\newif\if@tud@x@bm@requested
%<*package>
\@ifpackageloaded{bm}{%
  \PackageWarning{tudscrfonts}{%
    Package `bm' must be loaded after `tudscrfonts'.\MessageBreak%
    Otherwise the functionality can not be guaranteed%
  }%
}{%
%</package>
\PreventPackageFromLoading[\@tud@x@bm@requestedtrue]{bm}
%<*package>
}
%</package>
%    \end{macrocode}
% \end{macro}^^A \if@tud@x@bm@requested
%
% \iffalse
%</body>
%<*option>
% \fi
%
% \subsection{Schriftauswahl und -optionen}
%
% Hier werden die Optionen und Befehle zur Schriftauswahl definiert.
%
% \subsubsection{Schriften für den Fließtext}
% \begin{option}{cdfont}
% \changes{v2.02}{2014/07/08}{Schlüsselwerte für \opt{cdhead}}^^A
% \changes{v2.02}{2014/08/29}{Schlüsselwerte für \opt{fontspec}}^^A
% \changes{v2.05}{2016/04/04}{Schlüsselwerte für \val{ultrabold}}^^A
% \changes{v2.05l}{2017/03/29}{Standardeinstellungen für \opt{cdmath}}^^A
% \begin{macro}{\if@tud@cdfont@active}
% \changes{v2.06}{2018/06/26}{neu}^^A
% \begin{macro}{\if@tud@cdfont@heavy}
% \changes{v2.06}{2018/06/26}{neu}^^A
% \begin{macro}{\if@tud@cdfont@ultrabold}
% \changes{v2.05}{2016/04/04}{neu}^^A
% \begin{macro}{\if@tud@cdfont@lining}
% \changes{v2.06}{2018/07/10}{neu}^^A
% \begin{macro}{\tud@cdfont@activate}
% \changes{v2.06}{2018/07/02}{neu}^^A
% \begin{macro}{\tud@cdfont@deactivate}
% \changes{v2.06}{2018/07/24}{neu}^^A
% Mit dieser Option wird die zentrale Benutzerschnittstelle für alle möglichen
% Schrifteinstellungen innerhalb Dokumentes geschaffen. Durch diese Option
% können sowohl die verwendete Schrift für den Fließtext, als auch die Stärke
% der Schrift geändert werden. Zusätzlich lässt sich die Verwendung von 
% Minuskelziffern im Fließtext aktivieren. Außerdem kann eingestellt werden, ob 
% für den Mathematiksatz serifenlose Schriften zum Einsatz kommen sollen.
%    \begin{macrocode}
\newif\if@tud@cdfont@active
\newif\if@tud@cdfont@heavy
\newif\if@tud@cdfont@ultrabold
\newif\if@tud@cdfont@lining
\@tud@cdfont@liningtrue
\TUD@key{cdfont}[true]{%
  \TUD@set@numkey{cdfont}{@tempa}{%
    \TUD@bool@numkey,%
    {light}{1},{lightfont}{1},{lite}{1},{litefont}{1},{noheavyfont}{1},%
    {heavy}{2},{heavyfont}{2},{bold}{2},{boldfont}{2},%
    {normalbold}{3},{boldnormal}{3},%
    {ultrabold}{4},{boldultra}{4},{heavybold}{4},{boldheavy}{4},%
    {extendedbold}{4},{boldextended}{4},%
    {lf}{5},{nf}{5},{lining}{5},{liningfigures}{5},{normalfigures}{5},%
    {osf}{6},{osfigures}{6},{oldstyle}{6},{oldstylefigures}{6},{oldnumbers}{6},%
    {nomath}{7},{nocdmath}{7},%
    {serifmath}{7},{serif}{7},{nosansmath}{7},{nosans}{7},%
    {math}{8},{cdmath}{8},{sansmath}{8},{sans}{8},%
    {upgreek}{9},{uprightgreek}{9},{uprightgreeks}{9},%
    {slgreek}{10},{slantedgreek}{10},{slantedgreeks}{10},%
    {itgreek}{10},{italicgreek}{10},{italicgreeks}{10},%
    {texgreek}{11},{latexgreek}{11},{texgreeks}{11},{latexgreeks}{11},%
    {standardgreek}{11},{standardgreeks}{11},%
    {fontspec}{12},{luatex}{12},{lualatex}{12}{xetex}{12},{xelatex}{12},%
    {nofontspec}{13},{nfss}{13},{pdftex}{13},{pdflatex}{13}%
%<*class>
    ,{nohead}{14},{nocdhead}{14},{noheadfont}{14},{noheadcdfont}{14},%
    {nocdheadfont}{14},{nobar}{14},{nobarfont}{14},%
    {head}{15},{headfont}{15},{cdhead}{15},{headcdfont}{15},{cdheadfont}{15},%
    {bar}{15},{barfont}{15},%
    {lighthead}{15},{lightheadfont}{15},{lightfonthead}{15},%
    {lightcdhead}{15},{lightheadcdfont}{15},{lightcdheadfont}{15},%
    {lightbar}{15},{lightbarfont}{15},{lightfontbar}{15},%
    {headlight}{15},{headfontlight}{15},{headlightfont}{15},%
    {cdheadlight}{15},{headcdfontlight}{15},{cdheadfontlight}{15},%
    {barlight}{15},{barfontlight}{15},{barlightfont}{15},%
    {litehead}{15},{liteheadfont}{15},{litefonthead}{15},%
    {litecdhead}{15},{liteheadcdfont}{15},{litecdheadfont}{15},%
    {litebar}{15},{litebarfont}{15},{litefontbar}{15},%
    {headlite}{15},{headfontlite}{15},{headlitefont}{15},%
    {cdheadlite}{15},{headcdfontlite}{15},{cdheadfontlite}{15},%
    {barlite}{15},{barfontlite}{15},{barlitefont}{15},%
    {noheavyhead}{15},{noheavyheadfont}{15},{noheavyfonthead}{15},%
    {noheavycdhead}{15},{noheavyheadcdfont}{15},{noheavycdheadfont}{15},%
    {noheavybar}{15},{noheavybarfont}{15},{noheavyfontbar}{15},%
    {headnoheavy}{15},{headfontnoheavy}{15},{headnoheavyfont}{15},%
    {cdheadnoheavy}{15},{headcdfontnoheavy}{15},{cdheadfontnoheavy}{15},%
    {barnoheavy}{15},{barfontnoheavy}{15},{barnoheavyfont}{15},%
    {heavyhead}{16},{heavyheadfont}{16},{heavyfonthead}{16},%
    {heavycdhead}{16},{heavyheadcdfont}{16},{heavycdheadfont}{16},%
    {heavybar}{16},{heavybarfont}{16},{heavyfontbar}{16},%
    {headheavy}{16},{headfontheavy}{16},{headheavyfont}{16},%
    {cdheadheavy}{16},{headcdfontheavy}{16},{cdheadfontheavy}{16},%
    {barheavy}{16},{barfontheavy}{16},{barheavyfont}{16}%
%</class>
  }{#1}%
  \ifx\FamilyKeyState\FamilyKeyStateProcessed%
%    \end{macrocode}
% Die ersten drei Werte dienen zum Umschalten auf die Schriften des \CDs für
% den Fließtext.
%    \begin{macrocode}
    \ifcase\@tempa\relax% false
      \tud@cdfont@deactivate%
    \or% true
      \@tud@cdfont@heavyfalse%
      \tud@cdfont@activate%
    \or% heavy
      \@tud@cdfont@heavytrue%
      \tud@cdfont@activate%
%    \end{macrocode}
% Für die Definition der fetten Schriftstärke kommt je nach Einstellung ein
% anderer Schriftschnitt zum Einsatz.
%    \begin{macrocode}
    \or% normalbold
      \@tud@cdfont@ultraboldfalse%
      \tud@cdfont@activate%
    \or% ultrabold
      \@tud@cdfont@ultraboldtrue%
      \tud@cdfont@activate%
%    \end{macrocode}
% Hier wird die Art der Ziffern für den Fleißtext festgelegt.
%    \begin{macrocode}
    \or% liningfigures
      \@tud@cdfont@liningtrue%
      \tud@cdfont@activate%
    \or% oldstylefigures
      \@tud@cdfont@liningfalse%
      \tud@cdfont@activate%
%    \end{macrocode}
% Die Mathematikschriften.
%    \begin{macrocode}
    \or% nomath
      \TUDoptions{cdmath=false}%
    \or% math
      \TUDoptions{cdmath=true}%
%    \end{macrocode}
% Die Neigung der großen griechischen Buchstaben.
%    \begin{macrocode}
    \or% uprightgreek
      \TUDoptions{slantedgreek=false}%
    \or% slantedgreek
      \TUDoptions{slantedgreek=true}%
    \or% standardgreek
      \TUDoptions{slantedgreek=standard}%
%    \end{macrocode}
% Die Verwendung von \pkg{fontspec}.
%    \begin{macrocode}
    \or% fontspec
      \TUDoptions{fontspec=true}%
    \or% nofontspec
      \TUDoptions{fontspec=false}%
%    \end{macrocode}
% Die Schrift für den Querbalken.
%    \begin{macrocode}
%<*class>
    \or% nohead
      \TUDoptions{cdhead=false}%
    \or% head
      \TUDoptions{cdhead=true}%
    \or% heavyhead
      \TUDoptions{cdhead=heavy}%
%</class>
    \fi%
%    \end{macrocode}
% Wird kein gültiger Wert für die Option angegeben, so wird davon ausgegangen, 
% dass es sich bei diesem um die gewünschte Schriftgröße handelt.
%    \begin{macrocode}
  \else%
    \ifdimen{#1}{%
      \ifdim#1>\z@\relax%
        \TUD@KOMAoptions{fontsize=#1}%
        \tud@cdfont@activate%
        \FamilyKeyStateProcessed%
      \fi%
    }{}%
  \fi%
}
\newcommand*\tud@cdfont@activate{%
  \@tud@cdfont@activetrue%
  \tud@locked@bool@preset{@tud@cdmath@active}{true}%
%    \end{macrocode}
% Die Schrift im Querbalken wird bei den Klassen angepasst.
%    \begin{macrocode}
%<*class>
  \if@tud@cdfont@heavy%
    \tud@locked@num@preset{tud@head@font@num}{2}%
  \else%
    \tud@locked@num@preset{tud@head@font@num}{1}%
  \fi%
%</class>
  \TUD@SpecialOptionAtDocument{tud@font@set}%
}
\newcommand*\tud@cdfont@deactivate{%
  \@tud@cdfont@activefalse%
  \tud@locked@bool@preset{@tud@cdmath@active}{false}%
%<*class>
  \tud@locked@num@preset{tud@head@font@num}{0}%
%</class>
  \TUD@SpecialOptionAtDocument{tud@font@set}%
}
%    \end{macrocode}
% \end{macro}^^A \tud@cdfont@deactivate
% \end{macro}^^A \tud@cdfont@activate
% \end{macro}^^A \if@tud@cdfont@lining
% \end{macro}^^A \if@tud@cdfont@ultrabold
% \end{macro}^^A \if@tud@cdfont@heavy
% \end{macro}^^A \if@tud@cdfont@active
% \end{option}^^A cdfont
% \begin{option}{cdoldfont}
% \changes{v2.06}{2018/07/12}{neu}^^A
% \begin{macro}{\if@tud@cdoldfont@active}
% \changes{v2.06}{2018/07/12}{neu}^^A
% \begin{macro}{\if@tud@cdoldfont@nodin}
% \changes{v2.06}{2018/07/12}{neu}^^A
% Seit 2018 wird \OpenSans als Hausschrift im \TUDCD genutzt. Für die Nutzung 
% der vormals verwendeten Schriften \Univers und \DIN für ältere Dokumente wird 
% die Option \pkg{cdoldfont} bereitgestellt. Damit werden die Befehle zur 
% Schriftauswahl für die \TUDScript-Klassen oder das Paket \pkg{tudscrfonts} 
% entsprechend angepasst.
%    \begin{macrocode}
\newif\if@tud@cdoldfont@active
\newif\if@tud@cdoldfont@nodin
\TUD@key{cdoldfont}[true]{%
  \TUD@set@numkey{cdoldfont}{@tempa}{%
    \TUD@bool@numkey,%
    {din}{1},{dinbold}{1},%
    {nodin}{2},{nodinbold}{2},%
    {onlydin}{3},{onlydinbold}{3}%
  }{#1}%
  \ifx\FamilyKeyState\FamilyKeyStateProcessed%
    \ifcase\@tempa\relax% false
      \@tud@cdoldfont@activefalse%
      \@tud@cdoldfont@nodintrue%
      \tud@cdfont@deactivate%
    \or% true/din
      \@tud@cdoldfont@activetrue%
      \@tud@cdoldfont@nodinfalse%
      \tud@cdfont@activate%
    \or% nodin
      \@tud@cdoldfont@activetrue%
      \@tud@cdoldfont@nodintrue%
      \tud@cdfont@activate%
    \or% onlydin
      \@tud@cdoldfont@activetrue%
      \@tud@cdoldfont@nodinfalse%
      \tud@cdfont@deactivate%
    \fi%
  \else%
    \@tud@cdoldfont@activetrue%
    \TUDoptions{cdfont=#1}%
  \fi%
}
%    \end{macrocode}
% Wenn die Option \opt{cdoldfont} beim Laden der Klasse genutzt wurde, kann sie
% auch im Dokument genutzt werden. Um möglichst einfach alte Dokumente mit den 
% alten Schriften setzen zu können, wird in diesem Fall die Option \opt{cdfont}
% um die alten Werte erweitert. Andernfalls wird \opt{cdoldfont} schlichtweg
% deaktiviert.
%    \begin{macrocode}
%<*class>
\AtEndOfClass%
%</class>
%<*package>
\AtEndOfPackage%
%</package>
{%
  \if@tud@cdoldfont@active%
    \TUD@key[.comp]{cdfont}{%
      \TUDoptions{cdoldfont=#1}%
    }%
  \else%
    \TUD@key{cdoldfont}[]{%
%<*class>
      \ClassError{\TUD@Class@Name}%
%</class>
%<*package>
      \PackageError{tudscrfonts}%
%</package>
      {Option `cdoldfont' not active}{%
        Option `cdoldfont' must be activated while loading the\MessageBreak%
%<*class>
        class `\TUD@Class@Name'\space%
%</class>
%<*package>
        package `tudscrfonts'\space%
%</package>
        but you have tried to set it up later.%
      }%
      \FamilyKeyStateProcessed%
    }%
  \fi%
}
%    \end{macrocode}
% \end{macro}^^A \if@tud@cdoldfont@nodin
% \end{macro}^^A \if@tud@cdoldfont@active
% \end{option}^^A cdoldfont
% \begin{option}{fontsize}
% \changes{v2.04}{2015/04/24}{Option aus \KOMAScript{} erweitert}^^A
% \begin{macro}{\if@tud@fontsize@set}
% \changes{v2.05}{2015/07/24}{neu}^^A
% Die genutzte Grundschriftgröße kann mit der \KOMAScript-Option \opt{fontsize} 
% eingestellt werden. Um ein aktives Setzen der Schriftgröße zu erkennen, wird 
% für das Laden der \TUDScript-Klasse ein Schlüssel definiert, der den Wert an
% \KOMAScript{} weiterreicht.
%    \begin{macrocode}
\newif\if@tud@fontsize@set
\TUD@key{fontsize}{%
  \@tud@fontsize@settrue%
  \TUD@KOMAoptions{fontsize=#1}%
  \FamilyKeyStateProcessed%
}
%<*class>
\AtEndOfClass%
%</class>
%<*package>
\AtEndOfPackage%
%</package>
{%
  \RelaxFamilyKey{TUD}{fontsize}%
  \DefineFamilyMember{KOMA}%
  \DefineFamilyKey{KOMA}{fontsize}{%
    \@tud@fontsize@settrue%
%<*class>
    \ClassWarning{\TUD@Class@Name}%
%</class>
%<*package>
    \PackageWarning{tudscrfonts}%
%</package>
    {%
      The key `fontsize=#1' should only be used\MessageBreak%
      as class option%
    }%
    \TUD@SpecialOptionAtDocument{tud@font@set}%
    \FamilyKeyStateProcessed%
  }%
}
%    \end{macrocode}
% \end{macro}^^A \if@tud@fontsize@set
% \end{option}^^A fontsize
% \begin{option}{relspacing}
% \changes{v2.05}{2015/12/17}{neu}^^A
% \begin{macro}{\if@tud@relspacing}
% \changes{v2.05}{2015/12/17}{neu}^^
% Die Option \opt{relspacing} setzt einige Abstände und Längen in Abhängigkeit 
% der aktuellen Grundschriftgröße, wenn sie aktiviert wurde. Die verwendete 
% Grundschriftgröße kann mit der \KOMAScript-Option \opt{fontsize} eingestellt
% werden. Die Definiton der Längen wird durch das Makro \cs{tud@font@skip@set} 
% vorgenommen.
%    \begin{macrocode}
\newif\if@tud@relspacing
\TUD@key{relspacing}[true]{%
  \TUD@set@numkey{relspacing}{@tempa}{%
    \TUD@bool@numkey,%
    {absolute}{0},{classic}{0},{standard}{0},%
    {relative}{1},{fontsize}{1},{font}{1},{size}{1}%
  }{#1}%
  \ifx\FamilyKeyState\FamilyKeyStateProcessed%
    \ifcase\@tempa\relax% false
      \@tud@relspacingfalse%
    \or% true
      \@tud@relspacingtrue%
    \fi%
    \TUD@SpecialOptionAtDocument{tud@font@skip@set}%
  \fi%
}
%    \end{macrocode}
% \end{macro}^^A \if@tud@relspacing
% \end{option}^^A relspacing
%
% \iffalse
%</option>
%<*body>
% \fi
%
% Mit dem Paket \pkg{trimspaces} werden für die alternative Schriftauswahl mit 
% \cs{cdfont} bzw. \cs{textcdfont} eventuell vorhandene Leerzeichen aus dem 
% Argument entfernt. Auch beim Einlesen von Feldern wird dies genutzt.
%    \begin{macrocode}
\RequirePackage{trimspaces}[2009/09/17]
%    \end{macrocode}
% \begin{macro}{\tud@font@set}
% \changes{v2.04}{2015/03/01}{Unterstützung für das Paket \pkg{mweights}}^^A
% \begin{macro}{\bfseries@rm}
% \changes{v2.04}{2015/04/24}{Unterstützung für das Paket \pkg{mweights}}^^A
% \begin{macro}{\mdseries@rm}
% \changes{v2.04}{2015/04/24}{Unterstützung für das Paket \pkg{mweights}}^^A
% \begin{macro}{\bfseries@sf}
% \changes{v2.04}{2015/04/24}{Unterstützung für das Paket \pkg{mweights}}^^A
% \begin{macro}{\mdseries@sf}
% \changes{v2.04}{2015/04/24}{Unterstützung für das Paket \pkg{mweights}}^^A
% \begin{macro}{\bfseries@tt}
% \changes{v2.04}{2015/04/24}{Unterstützung für das Paket \pkg{mweights}}^^A
% \begin{macro}{\mdseries@tt}
% \changes{v2.04}{2015/04/24}{Unterstützung für das Paket \pkg{mweights}}^^A
% \begin{macro}{\rmdef@ult}
% \changes{v2.04}{2015/04/24}{Unterstützung für das Paket \pkg{mweights}}^^A
% \begin{macro}{\sfdef@ult}
% \changes{v2.04}{2015/04/24}{Unterstützung für das Paket \pkg{mweights}}^^A
% \begin{macro}{\ttdef@ult}
% \changes{v2.04}{2015/04/24}{Unterstützung für das Paket \pkg{mweights}}^^A
% \begin{macro}{\bfdef@ult}
% \changes{v2.04}{2015/04/24}{Unterstützung für das Paket \pkg{mweights}}^^A
% \begin{macro}{\mddef@ult}
% \changes{v2.04}{2015/04/24}{Unterstützung für das Paket \pkg{mweights}}^^A
% Mit \cs{tud@font@set} werden entweder die normalen oder aber die Schriften 
% des \CDs aktiviert. Dabei werden die Pakete \pkg{fontspec} und \pkg{mweights} 
% beachtet. 
%    \begin{macrocode}
\newcommand*\tud@font@set{%
%    \end{macrocode}
% Ist die Verwendung der Schriften des \CDs gewünscht, werden die Schnitte der 
% Standardschriften vorher vorher gesichert, um diese später vollständig 
% wiederherstellen zu können.
%    \begin{macrocode}
  \if@tud@cdfont@active%
    \tud@cs@store{rmdefault}%
    \tud@cs@store{sfdefault}%
    \tud@cs@store{ttdefault}%
    \tud@cs@store{bfdefault}%
    \tud@cs@store{mddefault}%
    \tud@cs@store{updefault}%
    \tud@cs@store{sldefault}%
    \tud@cs@store{itdefault}%
    \tud@cs@store{familydefault}%
    \tud@cs@store{seriesdefault}%
    \tud@cs@store{shapedefault}%
    \if@tud@x@mweights@enabled%
      \tud@cs@store{bfseries@rm}%
      \tud@cs@store{mdseries@rm}%
      \tud@cs@store{bfseries@sf}%
      \tud@cs@store{mdseries@sf}%
      \tud@cs@store{bfseries@tt}%
      \tud@cs@store{mdseries@tt}%
    \fi%
%    \end{macrocode}
% Das Paket \pkg{fontspec} definiert u.\,U. die Familienauswahlbefehle um, so
% dass diese ebenfalls gesichert werden.
%    \begin{macrocode}
    \if@tud@x@fontspec@enabled%
      \tud@cs@store{rmfamily}%
      \tud@cs@store{sffamily}%
      \tud@cs@store{ttfamily}%
    \fi%
%    \end{macrocode}
% Zunächst werden die Standardwerte für die Schriftfamilien gesetzt.
%    \begin{macrocode}
    \renewcommand*\rmdefault{\sfdefault}%
    \renewcommand*\updefault{n}%
    \renewcommand*\itdefault{it}%
    \renewcommand*\sldefault{sl}%
    \renewcommand*\familydefault{\sfdefault}%
    \renewcommand*\seriesdefault{\mddefault}%
    \renewcommand*\shapedefault{\updefault}%
%    \end{macrocode}
% Danach erfolgt die Definition der Schriften des \CDs. 
%    \begin{macrocode}
    \tud@cdfont@set%
    \tud@ttfont@set%
    \if@tud@x@mweights@enabled%
      \edef\bfseries@rm{\tud@cdfont@bf}%
      \edef\mdseries@rm{\tud@cdfont@md}%
      \edef\bfseries@sf{\tud@cdfont@bf}%
      \edef\mdseries@sf{\tud@cdfont@md}%
      \edef\bfseries@tt{\tud@ttfont@bf}%
      \edef\mdseries@tt{\tud@ttfont@md}%
    \fi%
    \let\bfdefault\tud@cdfont@bf%
    \let\mddefault\tud@cdfont@md%
%    \end{macrocode}
% Werden die Schriften des \CDs deaktiviert, wird auf die zu Dokumentbeginn
% gesicherten Schriften\footnote{mit und ohne Serifen} zurückgeschaltet.
%    \begin{macrocode}
  \else%
    \tud@cs@restore{rmdefault}%
    \tud@cs@restore{sfdefault}%
    \tud@cs@restore{ttdefault}%
    \tud@cs@restore{bfdefault}%
    \tud@cs@restore{mddefault}%
    \tud@cs@restore{updefault}%
    \tud@cs@restore{sldefault}%
    \tud@cs@restore{itdefault}%
    \tud@cs@restore{familydefault}%
    \tud@cs@restore{seriesdefault}%
    \tud@cs@restore{shapedefault}%
    \if@tud@x@mweights@enabled%
      \tud@cs@restore{bfseries@rm}%
      \tud@cs@restore{mdseries@rm}%
      \tud@cs@restore{bfseries@sf}%
      \tud@cs@restore{mdseries@sf}%
      \tud@cs@restore{bfseries@tt}%
      \tud@cs@restore{mdseries@tt}%
    \fi%
    \if@tud@x@fontspec@enabled%
      \tud@cs@restore{rmfamily}%
      \tud@cs@restore{sffamily}%
      \tud@cs@restore{ttfamily}%
    \fi%
  \fi%
  \if@tud@x@mweights@enabled%
    \edef\rmdef@ult{\rmdefault}%
    \edef\sfdef@ult{\sfdefault}%
    \edef\ttdef@ult{\ttdefault}%
    \edef\bfdef@ult{\bfdefault}%
    \edef\mddef@ult{\mddefault}%
    \edef\famdef@ult{\familydefault}%
  \fi%
%    \end{macrocode}
% Zum Ende werden einige Abstände abhängig von der gewählten Schriftgröße 
% definiert. Danach erfolgt die Behandlung der Mathematikschriften und die 
% gewählte Standardschrift aktiviert. Zum Schluss wird das Layout erneuert.
%    \begin{macrocode}
  \ifundef{\tud@math@init}{\tud@math@set}{}%
  \tud@font@skip@set%
  \normalfont\selectfont%
%<*class>
  \tud@cd@process%
  \global\@tud@head@font@settrue%
%</class>
}
%    \end{macrocode}
% \end{macro}^^A \mddef@ult
% \end{macro}^^A \bfdef@ult
% \end{macro}^^A \ttdef@ult
% \end{macro}^^A \sfdef@ult
% \end{macro}^^A \rmdef@ult
% \end{macro}^^A \mdseries@tt
% \end{macro}^^A \bfseries@tt
% \end{macro}^^A \mdseries@sf
% \end{macro}^^A \bfseries@sf
% \end{macro}^^A \mdseries@rm
% \end{macro}^^A \bfseries@rm
% \end{macro}^^A \tud@font@set
% \begin{macro}{\tud@cdfont@set}
% \changes{v2.06}{2018/06/28}{neu}^^A
% \begin{macro}{\tud@cdfont@@set}
% \changes{v2.06}{2018/06/28}{neu}^^A
% \begin{macro}{\tud@cdfont@md}
% \changes{v2.06}{2018/06/28}{neu}^^A
% \begin{macro}{\tud@cdfont@bf}
% \changes{v2.06}{2018/06/28}{neu}^^A
% \begin{macro}{\tud@cdfont@ebf}
% \changes{v2.06}{2018/06/28}{neu}^^A
% Hiermit erfolgt die eigentliche Auswahl der Schriftfamilien für das \CD.
%    \begin{macrocode}
\newcommand*\tud@cdfont@set{%
  \if@tud@cdfont@fam@exist%
    \if@tud@cdfont@lining%
      \let\tud@cdfont@fam\tud@cdfont@fam@lf%
    \else%
      \let\tud@cdfont@fam\tud@cdfont@fam@osf%
    \fi%
%    \end{macrocode}
% Das Aktivieren der Schriften erfolgt entweder mit dem Paket \pkg{fontspec}
% oder\dots
%    \begin{macrocode}
    \if@tud@x@fontspec@enabled%
      \setsansfont{\tud@cdfont@fam}%
%    \end{macrocode}
% \dots nach dem NFSS.
%    \begin{macrocode}
    \else%
      \renewcommand*\sfdefault{\tud@cdfont@fam}%
    \fi%
    \tud@cdfont@@set%
  \fi%
%    \end{macrocode}
% Sollte versucht worden sein, die Schriften zu aktivieren, obwohl diese nicht
% installiert sind, werden die entrsprechenden Warnungen ausgegeben.
%    \begin{macrocode}
  \@tud@res@swafalse%
  \if@tud@cdfont@fam@exist\else%
    \@tud@res@swatrue%
  \fi%
  \if@tud@cdoldfont@active%
    \if@tud@cdoldfont@nodin\else\if@tud@cdfont@db@exist\else%
      \@tud@res@swatrue%
    \fi\fi%
  \fi%
  \if@tud@res@swa%
    \tud@cdfont@missing@wrn%
  \fi%
}
\newcommand*\tud@cdfont@md{}
\newcommand*\tud@cdfont@bf{}
\newcommand*\tud@cdfont@ebf{}
\newcommand*\tud@cdfont@@set{%
  \if@tud@cdfont@heavy%
    \renewcommand*\tud@cdfont@md{m}%
    \if@tud@cdfont@ultrabold%
      \renewcommand*\tud@cdfont@bf{b}%
      \renewcommand*\tud@cdfont@ebf{eb}%
    \else%
      \renewcommand*\tud@cdfont@bf{sb}%
      \renewcommand*\tud@cdfont@ebf{b}%
    \fi%
  \else%
    \renewcommand*\tud@cdfont@md{l}%
    \if@tud@cdfont@ultrabold%
      \renewcommand*\tud@cdfont@bf{sb}%
      \renewcommand*\tud@cdfont@ebf{b}%
    \else%
      \renewcommand*\tud@cdfont@bf{m}%
      \renewcommand*\tud@cdfont@ebf{sb}%
    \fi%
  \fi%
}
\if@tud@cdoldfont@active
  \renewcommand*\tud@cdfont@@set{%
    \if@tud@cdfont@heavy%
      \renewcommand*\tud@cdfont@md{m}%
      \if@tud@cdfont@ultrabold%
        \renewcommand*\tud@cdfont@bf{eb}%
        \renewcommand*\tud@cdfont@ebf{eb}%
      \else%
        \renewcommand*\tud@cdfont@bf{b}%
        \renewcommand*\tud@cdfont@ebf{eb}%
      \fi%
    \else%
      \renewcommand*\tud@cdfont@md{l}%
      \if@tud@cdfont@ultrabold%
        \renewcommand*\tud@cdfont@bf{b}%
        \renewcommand*\tud@cdfont@ebf{eb}%
      \else%
        \renewcommand*\tud@cdfont@bf{m}%
        \renewcommand*\tud@cdfont@ebf{b}%
      \fi%
    \fi%
  }%
\fi
%    \end{macrocode}
% \end{macro}^^A \tud@cdfont@ebf
% \end{macro}^^A \tud@cdfont@bf
% \end{macro}^^A \tud@cdfont@md
% \end{macro}^^A \tud@cdfont@@set
% \end{macro}^^A \tud@cdfont@set
% \begin{macro}{\tud@font@skip@set}
% \changes{v2.04}{2015/02/17}{neu}^^A
% \begin{macro}{\tud@font@skip@@set}
% \changes{v2.04}{2015/03/27}{neu}^^A
% \begin{macro}{\smallskipamount}
% \changes{v2.04}{2015/04/24}{Wert abhängig von der Schriftgröße}^^A
% \begin{macro}{\medskipamount}
% \changes{v2.04}{2015/04/24}{Wert abhängig von der Schriftgröße}^^A
% \begin{macro}{\bigskipamount}
% \changes{v2.04}{2015/04/24}{Wert abhängig von der Schriftgröße}^^A
% \begin{macro}{\abovecaptionskip}
% \changes{v2.04}{2015/04/24}{Wert abhängig von der Schriftgröße}^^A
% \begin{macro}{\belowcaptionskip}
% \begin{macro}{\columnsep}
% \changes{v2.04}{2015/04/24}{Wert abhängig von der Schriftgröße}^^A
% Mit diesem Befehl wird nach der Auswahl der Schriftgröße die Länge einiger 
% Abstände angepasst, die bei den \KOMAScript-Klassen nicht abhängig von dieser
% sind.
%    \begin{macrocode}
\newcommand*\tud@font@skip@set{%
%    \end{macrocode}
% Es werden die Standardeinstellungen abhängig von \opt{cdfont} entweder 
% wiederhergestellt oder gesichert und neu definiert.
%    \begin{macrocode}
  \if@tud@relspacing%
    \tud@skip@store{smallskipamount}%
    \tud@skip@store{medskipamount}%
    \tud@skip@store{bigskipamount}%
    \tud@skip@store{abovecaptionskip}%
    \tud@skip@store{belowcaptionskip}%
    \tud@skip@store{columnsep}%
    \tud@font@skip@@set%
  \else%
    \tud@skip@restore{smallskipamount}%
    \tud@skip@restore{medskipamount}%
    \tud@skip@restore{bigskipamount}%
    \tud@skip@restore{abovecaptionskip}%
    \tud@skip@restore{belowcaptionskip}%
    \tud@skip@restore{columnsep}%
  \fi%
}
%    \end{macrocode}
% Hiermit erfolgt das eigentliche Setzen der Abstände.
%    \begin{macrocode}
\newcommand*\tud@font@skip@@set{%
  \smallskipamount=\partopsep%
  \medskipamount=\partopsep%
  \multiply\medskipamount by 2%
  \bigskipamount=\partopsep%
  \multiply\bigskipamount by 4%
  \tud@skip@set{abovecaptionskip}{\dimexpr\bigskipamount*5/6\relax}%
  \tud@skip@set{belowcaptionskip}{\z@}%
  \tud@skip@set{columnsep}{1.1\baselineskip}%
}
%    \end{macrocode}
% \end{macro}^^A \columnsep
% \end{macro}^^A \belowcaptionskip
% \end{macro}^^A \abovecaptionskip
% \end{macro}^^A \bigskipamount
% \end{macro}^^A \medskipamount
% \end{macro}^^A \smallskipamount
% \end{macro}^^A \tud@font@skip@@set
% \end{macro}^^A \tud@font@skip@set
% \begin{macro}{\tud@x@fontspec@init}
% \changes{v2.06}{2018/06/28}{neu}^^A
% Dieses Makro wird innerhalb von \cs{tud@cdfont@init} aufgerufen, falls 
% \pkg{fontspec} aktiv ist. Das erste Argument enthält den NFSS-Namen der 
% Schriftfamilie, das zweite Argument spezifiziert die Schriften für die 
% jeweiligen Schriftschnitte.
%    \begin{macrocode}
\newcommand*\tud@x@fontspec@init[2]{%
%    \end{macrocode}
% Durch die Verwendung des optionalen Argumentes von \cs{defaultfontfeatures} 
% werden die gewünschten Features einem Schriftfamiliennamen zugewiesen. Diese 
% können anschließend einfach bspw. mit \cs{setsansfont}\marg{Name} genutzt 
% werden. Damit alle Änderungen lokal bleiben, werden die Argumente in einer 
% Gruppe expandiert.
%    \begin{macrocode}
  \begingroup%
    \edef\tud@res@a{%
      \noexpand\defaultfontfeatures[#1]{%
        NFSSFamily=#1,%
        Ligatures=TeX,%
        BoldFont={},%
        ItalicFont={},%
        BoldItalicFont={},%
        SmallCapsFont={},%
        #2%
      }%
    }%
%    \end{macrocode}
% Damit die Schrift mit ihrem NFSS-Namen auf jeden Fall definiert ist, auch 
% wenn die Schriften des \CDs nicht aktiviert wurden, wird \cs{newfontfamily} 
% aufgerufen.
%    \begin{macrocode}
    \edef\tud@res@b{%
      \noexpand\newfontfamily%
        \expandafter\noexpand\csname tud@x@fontspec@fam@#1\endcsname{#1}%
    }%
%    \end{macrocode}
% Die expandierten Argumente werden nach der Gruppe ausgeführt.
%    \begin{macrocode}
    \edef\tud@res@c{%
      \endgroup\expandonce\tud@res@a\expandonce\tud@res@b%
    }%
  \tud@res@c%
}
%    \end{macrocode}
% \end{macro}^^A \tud@x@fontspec@init
% \begin{macro}{\tud@cdfont@init}
% \changes{v2.04}{2015/03/10}{neu}^^A
% \begin{macro}{\cdfontln}
% \changes{v2.06}{2018/06/29}{neu}^^A
% \begin{macro}{\cdfontrn}
% \changes{v2.06}{2018/06/29}{neu}^^A
% \begin{macro}{\cdfontsn}
% \changes{v2.06}{2018/06/29}{neu}^^A
% \begin{macro}{\cdfontbn}
% \changes{v2.06}{2018/06/29}{neu}^^A
% \begin{macro}{\cdfontxn}
% \changes{v2.06}{2018/06/29}{neu}^^A
% \begin{macro}{\cdfontli}
% \changes{v2.06}{2018/06/29}{neu}^^A
% \begin{macro}{\cdfontri}
% \changes{v2.06}{2018/06/29}{neu}^^A
% \begin{macro}{\cdfontsi}
% \changes{v2.06}{2018/06/29}{neu}^^A
% \begin{macro}{\cdfontbi}
% \changes{v2.06}{2018/06/29}{neu}^^A
% \begin{macro}{\cdfontxi}
% \changes{v2.06}{2018/06/29}{neu}^^A
% \begin{macro}{\textcdln}
% \changes{v2.06}{2018/06/29}{neu}^^A
% \begin{macro}{\textcdrn}
% \changes{v2.06}{2018/06/29}{neu}^^A
% \begin{macro}{\textcdsn}
% \changes{v2.06}{2018/06/29}{neu}^^A
% \begin{macro}{\textcdbn}
% \changes{v2.06}{2018/06/29}{neu}^^A
% \begin{macro}{\textcdxn}
% \changes{v2.06}{2018/06/29}{neu}^^A
% \begin{macro}{\textcdli}
% \changes{v2.06}{2018/06/29}{neu}^^A
% \begin{macro}{\textcdri}
% \changes{v2.06}{2018/06/29}{neu}^^A
% \begin{macro}{\textcdsi}
% \changes{v2.06}{2018/06/29}{neu}^^A
% \begin{macro}{\textcdbi}
% \changes{v2.06}{2018/06/29}{neu}^^A
% \begin{macro}{\textcdxs}
% \changes{v2.06}{2018/06/29}{neu}^^A
% Das Makro \cs{tud@cdfont@init} wird für die Defintion aller Befehle zur
% Auswahl der Schriften des \CDs verwendet.
%    \begin{macrocode}
\newcommand*\tud@cdfont@init{%
%    \end{macrocode}
% Zur Nutztung der OTF"~Schriften werden mit \cs{tud@x@fontspec@init} alle 
% benötigten Schriftfamilien mit ihren einzelnen Schriftschnitten definiert.
%    \begin{macrocode}
  \if@tud@cdfont@fam@exist%
    \if@tud@x@fontspec@enabled%
      \def\tud@res@a{%
        UprightFont={OpenSansRegular},%
        FontFace={l}{n}{OpenSansLight},%
        FontFace={m}{n}{OpenSansRegular},%
        FontFace={sb}{n}{OpenSansSemiBold},%
        FontFace={b}{n}{OpenSansBold},%
        FontFace={bx}{n}{OpenSansBold},%
        FontFace={eb}{n}{OpenSansExtraBold},%
        FontFace={l}{it}{OpenSansLightItalic},%
        FontFace={m}{it}{OpenSansItalic},%
        FontFace={sb}{it}{OpenSansSemiBoldItalic},%
        FontFace={b}{it}{OpenSansBoldItalic},%
        FontFace={bx}{it}{OpenSansBoldItalic},%
        FontFace={eb}{it}{OpenSansExtraBoldItalic}%
      }%
      \tud@x@fontspec@init{\tud@cdfont@fam@osf}{Numbers=OldStyle,\tud@res@a}%
      \tud@x@fontspec@init{\tud@cdfont@fam@lf}{\tud@res@a}%
    \fi%
%    \end{macrocode}
% Es werden mehrere Schalter zur expliziten Auswahl für die Schriftschnitte
% \textcdln{Open~Sans~Light}, \textcdrn{Open~Sans~Regular}, 
% \textcdsn{Open~Sans~SemiBold}, \textcdbn{Open~Sans~Bold} und 
% \textcdxn{Open~Sans~ExtraBold} sowie \textcdli{Open~Sans~Light~Italic}, 
% \textcdri{Open~Sans~Regular~Italic}, \textcdsi{Open~Sans~SemiBold}
% \textcdbi{Open~Sans~Bold~Italic} und \textcdxi{Open~Sans~ExtraBold~Italic}
% definiert. Diese können so im Dokument unabhängig von der Option 
% \opt{cdfont} genutzt werden.
%    \begin{macrocode}
    \newrobustcmd*\cdfontln{\usefont{\encodingdefault}{\tud@cdfont@fam}{l}{n}}
    \newrobustcmd*\cdfontrn{\usefont{\encodingdefault}{\tud@cdfont@fam}{m}{n}}
    \newrobustcmd*\cdfontsn{\usefont{\encodingdefault}{\tud@cdfont@fam}{sb}{n}}
    \newrobustcmd*\cdfontbn{\usefont{\encodingdefault}{\tud@cdfont@fam}{b}{n}}
    \newrobustcmd*\cdfontxn{\usefont{\encodingdefault}{\tud@cdfont@fam}{eb}{n}}
    \newrobustcmd*\cdfontli{\usefont{\encodingdefault}{\tud@cdfont@fam}{l}{it}}
    \newrobustcmd*\cdfontri{\usefont{\encodingdefault}{\tud@cdfont@fam}{m}{it}}
    \newrobustcmd*\cdfontsi{\usefont{\encodingdefault}{\tud@cdfont@fam}{sb}{it}}
    \newrobustcmd*\cdfontbi{\usefont{\encodingdefault}{\tud@cdfont@fam}{b}{it}}
    \newrobustcmd*\cdfontxi{\usefont{\encodingdefault}{\tud@cdfont@fam}{eb}{it}}
%    \end{macrocode}
% Sind die notwendigen Schriften nicht installiert, werden die Schalter für die
% Rückfallebene so definiert, dass diese eine Warnung ausgeben.
%    \begin{macrocode}
  \else%
    \tud@cdfont@fallback\cdfontln{OpenSansLight}{md}{up}
    \tud@cdfont@fallback\cdfontrn{OpenSansRegular}{md}{up}
    \tud@cdfont@fallback\cdfontsn{OpenSansSemiBold}{md}{up}
    \tud@cdfont@fallback\cdfontbn{OpenSansBold}{bf}{up}
    \tud@cdfont@fallback\cdfontxn{OpenSansExtraBold}{bf}{up}
    \tud@cdfont@fallback\cdfontli{OpenSansLightItalic}{md}{it}
    \tud@cdfont@fallback\cdfontri{OpenSansItalic}{md}{it}
    \tud@cdfont@fallback\cdfontsi{OpenSansSemiBoldItalic}{md}{it}
    \tud@cdfont@fallback\cdfontbi{OpenSansBoldItalic}{bf}{it}
    \tud@cdfont@fallback\cdfontxi{OpenSansExtraBoldItalic}{bf}{it}
  \fi%
%    \end{macrocode}
% Außerdem werden für die Schriften noch die entsprechenden Textauswahlbefehle
% bereitgestellt.
%    \begin{macrocode}
  \newcommand*\textcdln{}%
  \newcommand*\textcdrn{}%
  \newcommand*\textcdsn{}%
  \newcommand*\textcdbn{}%
  \newcommand*\textcdxn{}%
  \newcommand*\textcdli{}%
  \newcommand*\textcdri{}%
  \newcommand*\textcdsi{}%
  \newcommand*\textcdbi{}%
  \newcommand*\textcdxi{}%
  \DeclareTextFontCommand\textcdln{\cdfontln}%
  \DeclareTextFontCommand\textcdrn{\cdfontrn}%
  \DeclareTextFontCommand\textcdsn{\cdfontsn}%
  \DeclareTextFontCommand\textcdbn{\cdfontbn}%
  \DeclareTextFontCommand\textcdxn{\cdfontxn}%
  \DeclareTextFontCommand\textcdli{\cdfontli}%
  \DeclareTextFontCommand\textcdri{\cdfontri}%
  \DeclareTextFontCommand\textcdsi{\cdfontsi}%
  \DeclareTextFontCommand\textcdbi{\cdfontbi}%
  \DeclareTextFontCommand\textcdxi{\cdfontxi}%
}
%    \end{macrocode}
% \end{macro}^^A \textcdxi
% \end{macro}^^A \textcdbi
% \end{macro}^^A \textcdsi
% \end{macro}^^A \textcdri
% \end{macro}^^A \textcdli
% \end{macro}^^A \textcdxn
% \end{macro}^^A \textcdbn
% \end{macro}^^A \textcdsn
% \end{macro}^^A \textcdrn
% \end{macro}^^A \textcdln
% \end{macro}^^A \cdfontxi
% \end{macro}^^A \cdfontbi
% \end{macro}^^A \cdfontsi
% \end{macro}^^A \cdfontri
% \end{macro}^^A \cdfontli
% \end{macro}^^A \cdfontxn
% \end{macro}^^A \cdfontbn
% \end{macro}^^A \cdfontsn
% \end{macro}^^A \cdfontrn
% \end{macro}^^A \cdfontln
% \end{macro}^^A \tud@cdfont@init
% \begin{macro}{\univln}
% \begin{macro}{\univrn}
% \begin{macro}{\univbn}
% \begin{macro}{\univxn}
% \begin{macro}{\univls}
% \begin{macro}{\univrs}
% \begin{macro}{\univbs}
% \begin{macro}{\univxs}
% \begin{macro}{\dinbn}
% \begin{macro}{\tud@db}
% \begin{macro}{\textuln}
% \begin{macro}{\texturn}
% \begin{macro}{\textubn}
% \begin{macro}{\textuxn}
% \begin{macro}{\textuls}
% \begin{macro}{\texturs}
% \begin{macro}{\textubs}
% \begin{macro}{\textuxs}
% \begin{macro}{\textdbn}
% Die Definition der Befehle der veralteten Schriften des \CDs.
%    \begin{macrocode}
\if@tud@cdoldfont@active
  \renewcommand*\tud@cdfont@init{%
    \if@tud@cdfont@fam@exist%
      \if@tud@x@fontspec@enabled%
        \def\tud@res@a{%
          UprightFont={Univers},%
          FontFace={l}{n}{UniversLight},%
          FontFace={m}{n}{Univers},%
          FontFace={b}{n}{UniversBold},%
          FontFace={bx}{n}{UniversBold},%
          FontFace={eb}{n}{UniversBlack},%
          FontFace={l}{it}{UniversLightOblique},%
          FontFace={m}{it}{UniversOblique},%
          FontFace={b}{it}{UniversBoldOblique},%
          FontFace={bx}{it}{UniversBoldOblique},%
          FontFace={eb}{it}{UniversBlackOblique}%
        }%
        \tud@x@fontspec@init{\tud@cdfont@fam@osf}{\tud@res@a}%
        \tud@x@fontspec@init{\tud@cdfont@fam@lf}{\tud@res@a}%
      \fi%
%    \end{macrocode}
% Die Schalter zur expliziten Auswahl für die \Univers-Schriftschnitte.
%    \begin{macrocode}
      \newrobustcmd*\univln{\usefont{\encodingdefault}{\tud@cdfont@fam}{l}{n}}
      \newrobustcmd*\univrn{\usefont{\encodingdefault}{\tud@cdfont@fam}{m}{n}}
      \newrobustcmd*\univbn{\usefont{\encodingdefault}{\tud@cdfont@fam}{b}{n}}
      \newrobustcmd*\univxn{\usefont{\encodingdefault}{\tud@cdfont@fam}{eb}{n}}
      \newrobustcmd*\univls{\usefont{\encodingdefault}{\tud@cdfont@fam}{l}{sl}}
      \newrobustcmd*\univrs{\usefont{\encodingdefault}{\tud@cdfont@fam}{m}{sl}}
      \newrobustcmd*\univbs{\usefont{\encodingdefault}{\tud@cdfont@fam}{b}{sl}}
      \newrobustcmd*\univxs{\usefont{\encodingdefault}{\tud@cdfont@fam}{eb}{sl}}
%    \end{macrocode}
% Sind die notwendigen Schriften nicht installiert, werden die Schalter für die
% Rückfallebene so definiert, dass diese eine Warnung ausgeben.
%    \begin{macrocode}
    \else%
      \tud@cdfont@fallback\univln{UniversLight}{md}{up}
      \tud@cdfont@fallback\univrn{Univers}{md}{up}
      \tud@cdfont@fallback\univbn{UniversBold}{bf}{up}
      \tud@cdfont@fallback\univxn{UniversBlack}{bf}{up}
      \tud@cdfont@fallback\univls{UniversLightOblique}{md}{sl}
      \tud@cdfont@fallback\univrs{UniversOblique}{md}{sl}
      \tud@cdfont@fallback\univbs{UniversBoldOblique}{bf}{sl}
      \tud@cdfont@fallback\univxs{UniversBlackOblique}{bf}{sl}
    \fi%
%    \end{macrocode}
% Außerdem werden für die Schriften noch die entsprechenden Textauswahlbefehle
% bereitgestellt.
%    \begin{macrocode}
    \newcommand*\textuln{}%
    \newcommand*\texturn{}%
    \newcommand*\textubn{}%
    \newcommand*\textuxn{}%
    \newcommand*\textuls{}%
    \newcommand*\texturs{}%
    \newcommand*\textubs{}%
    \newcommand*\textuxs{}%
    \DeclareTextFontCommand\textuln{\univln}%
    \DeclareTextFontCommand\texturn{\univrn}%
    \DeclareTextFontCommand\textubn{\univbn}%
    \DeclareTextFontCommand\textuxn{\univxn}%
    \DeclareTextFontCommand\textuls{\univls}%
    \DeclareTextFontCommand\texturs{\univrs}%
    \DeclareTextFontCommand\textubs{\univbs}%
    \DeclareTextFontCommand\textuxs{\univxs}%
%    \end{macrocode}
% Nach der \Univers-Schriftfamilie wird für \DIN äquivalent verfahren.
%    \begin{macrocode}
    \if@tud@cdfont@db@exist%
      \if@tud@x@fontspec@enabled%
        \def\tud@res@a{%
          UprightFont={DIN-Bold},%
          FontFace={l}{n}{DIN-Bold},%
          FontFace={m}{n}{DIN-Bold},%
          FontFace={b}{n}{DIN-Bold},%
          FontFace={bx}{n}{DIN-Bold},%
          FontFace={eb}{n}{DIN-Bold},%
          FontFace={l}{it}{Font=DIN-Bold,FakeSlant=0.15},%
          FontFace={m}{it}{Font=DIN-Bold,FakeSlant=0.15},%
          FontFace={b}{it}{Font=DIN-Bold,FakeSlant=0.15},%
          FontFace={bx}{it}{Font=DIN-Bold,FakeSlant=0.15},%
          FontFace={eb}{it}{Font=DIN-Bold,FakeSlant=0.15}%
        }%
        \tud@x@fontspec@init{\tud@cdfont@db}{\tud@res@a}%
      \fi%
%    \end{macrocode}
% Bei der Schriftauswahl wird die entsprechende Mathematikschrift aktiviert.
%    \begin{macrocode}
      \newrobustcmd*\tud@db{%
        \usefont{\encodingdefault}{\tud@cdfont@db}{b}{n}%
        \mathversion{tud@dm}%
      }%
      \newrobustcmd*\dinbn{\tud@db}%
%    \end{macrocode}
% Da der Schalter \cs{dinbn} innerhalb der Überschriften verwendet wird, sollte 
% dieser zumindest auch auf den Überschriftenstil schalten, falls \DIN selbst 
% nicht installiert ist.
%    \begin{macrocode}
    \else%
      \newrobustcmd*\tud@db{\sectfont}%
      \newrobustcmd*\dinbn{\tud@cdfont@wrn{DIN-Bold}\tud@db}%
    \fi%
    \newcommand*\textdbn{}%
    \DeclareTextFontCommand\textdbn{\dinbn}%
  }%
\fi
%    \end{macrocode}
% \end{macro}^^A \textdbn
% \end{macro}^^A \textuxs
% \end{macro}^^A \textubs
% \end{macro}^^A \texturs
% \end{macro}^^A \textuls
% \end{macro}^^A \textuxn
% \end{macro}^^A \textubn
% \end{macro}^^A \texturn
% \end{macro}^^A \textuln
% \end{macro}^^A \tud@db
% \end{macro}^^A \dinbn
% \end{macro}^^A \univxs
% \end{macro}^^A \univbs
% \end{macro}^^A \univrs
% \end{macro}^^A \univls
% \end{macro}^^A \univxn
% \end{macro}^^A \univbn
% \end{macro}^^A \univrn
% \end{macro}^^A \univln
% \begin{macro}{\tud@cdfont@fallback}
% \changes{v2.05}{2015/12/14}{neu}^^A
% \begin{macro}{\tud@cdfont@wrn}
% \changes{v2.05}{2015/12/14}{neu}^^A
% Hier wird die Rückfallebene für die Schriftschalter sowie die dazugehörige 
% Warnung definiert.
%    \begin{macrocode}
\newcommand*\tud@cdfont@fallback[4]{%
  \newrobustcmd*#1{%
    \tud@cdfont@wrn{#2}%
    \sffamily\csuse{#3series}\csuse{#4shape}%
  }%
}
\newcommand*\tud@cdfont@wrn[1]{%
%<*class>
  \ClassWarning{\TUD@Class@Name}%
%</class>
%<*package>
  \PackageWarning{tudscrfonts}%
%</package>
  {%
    The requested font `#1'\MessageBreak%
    is not available, a fallback layout is used%
  }%
}
%    \end{macrocode}
% \end{macro}^^A \tud@cdfont@wrn
% \end{macro}^^A \tud@cdfont@fallback
% \begin{macro}{\cdfont}
% \changes{v2.04}{2015/03/04}{neu}^^A
% \begin{macro}{\tudfont}
% \changes{v2.04}{2015/04/24}{\cls{tudmathposter} wird unterstützt}^^A
% \begin{macro}{\textcdfont}
% \changes{v2.04}{2015/03/04}{neu}^^A
% \begin{macro}{\tud@cdfont@get}
% \changes{v2.04}{2015/03/04}{neu}^^A
% \changes{v2.06}{2018/06/29}{für \OpenSans überarbeitet}^^A
% \begin{macro}{\tud@cdfont@trim}
% \changes{v2.04}{2015/03/04}{neu}^^A
% \changes{v2.06}{2018/06/29}{für \OpenSans überarbeitet}^^A
% \begin{macro}{\tud@cdfont@@trim}
% \changes{v2.04}{2015/03/04}{neu}^^A
% Die Klassen \cls{tudposter} und \cls{tudmathposter} definieren den Befehl 
% \cs{tudfont}, mit welchem die Schriften des \CDs der \TnUD auch mit Klarnamen
% ausgewählt werden können. Zur Kompatibilität wird der Befehl \cs{cdfont} 
% definiert, welcher die gleiche Funktionalität bereitstellt. Zusätzlich wird 
% der Befehl \cs{textcdfont} bereitgestellt, der die Auswahl der Schrift als 
% Textbefehl ermöglicht.
%    \begin{macrocode}
\providecommand*\tudfont[1]{\cdfont{#1}}
\newcommand*\cdfont[1]{%
  \tud@cdfont@get{#1}%
  \ifx\@tempc\relax\else%
    \csuse{os\@tempc}%
  \fi%
}
\newcommand*\textcdfont[2]{%
  \tud@cdfont@get{#1}%
  \ifx\@tempc\relax%
    \textsf{#2}%
  \else%
    \csuse{textcd\@tempc}{#2}%
  \fi%
}
\if@tud@cdoldfont@active
  \renewcommand*\cdfont[1]{%
    \tud@cdfont@get{#1}%
    \ifx\@tempc\relax\else%
      \ifstr{\@tempc}{dinbn}{\dinbn}{\csuse{univ\@tempc}}%
    \fi%
  }%
  \renewcommand*\textcdfont[2]{%
    \tud@cdfont@get{#1}%
    \ifx\@tempc\relax%
      \textsf{#2}%
    \else%
      \ifstr{\@tempc}{dinbn}{\textdbn{#2}}{\csuse{textu\@tempc}{#2}}%
    \fi%
  }%
\fi
\newcommand*\tud@cdfont@get[1]{%
  \tud@cdfont@trim\@tempa{#1}%
  \ifstr{\@tempa}{l}{\def\@tempc{ln}}{%
  \ifstr{\@tempa}{}{\def\@tempc{rn}}{%
  \ifstr{\@tempa}{r}{\def\@tempc{rn}}{%
  \ifstr{\@tempa}{s}{\def\@tempc{sn}}{%
  \ifstr{\@tempa}{b}{\def\@tempc{bn}}{%
  \ifstr{\@tempa}{x}{\def\@tempc{xn}}{%
  \ifstr{\@tempa}{li}{\def\@tempc{li}}{%
  \ifstr{\@tempa}{il}{\def\@tempc{li}}{%
  \ifstr{\@tempa}{i}{\def\@tempc{ri}}{%
  \ifstr{\@tempa}{ri}{\def\@tempc{ri}}{%
  \ifstr{\@tempa}{ir}{\def\@tempc{ri}}{%
  \ifstr{\@tempa}{si}{\def\@tempc{si}}{%
  \ifstr{\@tempa}{is}{\def\@tempc{si}}{%
  \ifstr{\@tempa}{bi}{\def\@tempc{bi}}{%
  \ifstr{\@tempa}{ib}{\def\@tempc{bi}}{%
  \ifstr{\@tempa}{xi}{\def\@tempc{xi}}{%
  \ifstr{\@tempa}{ix}{\def\@tempc{xi}}{%
    \let\@tempc\relax%
%<*class>
    \ClassWarning{\TUD@Class@Name}%
%</class>
%<*package>
    \PackageWarning{tudscrfonts}%
%</package>
      {The font `#1' is unknown}%
  }}}}}}}}}}}}}}}}}%
}
\if@tud@cdoldfont@active
  \renewcommand*\tud@cdfont@get[1]{%
    \tud@cdfont@trim\@tempa{#1}%
    \ifstr{\@tempa}{45}{\def\@tempc{ln}}{%
    \ifstr{\@tempa}{55}{\def\@tempc{rn}}{%
    \ifstr{\@tempa}{65}{\def\@tempc{bn}}{%
    \ifstr{\@tempa}{75}{\def\@tempc{xn}}{%
    \ifstr{\@tempa}{45oblique}{\def\@tempc{ls}}{%
    \ifstr{\@tempa}{55oblique}{\def\@tempc{rs}}{%
    \ifstr{\@tempa}{65oblique}{\def\@tempc{bs}}{%
    \ifstr{\@tempa}{75oblique}{\def\@tempc{xs}}{%
    \ifstr{\@tempa}{oblique45}{\def\@tempc{ls}}{%
    \ifstr{\@tempa}{oblique55}{\def\@tempc{rs}}{%
    \ifstr{\@tempa}{oblique65}{\def\@tempc{bs}}{%
    \ifstr{\@tempa}{oblique75}{\def\@tempc{xs}}{%
    \ifstr{\@tempa}{din}{\def\@tempc{dinbn}}{%
    \ifstr{\@tempa}{din65}{\def\@tempc{dinbn}}{%
      \let\@tempc\relax%
%<*class>
      \ClassWarning{\TUD@Class@Name}%
%</class>
%<*package>
      \PackageWarning{tudscrfonts}%
%</package>
      {The font `#1' is unknown}%
    }}}}}}}}}}}}}}%
  }%
\fi
\newcommand*\tud@cdfont@trim[2]{%
  \protected@edef\@tempa{\trim@spaces{#2}}%
  \tud@lowerstring{\@tempa}{\@tempa}%
  \tud@cdfont@@trim{ }{}%
  \tud@cdfont@@trim{-}{}%
  \tud@cdfont@@trim{_}{}%
  \tud@cdfont@@trim{~}{}%
  \tud@cdfont@@trim{opensans}{}%
  \tud@cdfont@@trim{light}{l}%
  \tud@cdfont@@trim{regular}{r}%
  \tud@cdfont@@trim{book}{r}%
  \tud@cdfont@@trim{semibold}{s}%
  \tud@cdfont@@trim{demi}{s}%
  \tud@cdfont@@trim{bold}{b}%
  \tud@cdfont@@trim{extrabold}{x}%
  \tud@cdfont@@trim{black}{x}%
  \tud@cdfont@@trim{italic}{i}%
  \tud@cdfont@@trim{oblique}{i}%
  \tud@cdfont@@trim{slanted}{i}%
  \let#1\@tempa%
}
\if@tud@cdoldfont@active
  \renewcommand*\tud@cdfont@trim[2]{%
    \protected@edef\@tempa{\trim@spaces{#2}}%
    \tud@lowerstring{\@tempa}{\@tempa}%
    \tud@cdfont@@trim{ }{}%
    \tud@cdfont@@trim{-}{}%
    \tud@cdfont@@trim{_}{}%
    \tud@cdfont@@trim{~}{}%
    \tud@cdfont@@trim{ce}{}%
    \tud@cdfont@@trim{ff}{}%
    \tud@cdfont@@trim{univers}{}%
    \tud@cdfont@@trim{60}{65}%
    \tud@cdfont@@trim{light}{45}%
    \tud@cdfont@@trim{regular}{55}%
    \tud@cdfont@@trim{bold}{65}%
    \tud@cdfont@@trim{black}{75}%
    \tud@cdfont@@trim{4545}{45}%
    \tud@cdfont@@trim{5555}{55}%
    \tud@cdfont@@trim{6565}{65}%
    \tud@cdfont@@trim{7575}{75}%
    \tud@cdfont@@trim{italic}{oblique}%
    \tud@cdfont@@trim{slanted}{oblique}%
    \let#1\@tempa%
  }%
\fi
\newcommand*\tud@cdfont@@trim[2]{%
  \def\@tempb##1#1##2\relax{\def\@tempa{##1#2##2}}%
  \@expandtwoargs\in@{#1}{\@tempa}%
  \@whilesw\ifin@\fi{%
    \expandafter\@tempb\@tempa\relax%
    \@expandtwoargs\in@{#1}{\@tempa}%
  }%
}
%    \end{macrocode}
% \end{macro}^^A \tud@cdfont@@trim
% \end{macro}^^A \tud@cdfont@trim
% \end{macro}^^A \tud@cdfont@get
% \end{macro}^^A \textcdfont
% \end{macro}^^A \tudfont
% \end{macro}^^A \cdfont
%
%
% \iffalse
%</body>
%<*option>
% \fi
%
% \subsubsection{TypeWriter-Schriften}
%
% \begin{option}{ttfont}
% \changes{v2.06}{2018/07/12}{neu}^^A
% \begin{macro}{\tud@ttfont@num}
% \changes{v2.06}{2018/07/12}{neu}^^A
% Mit dieser Option regelt, ob die TypeWriter-Schriften durch den Anwender
% manuell eingestellt werden oder aber die entsprechenden Schriften aus den 
% Paketen \pkg{lmodern} oder \pkg{roboto-mono} zum Einsatz kommt.
%    \begin{macrocode}
\newcommand*\tud@ttfont@num{0}
\TUD@key{ttfont}[true]{%
  \TUD@set@numkey{ttfont}{@tempa}{%
    \TUD@bool@numkey,%
    {lm}{1},{lmtt}{1},{lmodern}{1},%
    {roboto}{2},{roboto-mono}{2},{robo}{2}%
  }{#1}%
  \ifx\FamilyKeyState\FamilyKeyStateProcessed%
    \ifcase\@tempa\relax% false
      \renewcommand*\tud@ttfont@num{0}%
    \or% lmodern
      \renewcommand*\tud@ttfont@num{1}%
    \or% roboto
      \renewcommand*\tud@ttfont@num{2}%
    \fi%
    \TUD@SpecialOptionAtDocument{tud@font@set}%
  \fi%
}
%    \end{macrocode}
% \end{macro}^^A \tud@ttfont@num
% \end{option}^^A ttfont
%
% \iffalse
%</option>
%<*body>
% \fi
%
% \begin{macro}{\tud@ttfont@set}
% \changes{v2.06}{2018/07/12}{neu}^^A
% \begin{macro}{\tud@ttfont@@set}
% \changes{v2.06}{2018/07/12}{neu}^^A
% \begin{macro}{\tud@ttfont@md}
% \changes{v2.06}{2018/07/12}{neu}^^A
% \begin{macro}{\tud@ttfont@bf}
% \changes{v2.06}{2018/07/12}{neu}^^A
% Die Auswahl der TypeWriter-Schrift.
%    \begin{macrocode}
\newcommand*\tud@ttfont@set{%
  \if@tud@ttfont@rb@exist\else\ifnum\tud@ttfont@num=\tw@\relax%
%<*class>
    \ClassWarning{\TUD@Class@Name}%
%</class>
%<*package>
    \PackageWarning{tudscrfonts}%
%</package>
    {%
      The package `roboto-mono' is missing so the\MessageBreak%
      typewriter font of package `lmodern' is used%
    }%
    \renewcommand*\tud@ttfont@num{1}%
  \fi\fi%
  \ifcase\tud@ttfont@num\relax% false
    \let\tud@ttfont@fam\ttdefault%
  \or% lmodern
    \let\tud@ttfont@fam\tud@ttfont@lm%
  \or% roboto
    \let\tud@ttfont@fam\tud@ttfont@rb%
  \fi%
  \ifcase\tud@ttfont@num\relax\else% lmodern
    \if@tud@x@fontspec@enabled%
      \ifx\ttdefault\tud@ttfont@fam\else%
        \setmonofont{\tud@ttfont@fam}%
      \fi%
    \else%
      \renewcommand*\ttdefault{\tud@ttfont@fam}%
    \fi%
  \fi%
  \tud@ttfont@@set%
}
\newcommand*\tud@ttfont@md{}
\newcommand*\tud@ttfont@bf{}
\newcommand*\tud@ttfont@@set{%
  \ifcase\tud@ttfont@num\relax%
    \ifundef{\mdseries@tt}{%
      \edef\tud@ttfont@md{\tud@cs@use{mddefault}}%
    }{%
      \edef\tud@ttfont@md{\mdseries@tt}%
    }%
    \ifundef{\bfseries@tt}{%
      \edef\tud@ttfont@bf{\tud@cs@use{bfdefault}}%
    }{%
      \edef\tud@ttfont@bf{\bfseries@tt}%
    }%
  \or% lmodern
    \if@tud@cdfont@heavy%
      \renewcommand*\tud@ttfont@md{m}%
      \renewcommand*\tud@ttfont@bf{b}%
    \else%
      \renewcommand*\tud@ttfont@md{l}%
      \if@tud@cdfont@ultrabold%
        \renewcommand*\tud@ttfont@bf{b}%
      \else%
        \renewcommand*\tud@ttfont@bf{m}%
      \fi%
    \fi%
  \else%
    \renewcommand*\tud@ttfont@md{\tud@cdfont@md}%
    \renewcommand*\tud@ttfont@bf{\tud@cdfont@bf}%
  \fi%
}
%    \end{macrocode}
% \end{macro}^^A \tud@ttfont@bf
% \end{macro}^^A \tud@ttfont@md
% \end{macro}^^A \tud@ttfont@@set
% \end{macro}^^A \tud@ttfont@set
% \begin{macro}{\tud@ttfont@init}
% \changes{v2.06}{2018/07/12}{neu}^^A
% Mit \cs{tud@ttfont@init} erfolgt die Defintion der TypeWriter-Schriften.
%    \begin{macrocode}
\newcommand*\tud@ttfont@init{%
  \if@tud@x@fontspec@enabled%
    \if@tud@ttfont@rb@exist%
      \tud@x@fontspec@init{\tud@ttfont@rb}{%
        Extension={.ttf},%
        UprightFont={RobotoMono-Regular},%
        FontFace={l}{n}{RobotoMono-Light},%
        FontFace={m}{n}{RobotoMono-Regular},%
        FontFace={sb}{n}{RobotoMono-Medium},%
        FontFace={b}{n}{RobotoMono-Bold},%
        FontFace={bx}{n}{RobotoMono-Bold},%
        FontFace={eb}{n}{RobotoMono-Bold},%
        FontFace={l}{it}{RobotoMono-LightItalic},%
        FontFace={m}{it}{RobotoMono-Italic},%
        FontFace={sb}{it}{RobotoMono-MediumItalic},%
        FontFace={b}{it}{RobotoMono-BoldItalic},%
        FontFace={bx}{it}{RobotoMono-BoldItalic},%
        FontFace={eb}{it}{RobotoMono-BoldItalic}%
      }%
    \fi%
%    \end{macrocode}
% Gewöhnlich ist die Schrift \texttt{lmtt} via \pkg{fontspec} voreingestellt. 
% Allerdings wird diese von einigen Paketen mit \cs{setmonofont} überschrieben. 
% Für diesen Fall wird die Schriftfamilie \texttt{lmtt} mit den Mitteln von 
% \pkg{fontspec} definiert.
%    \begin{macrocode}
    \ifx\ttdefault\tud@ttfont@lm\else%
      \tud@x@fontspec@init{\tud@ttfont@lm}{%
        UprightFont={LMMono10-Regular},%
        FontFace={l}{n}{LMMonoLt10-Regular},%
        FontFace={m}{n}{LMMono10-Regular},%
        FontFace={sb}{n}{LMMono10-Regular},%
        FontFace={b}{n}{LMMonoLt10-Bold},%
        FontFace={bx}{n}{LMMonoLt10-Bold},%
        FontFace={eb}{n}{LMMonoLt10-Bold},%
        FontFace={l}{it}{LMMonoLt10-Oblique},%
        FontFace={m}{it}{LMMonoSlant10-Regular},%
        FontFace={sb}{it}{LMMonoSlant10-Regular},%
        FontFace={b}{it}{LMMonoLt10-BoldOblique},%
        FontFace={bx}{it}{LMMonoLt10-BoldOblique},%
        FontFace={eb}{it}{LMMonoLt10-BoldOblique}%
      }%
    \fi%
  \fi%
}
%    \end{macrocode}
% \end{macro}^^A \tud@ttfont@init
%
% \iffalse
%</body>
%<*option>
% \fi
%
% \subsubsection{Schriften für den Mathematiksatz}
%
% Für den mathematischen Satz werden ergänzend zur \OpenSans die Symbole aus 
% dem Paket \pkg{mdsymbol} genutzt. Diese werden bereits bei der Erstellung der 
% Type1-Schriften eingefügt.
%
% Die veralteten \Univers-Schriften des \CDs beeinhalten keine Glyphen für den 
% Mathematiksatz. Daher musste~-- insbesondere für griechische Symbole~-- eine
% passende Alternative gefunden werden. Die Auswahl an Mathematikschriften für
% \LaTeX{} ist relativ gering. Dennoch schien das \pkg{cmbright}-Paket dafür
% geeignet, weshalb die darin enthaltenen Glyphen direkt bei der Installation 
% der Schriften eingebunden werden.
%
% \begin{option}{cdmath}
% \begin{macro}{\if@tud@cdmath@active}
% \begin{macro}{\if@tud@cdmath@active@locked}
% Mit dieser Option kann die genutzte Standardschrift für den Mathematiksatz
% für das gesamte Dokument umgestellt werden.
%    \begin{macrocode}
\tud@locked@newbool{@tud@cdmath@active}
\tud@locked@bool@preset{@tud@cdmath@active}{true}%
\TUD@key{cdmath}[true]{%
  \TUD@set@numkey{cdmath}{@tempa}{%
    \TUD@bool@numkey,%
    {serifmath}{0},{serif}{0},{nosansmath}{0},{nosans}{0},%
    {sansmath}{1},{sans}{1},%
    {upgreek}{2},{uprightgreek}{2},{uprightgreeks}{2},%
    {slgreek}{3},{slantedgreek}{3},{slantedgreeks}{3},%
    {itgreek}{3},{italicgreek}{3},{italicgreeks}{3},%
    {texgreek}{4},{latexgreek}{4},{texgreeks}{4},{latexgreeks}{4},%
    {standardgreek}{4},{standardgreeks}{4}%
  }{#1}%
  \ifx\FamilyKeyState\FamilyKeyStateProcessed%
    \ifcase\@tempa\relax% false
      \tud@locked@bool@set{@tud@cdmath@active}{false}%
      \TUD@SpecialOptionAtDocument{tud@math@set}%
    \or% true
      \tud@locked@bool@set{@tud@cdmath@active}{true}%
      \TUD@SpecialOptionAtDocument{tud@math@set}%
    \or% uprightgreek
      \TUDoptions{slantedgreek=false}%
    \or% slantedgreek
      \TUDoptions{slantedgreek=true}%
    \or% standardgreek
      \TUDoptions{slantedgreek=standard}%
    \fi%
  \fi%
}
%    \end{macrocode}
% \end{macro}^^A \if@tud@cdmath@active@locked
% \end{macro}^^A \if@tud@cdmath@active
% \end{option}^^A cdmath
% \begin{option}{slantedgreek}
% \begin{option}{slantedGreek}
% \begin{option}{slantedGreeks}
% \changes{v2.04}{2015/02/27}{neu}^^A
% \begin{macro}{\if@tud@slantedgreek}
% Mit dieser Option kann die Neigung der griechischen Majuskeln geändert werden.
%    \begin{macrocode}
\newcommand*\tud@slantedgreek@num{0}
\TUD@key{slantedgreek}[true]{%
  \TUD@set@numkey{slantedgreek}{@tempa}{%
    \TUD@bool@numkey,%
    {up}{0},{upright}{0},%
    {sl}{1},{slanted}{1},{it}{1},{italic}{1},%
    {standard}{2},{tex}{2},{latex}{2},{standardgreek}{2},{standardgreeks}{2},%
    {texgreek}{2},{latexgreek}{2},{texgreeks}{2},{latexgreeks}{2}%
  }{#1}%
  \ifx\FamilyKeyState\FamilyKeyStateProcessed%
    \ifcase\@tempa\relax% upright
      \renewcommand*\tud@slantedgreek@num{0}%
    \or% slanted
      \renewcommand*\tud@slantedgreek@num{1}%
    \or% standard
      \renewcommand*\tud@slantedgreek@num{2}%
    \fi%
    \TUD@SpecialOptionAtDocument{tud@math@set}%
  \fi%
}
\TUD@key{slantedGreek}[true]{\TUDoptions{slantedgreek=#1}}
\TUD@key{slantedGreeks}[true]{\TUDoptions{slantedgreek=#1}}
%    \end{macrocode}
% \end{macro}^^A \if@tud@slantedgreek
% \end{option}^^A slantedGreeks
% \end{option}^^A slantedGreek
% \end{option}^^A slantedgreek
%
% \iffalse
%</option>
%<*body>
% \fi
%
% \begin{macro}{\tud@math@init}
% \changes{v2.04}{2015/03/10}{neu}^^A
%    \begin{macrocode}
\newcommand*\tud@math@init{%
%    \end{macrocode}
% Wird das Paket \pkg{unicode-math} verwendet, werden keine Einstellungen für 
% die Mathematikschriften vorgenommen\dots
%    \begin{macrocode}
  \@ifpackageloaded{unicode-math}{%
    \if@tud@cdmath@active%
%<*class>
      \ClassWarningNoLine{\TUD@Class@Name}%
%</class>
%<*package>
      \PackageWarningNoLine{tudscrfonts}%
%</package>
      {%
        Because package `unicode-math' was loaded, setting up\MessageBreak%
        corporate design fonts in math mode is deactivated%
      }%
    \fi%
    \tud@locked@bool@set{@tud@cdmath@active}{false}%
    \let\tud@math@set\relax%
  }{%
%    \end{macrocode}
% Die aufrechten griechischen Minuskeln sind in den Type1-Schriften von \LaTeXe 
% leider nicht enthalten. Deshalb werden behelfsmäßig die kursiven Minuskeln 
% genutzt, wenn die \OpenSans nicht für den Mathematikmodus verwendet wird.
%    \begin{macrocode}
    \ifundef{\symupletters}{%
      \DeclareSymbolFont{upletters}{OML}{cmm}{m}{it}%
      \SetSymbolFont{upletters}{bold}{OML}{cmm}{b}{it}%
    }{}%
%    \end{macrocode}
% Die Symbole für aufrechte und kursive griechische Glyphen werden definiert.
%    \begin{macrocode}
  \@tempcnta=\z@\relax%
  \tud@math@greek@loop@uc\tud@math@greek@declare@uc%
  \tud@math@greek@loop@lc\tud@math@greek@declare@lc%
%    \end{macrocode}
% Für die griechischen Buchstaben wird mit \cs{tud@math@greek@check@slanted}
% geprüft, ob durch ein Paket diese auf die kursive Variante gesetzt wurden.
% Ist dies der Fall, wird der Standardwert für die Option \opt{slantedgreek}
% auf \val{true} gesetzt. Anschließend erfolgt die \emph{Intialisierung} der
% griechischen Großbuchstaben mit dem Befehl \cs{tud@math@greek@init}.
%    \begin{macrocode}
    \tud@math@greek@check@slanted%
    \tud@math@greek@loop\tud@math@greek@init%
%    \end{macrocode}
% Wurde angegeben, dass kursive griechische Majuskeln genutzt werden sollen, so 
% werden diese initialisiert. Dies ist insbesondere notwendig, wenn die
% Standardschriften für den Mathematiksatz genutzt werden.
%    \begin{macrocode}
    \ifnum\tud@slantedgreek@num=\@ne\relax%
      \tud@math@greek@loop@uc\tud@math@greek@slanted%
    \fi%
%    \end{macrocode}
% Hier werden die Schriften des \CDs für den Mathematikmodus initialisiert.
%    \begin{macrocode}
    \tud@cdmath@init%
%    \end{macrocode}
% Zum Schluss werden abhängig von der gewählten Einstellung für die Schriftart 
% der Option \opt{cdfont} die benötigten Mathematikversionen geladen.
%    \begin{macrocode}
    \tud@math@set%
  }%
}
%    \end{macrocode}
% \end{macro}^^A \tud@math@init
% \begin{macro}{\tud@math@greek@loop}
% \changes{v2.04}{2015/03/10}{neu}^^A
% \begin{macro}{\tud@math@greek@loop@uc}
% \changes{v2.06}{2018/07/17}{neu}^^A
% \begin{macro}{\tud@math@greek@loop@lc}
% \changes{v2.06}{2018/07/17}{neu}^^A
% Um Änderungen an den griechischen Glyphen vornehmen zu können, wird das Makro
% \cs{tud@math@greek@loop} definert, mit dem eine Namensliste der Buchstaben
% durchlaufen wird. Dabei wird in Majuskeln und Minuskeln unterschieden.
%    \begin{macrocode}
\newcommand*\tud@math@greek@loop[1]{%
  \tud@math@greek@loop@uc{#1}%
  \tud@math@greek@loop@lc{#1}%
}
\newcommand*\tud@math@greek@loop@uc[1]{%
  \@for\@tempa:={%
    Gamma,Delta,Theta,Lambda,Xi,Pi,Sigma,Upsilon,Phi,Psi,Omega%
  }\do{#1{\@tempa}}%
}
\newcommand*\tud@math@greek@loop@lc[1]{%
  \@for\@tempa:={%
    alpha,beta,gamma,delta,epsilon,zeta,eta,theta,iota,kappa,lambda,%
    mu,nu,xi,pi,rho,sigma,tau,upsilon,phi,chi,psi,omega,%
    varepsilon,vartheta,varpi,varrho,varsigma,varphi%
  }\do{#1{\@tempa}}%
}
%    \end{macrocode}
% \end{macro}^^A \tud@math@greek@loop@lc
% \end{macro}^^A \tud@math@greek@loop@uc
% \end{macro}^^A \tud@math@greek@loop
% \begin{macro}{\tud@math@greek@declare@uc}
% \changes{v2.04}{2015/03/10}{neu}^^A
% \begin{macro}{\tud@math@greek@upGamma}
% \begin{macro}{\tud@math@greek@upDelta}
% \begin{macro}{\tud@math@greek@upTheta}
% \begin{macro}{\tud@math@greek@upLambda}
% \begin{macro}{\tud@math@greek@upXi}
% \begin{macro}{\tud@math@greek@upPi}
% \begin{macro}{\tud@math@greek@upSigma}
% \begin{macro}{\tud@math@greek@upUpsilon}
% \begin{macro}{\tud@math@greek@upPhi}
% \begin{macro}{\tud@math@greek@upPsi}
% \begin{macro}{\tud@math@greek@upOmega}
% \begin{macro}{\tud@math@greek@itGamma}
% \begin{macro}{\tud@math@greek@itDelta}
% \begin{macro}{\tud@math@greek@itTheta}
% \begin{macro}{\tud@math@greek@itLambda}
% \begin{macro}{\tud@math@greek@itXi}
% \begin{macro}{\tud@math@greek@itPi}
% \begin{macro}{\tud@math@greek@itSigma}
% \begin{macro}{\tud@math@greek@itUpsilon}
% \begin{macro}{\tud@math@greek@itPhi}
% \begin{macro}{\tud@math@greek@itPsi}
% \begin{macro}{\tud@math@greek@itOmega}
% In Anlehnung an verschiedene Pakete für Mathematikschriften werden Symbole 
% für aufrechte und kursive Majuskeln der griechischen Lettern definiert. Damit 
% auch bei diesen Buchstaben ein Umschalten für die unterschiedlichen Schriften 
% möglich ist, wird eine interne Version definiert und im Bedarfsfall der
% dazugehörige \LaTeX-Befehl mit dieser überschrieben. Hierfür sind die beiden 
% Befehle \cs{tud@cdmath@greek@shape@set} und \cs{tud@cdmath@greek@shape@reset}
% verantwortlich.
%    \begin{macrocode}
\newcommand*\tud@math@greek@declare@uc[1]{%
  \edef\@tempb{%
    \noexpand\DeclareMathSymbol{\csname tud@math@greek@up#1\endcsname}%
      {\noexpand\mathalpha}{operators}{\the\@tempcnta}%
  }\@tempb%
  \edef\@tempb{%
    \noexpand\DeclareMathSymbol{\csname tud@math@greek@it#1\endcsname}%
      {\noexpand\mathalpha}{letters}{\the\@tempcnta}%
  }\@tempb%
  \advance \@tempcnta by 1%
}
%    \end{macrocode}
% \end{macro}^^A \tud@math@greek@itOmega
% \end{macro}^^A \tud@math@greek@itPsi
% \end{macro}^^A \tud@math@greek@itPhi
% \end{macro}^^A \tud@math@greek@itUpsilon
% \end{macro}^^A \tud@math@greek@itSigma
% \end{macro}^^A \tud@math@greek@itPi
% \end{macro}^^A \tud@math@greek@itXi
% \end{macro}^^A \tud@math@greek@itLambda
% \end{macro}^^A \tud@math@greek@itTheta
% \end{macro}^^A \tud@math@greek@itDelta
% \end{macro}^^A \tud@math@greek@itGamma
% \end{macro}^^A \tud@math@greek@upOmega
% \end{macro}^^A \tud@math@greek@upPsi
% \end{macro}^^A \tud@math@greek@upPhi
% \end{macro}^^A \tud@math@greek@upUpsilon
% \end{macro}^^A \tud@math@greek@upSigma
% \end{macro}^^A \tud@math@greek@upPi
% \end{macro}^^A \tud@math@greek@upXi
% \end{macro}^^A \tud@math@greek@upLambda
% \end{macro}^^A \tud@math@greek@upTheta
% \end{macro}^^A \tud@math@greek@upDelta
% \end{macro}^^A \tud@math@greek@upGamma
% \end{macro}^^A \tud@math@greek@declare@uc
% \begin{macro}{\tud@math@greek@declare@lc}
% \changes{v2.06}{2018/07/17}{neu}^^A
% \begin{macro}{\tud@math@greek@upalpha}
% \begin{macro}{\tud@math@greek@upbeta}
% \begin{macro}{\tud@math@greek@upgamma}
% \begin{macro}{\tud@math@greek@updelta}
% \begin{macro}{\tud@math@greek@upepsilon}
% \begin{macro}{\tud@math@greek@upzeta}
% \begin{macro}{\tud@math@greek@upeta}
% \begin{macro}{\tud@math@greek@uptheta}
% \begin{macro}{\tud@math@greek@upiota}
% \begin{macro}{\tud@math@greek@upkappa}
% \begin{macro}{\tud@math@greek@uplambda}
% \begin{macro}{\tud@math@greek@upmu}
% \begin{macro}{\tud@math@greek@upnu}
% \begin{macro}{\tud@math@greek@upxi}
% \begin{macro}{\tud@math@greek@uppi}
% \begin{macro}{\tud@math@greek@uprho}
% \begin{macro}{\tud@math@greek@upsigma}
% \begin{macro}{\tud@math@greek@uptau}
% \begin{macro}{\tud@math@greek@upupsilon}
% \begin{macro}{\tud@math@greek@upphi}
% \begin{macro}{\tud@math@greek@upchi}
% \begin{macro}{\tud@math@greek@uppsi}
% \begin{macro}{\tud@math@greek@upomega}
% \begin{macro}{\tud@math@greek@upvarepsilon}
% \begin{macro}{\tud@math@greek@upvartheta}
% \begin{macro}{\tud@math@greek@upvarpi}
% \begin{macro}{\tud@math@greek@upvarrho}
% \begin{macro}{\tud@math@greek@upvarsigma}
% \begin{macro}{\tud@math@greek@upvarphi}
% \begin{macro}{\tud@math@greek@italpha}
% \begin{macro}{\tud@math@greek@itbeta}
% \begin{macro}{\tud@math@greek@itgamma}
% \begin{macro}{\tud@math@greek@itdelta}
% \begin{macro}{\tud@math@greek@itepsilon}
% \begin{macro}{\tud@math@greek@itzeta}
% \begin{macro}{\tud@math@greek@iteta}
% \begin{macro}{\tud@math@greek@ittheta}
% \begin{macro}{\tud@math@greek@itiota}
% \begin{macro}{\tud@math@greek@itkappa}
% \begin{macro}{\tud@math@greek@itlambda}
% \begin{macro}{\tud@math@greek@itmu}
% \begin{macro}{\tud@math@greek@itnu}
% \begin{macro}{\tud@math@greek@itxi}
% \begin{macro}{\tud@math@greek@itpi}
% \begin{macro}{\tud@math@greek@itrho}
% \begin{macro}{\tud@math@greek@itsigma}
% \begin{macro}{\tud@math@greek@ittau}
% \begin{macro}{\tud@math@greek@itupsilon}
% \begin{macro}{\tud@math@greek@itphi}
% \begin{macro}{\tud@math@greek@itchi}
% \begin{macro}{\tud@math@greek@itpsi}
% \begin{macro}{\tud@math@greek@itomega}
% \begin{macro}{\tud@math@greek@itvarepsilon}
% \begin{macro}{\tud@math@greek@itvartheta}
% \begin{macro}{\tud@math@greek@itvarpi}
% \begin{macro}{\tud@math@greek@itvarrho}
% \begin{macro}{\tud@math@greek@itvarsigma}
% \begin{macro}{\tud@math@greek@itvarphi}
% Mit der Version~v2.06 und der eingeführten \OpenSans ist auch für griechische 
% Minuskeln eine aufrechte Variante verfügbar.
%    \begin{macrocode}
\newcommand*\tud@math@greek@declare@lc[1]{%
  \edef\@tempb{%
    \noexpand\DeclareMathSymbol{\csname tud@math@greek@up#1\endcsname}%
      {\noexpand\mathalpha}{upletters}{\the\@tempcnta}%
  }\@tempb%
  \edef\@tempb{%
    \noexpand\DeclareMathSymbol{\csname tud@math@greek@it#1\endcsname}%
      {\noexpand\mathalpha}{letters}{\the\@tempcnta}%
  }\@tempb%
  \advance \@tempcnta by 1%
}
%    \end{macrocode}
% \end{macro}^^A \tud@math@greek@itvarphi
% \end{macro}^^A \tud@math@greek@itvarsigma
% \end{macro}^^A \tud@math@greek@itvarrho
% \end{macro}^^A \tud@math@greek@itvarpi
% \end{macro}^^A \tud@math@greek@itvartheta
% \end{macro}^^A \tud@math@greek@itvarepsilon
% \end{macro}^^A \tud@math@greek@itomega
% \end{macro}^^A \tud@math@greek@itpsi
% \end{macro}^^A \tud@math@greek@itchi
% \end{macro}^^A \tud@math@greek@itphi
% \end{macro}^^A \tud@math@greek@itupsilon
% \end{macro}^^A \tud@math@greek@ittau
% \end{macro}^^A \tud@math@greek@itsigma
% \end{macro}^^A \tud@math@greek@itrho
% \end{macro}^^A \tud@math@greek@itpi
% \end{macro}^^A \tud@math@greek@itxi
% \end{macro}^^A \tud@math@greek@itnu
% \end{macro}^^A \tud@math@greek@itmu
% \end{macro}^^A \tud@math@greek@itlambda
% \end{macro}^^A \tud@math@greek@itkappa
% \end{macro}^^A \tud@math@greek@itiota
% \end{macro}^^A \tud@math@greek@ittheta
% \end{macro}^^A \tud@math@greek@iteta
% \end{macro}^^A \tud@math@greek@itzeta
% \end{macro}^^A \tud@math@greek@itepsilon
% \end{macro}^^A \tud@math@greek@itdelta
% \end{macro}^^A \tud@math@greek@itgamma
% \end{macro}^^A \tud@math@greek@itbeta
% \end{macro}^^A \tud@math@greek@italpha
% \end{macro}^^A \tud@math@greek@upvarphi
% \end{macro}^^A \tud@math@greek@upvarsigma
% \end{macro}^^A \tud@math@greek@upvarrho
% \end{macro}^^A \tud@math@greek@upvarpi
% \end{macro}^^A \tud@math@greek@upvartheta
% \end{macro}^^A \tud@math@greek@upvarepsilon
% \end{macro}^^A \tud@math@greek@upomega
% \end{macro}^^A \tud@math@greek@uppsi
% \end{macro}^^A \tud@math@greek@upchi
% \end{macro}^^A \tud@math@greek@upphi
% \end{macro}^^A \tud@math@greek@upupsilon
% \end{macro}^^A \tud@math@greek@uptau
% \end{macro}^^A \tud@math@greek@upsigma
% \end{macro}^^A \tud@math@greek@uprho
% \end{macro}^^A \tud@math@greek@uppi
% \end{macro}^^A \tud@math@greek@upxi
% \end{macro}^^A \tud@math@greek@upnu
% \end{macro}^^A \tud@math@greek@upmu
% \end{macro}^^A \tud@math@greek@uplambda
% \end{macro}^^A \tud@math@greek@upkappa
% \end{macro}^^A \tud@math@greek@upiota
% \end{macro}^^A \tud@math@greek@uptheta
% \end{macro}^^A \tud@math@greek@upeta
% \end{macro}^^A \tud@math@greek@upzeta
% \end{macro}^^A \tud@math@greek@upepsilon
% \end{macro}^^A \tud@math@greek@updelta
% \end{macro}^^A \tud@math@greek@upgamma
% \end{macro}^^A \tud@math@greek@upbeta
% \end{macro}^^A \tud@math@greek@upalpha
% \end{macro}^^A \tud@math@greek@declare@lc
% \begin{macro}{\tud@math@greek@check@slanted}
% \changes{v2.05d}{2016/09/22}{neu}^^A
% \begin{macro}{\if@tud@math@greek@slanted}
% \changes{v2.06}{2018/07/17}{neu}^^A
% Dieser Befehl prüft die aktuelle Definition von \cs{Gamma} und vergleicht 
% diese mit den kursiven Varianten des Buchstabens. Sind diese identisch, wird 
% davon ausgegangen, dass ein Paket mit der Option \opt{slantedGreek} oder 
% ähnlich geladen wurde. In diesem Fall wird der voreingestellte Wert für die 
% entsprechende \TUDScript-Option gesetzt.
%    \begin{macrocode}
\newif\if@tud@math@greek@slanted
\newcommand*\tud@math@greek@check@slanted{%
  \let\@tempb\relax%
  \let\@tempc\relax%
  \DeclareMathSymbol{\@tempb}{\mathalpha}{letters}{0}%
  \DeclareMathSymbol{\@tempc}{\mathord}{letters}{0}%
  \@tud@math@greek@slantedfalse%
  \ifx\Gamma\@tempb\relax\@tud@math@greek@slantedtrue\fi%
  \ifx\Gamma\@tempc\relax\@tud@math@greek@slantedtrue\fi%
  \if@tud@math@greek@slanted%
    \TUD@key@preset{slantedgreek}{true}%
  \fi%
  \let\tud@math@greek@check@slanted\relax%
}
%    \end{macrocode}
% \end{macro}^^A \if@tud@math@greek@slanted
% \end{macro}^^A \tud@math@greek@check@slanted
% \begin{macro}{\tud@math@greek@init}
% \changes{v2.04}{2015/03/10}{neu}^^A
% \begin{macro}{\upDelta}
% \begin{macro}{\upGamma}
% \begin{macro}{\upLambda}
% \begin{macro}{\upOmega}
% \begin{macro}{\upPhi}
% \begin{macro}{\upPi}
% \begin{macro}{\upPsi}
% \begin{macro}{\upSigma}
% \begin{macro}{\upTheta}
% \begin{macro}{\upUpsilon}
% \begin{macro}{\upXi}
% \begin{macro}{\upalpha}
% \begin{macro}{\upbeta}
% \begin{macro}{\upgamma}
% \begin{macro}{\updelta}
% \begin{macro}{\upepsilon}
% \begin{macro}{\upzeta}
% \begin{macro}{\upeta}
% \begin{macro}{\uptheta}
% \begin{macro}{\upiota}
% \begin{macro}{\upkappa}
% \begin{macro}{\uplambda}
% \begin{macro}{\upmu}
% \begin{macro}{\upnu}
% \begin{macro}{\upxi}
% \begin{macro}{\uppi}
% \begin{macro}{\uprho}
% \begin{macro}{\upsigma}
% \begin{macro}{\uptau}
% \begin{macro}{\upupsilon}
% \begin{macro}{\upphi}
% \begin{macro}{\upchi}
% \begin{macro}{\uppsi}
% \begin{macro}{\upomega}
% \begin{macro}{\upvarepsilon}
% \begin{macro}{\upvartheta}
% \begin{macro}{\upvarpi}
% \begin{macro}{\upvarrho}
% \begin{macro}{\upvarsigma}
% \begin{macro}{\upvarphi}
% \begin{macro}{\itDelta}
% \begin{macro}{\itGamma}
% \begin{macro}{\itLambda}
% \begin{macro}{\itOmega}
% \begin{macro}{\itPhi}
% \begin{macro}{\itPi}
% \begin{macro}{\itPsi}
% \begin{macro}{\itSigma}
% \begin{macro}{\itTheta}
% \begin{macro}{\itUpsilon}
% \begin{macro}{\itXi}
% \begin{macro}{\italpha}
% \begin{macro}{\itbeta}
% \begin{macro}{\itgamma}
% \begin{macro}{\itdelta}
% \begin{macro}{\itepsilon}
% \begin{macro}{\itzeta}
% \begin{macro}{\iteta}
% \begin{macro}{\ittheta}
% \begin{macro}{\itiota}
% \begin{macro}{\itkappa}
% \begin{macro}{\itlambda}
% \begin{macro}{\itmu}
% \begin{macro}{\itnu}
% \begin{macro}{\itxi}
% \begin{macro}{\itpi}
% \begin{macro}{\itrho}
% \begin{macro}{\itsigma}
% \begin{macro}{\ittau}
% \begin{macro}{\itupsilon}
% \begin{macro}{\itphi}
% \begin{macro}{\itchi}
% \begin{macro}{\itpsi}
% \begin{macro}{\itomega}
% \begin{macro}{\itvarepsilon}
% \begin{macro}{\itvartheta}
% \begin{macro}{\itvarpi}
% \begin{macro}{\itvarrho}
% \begin{macro}{\itvarsigma}
% \begin{macro}{\itvarphi}
% \begin{macro}{\otherDelta}
% \begin{macro}{\otherGamma}
% \begin{macro}{\otherLambda}
% \begin{macro}{\otherOmega}
% \begin{macro}{\otherPhi}
% \begin{macro}{\otherPi}
% \begin{macro}{\otherPsi}
% \begin{macro}{\otherSigma}
% \begin{macro}{\otherTheta}
% \begin{macro}{\otherUpsilon}
% \begin{macro}{\otherXi}
% \begin{macro}{\otheralpha}
% \begin{macro}{\otherbeta}
% \begin{macro}{\othergamma}
% \begin{macro}{\otherdelta}
% \begin{macro}{\otherepsilon}
% \begin{macro}{\otherzeta}
% \begin{macro}{\othereta}
% \begin{macro}{\othertheta}
% \begin{macro}{\otheriota}
% \begin{macro}{\otherkappa}
% \begin{macro}{\otherlambda}
% \begin{macro}{\othermu}
% \begin{macro}{\othernu}
% \begin{macro}{\otherxi}
% \begin{macro}{\otherpi}
% \begin{macro}{\otherrho}
% \begin{macro}{\othersigma}
% \begin{macro}{\othertau}
% \begin{macro}{\otherupsilon}
% \begin{macro}{\otherphi}
% \begin{macro}{\otherchi}
% \begin{macro}{\otherpsi}
% \begin{macro}{\otheromega}
% \begin{macro}{\othervarepsilon}
% \begin{macro}{\othervartheta}
% \begin{macro}{\othervarpi}
% \begin{macro}{\othervarrho}
% \begin{macro}{\othervarsigma}
% \begin{macro}{\othervarphi}
% Dieses Makro sorgt dafür, dass neben den normalen Makros für die griechischen 
% Majuskeln die zusätzlichen Befehle für das explizite Auswählen aufrechter und 
% kursiver Buchstaben bereitstehen.
%    \begin{macrocode}
\newcommand*\tud@math@greek@init[1]{%
%    \end{macrocode}
% Zunächst wird nach möglichen Aliasbefehlen gesucht, um \cs{up\dots} für 
% griechische Glyphen zu definieren.
%    \begin{macrocode}
  \ifcsundef{up#1}{%
    \@for\tud@res@a:={#1up}\do{%
      \ifcsundef{\tud@res@a}{}{\csletcs{up#1}{\tud@res@a}}%
    }%
  }{}%
%    \end{macrocode}
% Wurde kein Aliasbefehl gefunden, werden entweder die normalen oder ggf. die 
% mit \cs{tud@math@greek@declare@\dots} Glyphen verwendet.
%    \begin{macrocode}
  \ifcsundef{up#1}{%
    \if@tud@math@greek@slanted%
      \csletcs{up#1}{tud@math@greek@up#1}%
    \else%
      \csletcs{up#1}{#1}%
    \fi%
  }{}%
%    \end{macrocode}
% Suchen nach möglichen Aliasbefehlen zum Definieren von \cs{it\dots} für 
% griechische Glyphen. 
%    \begin{macrocode}
  \ifcsundef{it#1}{%
    \@for\tud@res@a:={#1sl,sl#1,#1it}\do{%
      \ifcsundef{\tud@res@a}{}{\csletcs{it#1}{\tud@res@a}}%
    }%
  }{}%
%    \end{macrocode}
% Wurde kein Aliasbefehl gefunden, werden entweder die normalen oder ggf. die 
% mit \cs{tud@math@greek@declare@\dots} Glyphen verwendet.
%    \begin{macrocode}
  \ifcsundef{it#1}{%
    \if@tud@math@greek@slanted%
      \csletcs{it#1}{#1}%
    \else%
      \csletcs{it#1}{tud@math@greek@it#1}%
    \fi%
  }{}%
%    \end{macrocode}
% Zuletzt werden die invertierenden Makros definiert.
%    \begin{macrocode}
  \ifcsundef{other#1}{%
    \if@tud@math@greek@slanted%
      \csletcs{other#1}{up#1}%
    \else%
      \csletcs{other#1}{it#1}%
    \fi%
  }{}%
}
%    \end{macrocode}
% \end{macro}^^A \othervarphi
% \end{macro}^^A \othervarsigma
% \end{macro}^^A \othervarrho
% \end{macro}^^A \othervarpi
% \end{macro}^^A \othervartheta
% \end{macro}^^A \othervarepsilon
% \end{macro}^^A \otheromega
% \end{macro}^^A \otherpsi
% \end{macro}^^A \otherchi
% \end{macro}^^A \otherphi
% \end{macro}^^A \otherupsilon
% \end{macro}^^A \othertau
% \end{macro}^^A \othersigma
% \end{macro}^^A \otherrho
% \end{macro}^^A \otherpi
% \end{macro}^^A \otherxi
% \end{macro}^^A \othernu
% \end{macro}^^A \othermu
% \end{macro}^^A \otherlambda
% \end{macro}^^A \otherkappa
% \end{macro}^^A \otheriota
% \end{macro}^^A \othertheta
% \end{macro}^^A \othereta
% \end{macro}^^A \otherzeta
% \end{macro}^^A \otherepsilon
% \end{macro}^^A \otherdelta
% \end{macro}^^A \othergamma
% \end{macro}^^A \otherbeta
% \end{macro}^^A \otheralpha
% \end{macro}^^A \otherXi
% \end{macro}^^A \otherUpsilon
% \end{macro}^^A \otherTheta
% \end{macro}^^A \otherSigma
% \end{macro}^^A \otherPsi
% \end{macro}^^A \otherPi
% \end{macro}^^A \otherPhi
% \end{macro}^^A \otherOmega
% \end{macro}^^A \otherLambda
% \end{macro}^^A \otherGamma
% \end{macro}^^A \otherDelta
% \end{macro}^^A \itvarphi
% \end{macro}^^A \itvarsigma
% \end{macro}^^A \itvarrho
% \end{macro}^^A \itvarpi
% \end{macro}^^A \itvartheta
% \end{macro}^^A \itvarepsilon
% \end{macro}^^A \itomega
% \end{macro}^^A \itpsi
% \end{macro}^^A \itchi
% \end{macro}^^A \itphi
% \end{macro}^^A \itupsilon
% \end{macro}^^A \ittau
% \end{macro}^^A \itsigma
% \end{macro}^^A \itrho
% \end{macro}^^A \itpi
% \end{macro}^^A \itxi
% \end{macro}^^A \itnu
% \end{macro}^^A \itmu
% \end{macro}^^A \itlambda
% \end{macro}^^A \itkappa
% \end{macro}^^A \itiota
% \end{macro}^^A \ittheta
% \end{macro}^^A \iteta
% \end{macro}^^A \itzeta
% \end{macro}^^A \itepsilon
% \end{macro}^^A \itdelta
% \end{macro}^^A \itgamma
% \end{macro}^^A \itbeta
% \end{macro}^^A \italpha
% \end{macro}^^A \itXi
% \end{macro}^^A \itUpsilon
% \end{macro}^^A \itTheta
% \end{macro}^^A \itSigma
% \end{macro}^^A \itPsi
% \end{macro}^^A \itPi
% \end{macro}^^A \itPhi
% \end{macro}^^A \itOmega
% \end{macro}^^A \itLambda
% \end{macro}^^A \itGamma
% \end{macro}^^A \itDelta
% \end{macro}^^A \upvarphi
% \end{macro}^^A \upvarsigma
% \end{macro}^^A \upvarrho
% \end{macro}^^A \upvarpi
% \end{macro}^^A \upvartheta
% \end{macro}^^A \upvarepsilon
% \end{macro}^^A \upomega
% \end{macro}^^A \uppsi
% \end{macro}^^A \upchi
% \end{macro}^^A \upphi
% \end{macro}^^A \upupsilon
% \end{macro}^^A \uptau
% \end{macro}^^A \upsigma
% \end{macro}^^A \uprho
% \end{macro}^^A \uppi
% \end{macro}^^A \upxi
% \end{macro}^^A \upnu
% \end{macro}^^A \upmu
% \end{macro}^^A \uplambda
% \end{macro}^^A \upkappa
% \end{macro}^^A \upiota
% \end{macro}^^A \uptheta
% \end{macro}^^A \upeta
% \end{macro}^^A \upzeta
% \end{macro}^^A \upepsilon
% \end{macro}^^A \updelta
% \end{macro}^^A \upgamma
% \end{macro}^^A \upbeta
% \end{macro}^^A \upalpha
% \end{macro}^^A \upXi
% \end{macro}^^A \upUpsilon
% \end{macro}^^A \upTheta
% \end{macro}^^A \upSigma
% \end{macro}^^A \upPsi
% \end{macro}^^A \upPi
% \end{macro}^^A \upPhi
% \end{macro}^^A \upOmega
% \end{macro}^^A \upLambda
% \end{macro}^^A \upGamma
% \end{macro}^^A \upDelta
% \end{macro}^^A \tud@math@greek@init
% \begin{macro}{\tud@cdmath@init}
% \changes{v2.06}{2018/07/02}{neu}^^A
% \begin{macro}{\tud@cdmath@enc}
% \changes{v2.06}{2018/07/12}{neu}^^A
% \begin{macro}{\tud@cdmath@declare}
% \changes{v2.06}{2018/07/02}{neu}^^A
% \begin{macro}{\mv@tud@nm}
% \begin{macro}{\mv@tud@bm}
% \begin{macro}{\mv@tud@dm}
%    \begin{macrocode}
\newcommand*\mv@tud@nm{}
\newcommand*\mv@tud@bm{}
\newcommand*\mv@tud@dm{}
%    \end{macrocode}
% Mit \cs{tud@cdmath@declare} werden für den Mathematikmodus die Schriften im 
% \CD deklariert.
%    \begin{macrocode}
\newcommand*\tud@cdmath@declare[3]{%
  \DeclareMathVersion{tud@nm}%
  \DeclareMathVersion{tud@bm}%
%    \end{macrocode}
% Mit \cs{tud@cdfont@@set} werden die Schriftstärken gemäß den ausgewählten 
% Schriftoptionen aktiviert (\cs{tud@cdfont@md}, \cs{tud@cdfont@bf}, 
% \cs{tud@cdfont@ebf}), um diese auch für die Mathematikschriften zu nutzen.
%    \begin{macrocode}
  \tud@cdfont@@set%
  \tud@ttfont@@set%
  \SetSymbolFont{operators}{tud@nm}{OT1}{#2}{\tud@cdfont@md}{n}%
  \SetSymbolFont{operators}{tud@bm}{OT1}{#2}{\tud@cdfont@bf}{n}%
  \SetSymbolFont{letters}{tud@nm}{OML}{#2}{\tud@cdfont@md}{it}%
  \SetSymbolFont{letters}{tud@bm}{OML}{#2}{\tud@cdfont@bf}{it}%
%    \end{macrocode}
% Für die alten Schriften exisiteren keine aufrechten griechischen Minuskeln.
%    \begin{macrocode}
  \if@tud@cdoldfont@active%
    \SetSymbolFont{upletters}{tud@nm}{OML}{#2}{\tud@cdfont@md}{it}%
    \SetSymbolFont{upletters}{tud@bm}{OML}{#2}{\tud@cdfont@bf}{it}%
  \else%
    \SetSymbolFont{upletters}{tud@nm}{OML}{#2}{\tud@cdfont@md}{n}%
    \SetSymbolFont{upletters}{tud@bm}{OML}{#2}{\tud@cdfont@bf}{n}%
  \fi%
  \SetSymbolFont{symbols}{tud@nm}{OMS}{#2}{\tud@cdfont@md}{n}%
  \SetSymbolFont{symbols}{tud@bm}{OMS}{#2}{\tud@cdfont@bf}{n}%
  \SetSymbolFont{largesymbols}{tud@nm}{OMX}{#2}{\tud@cdfont@md}{n}%
  \SetSymbolFont{largesymbols}{tud@bm}{OMX}{#2}{\tud@cdfont@bf}{n}%
  \SetMathAlphabet{\mathnormal}{tud@nm}{OML}{#2}{\tud@cdfont@md}{it}%
  \SetMathAlphabet{\mathnormal}{tud@bm}{OML}{#2}{\tud@cdfont@bf}{it}%
  \SetMathAlphabet{\mathrm}{tud@nm}{#1}{#2}{\tud@cdfont@md}{n}%
  \SetMathAlphabet{\mathrm}{tud@bm}{#1}{#2}{\tud@cdfont@bf}{n}%
  \SetMathAlphabet{\mathsf}{tud@nm}{#1}{#2}{\tud@cdfont@md}{n}%
  \SetMathAlphabet{\mathsf}{tud@bm}{#1}{#2}{\tud@cdfont@bf}{n}%
  \SetMathAlphabet{\mathit}{tud@nm}{#1}{#2}{\tud@cdfont@md}{it}%
  \SetMathAlphabet{\mathit}{tud@bm}{#1}{#2}{\tud@cdfont@bf}{it}%
  \SetMathAlphabet{\mathbf}{tud@nm}{#1}{#2}{\tud@cdfont@bf}{n}%
  \SetMathAlphabet{\mathbf}{tud@bm}{#1}{#2}{\tud@cdfont@ebf}{n}%
  \SetMathAlphabet{\mathtt}{tud@nm}{#1}{#3}{\tud@ttfont@md}{n}%
  \SetMathAlphabet{\mathtt}{tud@bm}{#1}{#3}{\tud@ttfont@bf}{n}%
  \IfFileExists{ueus.fd}{%
    \SetMathAlphabet{\mathcal}{tud@nm}{U}{eus}{m}{n}%
    \SetMathAlphabet{\mathcal}{tud@bm}{U}{eus}{b}{n}%
  }{}%
  \ifundef{\mathfrak}{}{%
    \IfFileExists{ueuf.fd}{%
      \SetMathAlphabet{\mathfrak}{tud@nm}{U}{euf}{m}{n}%
      \SetMathAlphabet{\mathfrak}{tud@bm}{U}{euf}{b}{n}%
    }{}%
  }%
}
%    \end{macrocode}
% Bei der Initialisierung der Schriften wird für die alten Schriften nach
% \cs{tud@cdmath@declare} noch etwas korrigiert.
%    \begin{macrocode}
\newcommand*\tud@cdmath@enc{}%
\newcommand*\tud@cdmath@init{%
%    \end{macrocode}
% Abhängig von der verwendeten Kodierung wird selbige für die Mathematikschrift 
% gesetzt. Anschließend wird mit \cs{tud@cdmath@declare} alles initialisiert.
%    \begin{macrocode}
  \edef\tud@cdmath@enc{\encodingdefault}%
  \if@tud@x@fontspec@enabled\else%
    \ifstr{T1}{\tud@cdmath@enc}{}{\def\tud@cdmath@enc{OT1}}%
  \fi%
  \tud@cdmath@declare{\tud@cdmath@enc}{\tud@cdfont@fam@lf}{\tud@ttfont@fam}%
%    \end{macrocode}
% Für die alten Schriften sind noch Nacharbeiten notwendig. Im Fließtext werden 
% für die lateinischen Lettern die \Univers"=Schriften des \CDs verwendet.
% Darin enthalten sind die auch die griechischen Buchstaben aus dem Paket
% \pkg{cmbright}. Zusätzlich werden aus dem \pkg{iwona}-Paket die mathematische
% Symbole verwendet.
%    \begin{macrocode}
  \if@tud@cdoldfont@active%
    \SetSymbolFont{largesymbols}{tud@nm}{OMX}{iwona}{m}{n}%
    \SetSymbolFont{largesymbols}{tud@bm}{OMX}{iwona}{b}{n}%
%    \end{macrocode}
% Das Paket \pkg{cmbright} bringt zusätzlich noch Schriftschnitte für das Paket
% \pkg{amsfonts} mit. Diese werden gegebenenfalls geladen.
%    \begin{macrocode}
    \@ifpackageloaded{amsfonts}{%
      \DeclareFontFamily{U}{\tud@cdfont@fam@lf a}{}%
      \DeclareFontShape{U}{\tud@cdfont@fam@lf a}{m}{n}{<->cmbras10}{}%
      \SetSymbolFont{AMSa}{tud@nm}{U}{\tud@cdfont@fam@lf a}{m}{n}%
      \SetSymbolFont{AMSa}{tud@bm}{U}{\tud@cdfont@fam@lf a}{m}{n}%
      \DeclareFontFamily{U}{\tud@cdfont@fam@lf b}{}%
      \DeclareFontShape{U}{\tud@cdfont@fam@lf b}{m}{n}{<->cmbrbs10}{}%
      \SetSymbolFont{AMSb}{tud@nm}{U}{\tud@cdfont@fam@lf b}{m}{n}%
      \SetSymbolFont{AMSb}{tud@bm}{U}{\tud@cdfont@fam@lf b}{m}{n}%
    }{}%
%    \end{macrocode}
% Auch für die Schrift \DIN werden mathematische Glyphen bereitgestellt. Diese
% wurden aus dem \pkg{iwona}-Paket entnommen. Auch die verwendeten Symbole
% stammen aus diesem Paket.
%    \begin{macrocode}
    \DeclareMathVersion{tud@dm}%
    \SetSymbolFont{operators}{tud@dm}{OT1}{\tud@cdfont@db}{b}{n}%
    \SetSymbolFont{letters}{tud@dm}{OML}{\tud@cdfont@db}{b}{sl}%
    \SetSymbolFont{upletters}{tud@dm}{OML}{\tud@cdfont@db}{b}{sl}%
    \SetSymbolFont{symbols}{tud@dm}{OMS}{\tud@cdfont@db}{b}{n}%
    \SetSymbolFont{largesymbols}{tud@dm}{OMX}{iwona}{ebc}{n}%
    \SetMathAlphabet{\mathnormal}{tud@dm}{OML}{\tud@cdfont@db}{b}{sl}%
    \SetMathAlphabet{\mathrm}{tud@dm}{\tud@cdmath@enc}{\tud@cdfont@db}{b}{n}%
    \SetMathAlphabet{\mathsf}{tud@dm}{\tud@cdmath@enc}{\tud@cdfont@db}{b}{n}%
    \SetMathAlphabet{\mathit}{tud@dm}{\tud@cdmath@enc}{\tud@cdfont@db}{b}{sl}%
    \SetMathAlphabet{\mathbf}{tud@dm}{\tud@cdmath@enc}{\tud@cdfont@db}{b}{n}%
    \SetMathAlphabet{\mathtt}{tud@dm}{\tud@cdmath@enc}{\tud@ttfont@fam}{b}{n}%
    \IfFileExists{ueus.fd}{%
      \SetMathAlphabet{\mathcal}{tud@dm}{U}{eus}{b}{n}%
    }{}%
    \ifundef{\mathfrak}{}{%
      \IfFileExists{ueuf.fd}{%
        \SetMathAlphabet{\mathfrak}{tud@dm}{U}{euf}{b}{n}%
      }{}%
    }%
%    \end{macrocode}
% Daran anschließend werden die \DIN-Schriften in einer Box verwendet, um das
% Laden der Schriftdefinitionen am Dokumentbeginn zu forcieren.
%    \begin{macrocode}
    \AtBeginDocument{\sbox\z@{\tud@db$ $}}%
  \fi%
}
%    \end{macrocode}
% \end{macro}^^A \mv@tud@dm
% \end{macro}^^A \mv@tud@bm
% \end{macro}^^A \mv@tud@nm
% \end{macro}^^A \tud@cdmath@declare
% \end{macro}^^A \tud@cdmath@enc
% \end{macro}^^A \tud@cdmath@init
% \begin{macro}{\tud@math@set}
% Mit diesem Befehl werden die Einstellungen für den Mathematiksatz übernommen.
% Die Option \opt{cdmath} wird ausgewertet, um zwischen dem Standard"~ und
% neuem Mathematiksatz hin"~ und herschalten zu können. Dabei wird zum einen
% das Mapping der griechischen Glyphen geändert, zum anderen werden die
% originalen Mathematikversionen überschrieben respektive wiederhergestellt.
%    \begin{macrocode}
\newcommand*\tud@math@set{%
%    \end{macrocode}
% Wird das Paket \pkg{fontspec} verwendet, sollte die Option \opt{no-math} 
% übergeben werden, falls die Mathematikschriften des \CDs zum Einsatz kommen. 
% Wurde dieser nicht angegeben, wird eine einmalige Warnung erzeugt.
%    \begin{macrocode}
  \if@tud@cdmath@active\if@tud@x@fontspec@math@enabled%
%<*class>
    \ClassWarningNoLine{\TUD@Class@Name}%
%</class>
%<*package>
    \PackageWarningNoLine{tudscrfonts}%
%</package>
    {%
      It's recommended to load package `fontspec'\MessageBreak%
      with option `no-math', if you want to use the\MessageBreak%
      corporate design fonts in math mode%
    }%
    \@tud@x@fontspec@math@enabledfalse%
  \fi\fi%
  \if@tud@cdmath@active%
    \tud@cs@store{mv@normal}%
    \tud@cs@store{mv@bold}%
    \let\mv@normal\mv@tud@nm%
    \let\mv@bold\mv@tud@bm%
    \tud@math@greek@loop\tud@cdmath@greek@shape@set%
    \tud@math@greek@loop@uc\tud@cdmath@greek@var@set%
    \ifcase\tud@slantedgreek@num\relax% upright
      \tud@math@greek@loop\tud@math@greek@upright%
    \or% slanted
      \tud@math@greek@loop\tud@math@greek@slanted%
    \or% standard
      \tud@math@greek@loop@uc\tud@math@greek@upright%
      \tud@math@greek@loop@lc\tud@math@greek@slanted%
    \fi%
  \else%
    \tud@cs@restore{mv@normal}%
    \tud@cs@restore{mv@bold}%
    \tud@math@greek@loop\tud@cdmath@greek@shape@reset%
    \tud@math@greek@loop@uc\tud@cdmath@greek@var@reset%
  \fi%
  \mathversion{normal}%
}
%    \end{macrocode}
% \end{macro}^^A \tud@math@set
% \begin{macro}{\tud@cdmath@greek@shape@set}
% \changes{v2.04}{2015/03/10}{neu}^^A
% \begin{macro}{\tud@cdmath@greek@shape@reset}
% \changes{v2.04}{2015/03/10}{neu}^^A
% Mit diesen beiden Befehlen kann innerhalb des Dokumentes problemlos zwischen 
% griechischen Buchstaben für die CD-Schriften und den normalen gewechselt 
% werden.
%    \begin{macrocode}
\newcommand*\tud@cdmath@greek@shape@set[1]{%
  \tud@cs@store{#1}%
  \tud@cs@store{up#1}%
  \tud@cs@store{it#1}%
  \tud@cs@store{other#1}%
  \csletcs{up#1}{tud@math@greek@up#1}%
  \csletcs{it#1}{tud@math@greek@it#1}%
  \tud@math@greek@comp@set{#1up}{up#1}%
  \tud@math@greek@comp@set{#1it}{it#1}%
  \tud@math@greek@comp@set{sl#1}{it#1}%
  \tud@math@greek@comp@set{#1sl}{it#1}%
}
\newcommand*\tud@cdmath@greek@shape@reset[1]{%
  \tud@cs@restore{#1}%
  \tud@cs@restore{up#1}%
  \tud@cs@restore{it#1}%
  \tud@cs@restore{other#1}%
  \tud@math@greek@comp@reset{#1up}%
  \tud@math@greek@comp@reset{#1it}%
  \tud@math@greek@comp@reset{sl#1}%
  \tud@math@greek@comp@reset{#1sl}%
}
%    \end{macrocode}
% \end{macro}^^A \tud@cdmath@greek@shape@reset
% \end{macro}^^A \tud@cdmath@greek@shape@set
% \begin{macro}{\tud@math@greek@comp@set}
% \changes{v2.06}{2018/07/02}{neu}^^A
% \begin{macro}{\tud@math@greek@comp@reset}
% \changes{v2.06}{2018/07/02}{neu}^^A
% Es gibt eine Vielzahl von Paketen, die das Setzen von aufrechten griechischen 
% Glyphen adressieren. Allerdings gibt es keinen einheitlichen Standard, was 
% die Nomenklatur der Auswahlbefehle betrifft. Hiermit können Aliasbefehle 
% überschireben werden, wobei dies nur geschieht, wenn ein Paket diese zuvor
% auch definert hat.
%    \begin{macrocode}
\newcommand*\tud@math@greek@comp@set[2]{%
  \ifcsdef{#1}{\tud@cs@store{#1}\csletcs{#1}{#2}}{}%
}
\newcommand*\tud@math@greek@comp@reset[1]{%
  \ifcsdef{#1}{\tud@cs@restore{#1}}{}%
}
%    \end{macrocode}
% \end{macro}^^A \tud@math@greek@comp@reset
% \end{macro}^^A \tud@math@greek@comp@set
% \begin{macro}{\tud@math@greek@var@set}
% \changes{v2.06}{2018/07/02}{neu}^^A
% \begin{macro}{\tud@math@greek@var@reset}
% \changes{v2.06}{2018/07/02}{neu}^^A
% Einige Pakete stellen für griechische Majuskeln die Befehle \cs{var\dots} 
% bereit, um diese im kursiven Schnitt zu setzen. Sollte eines dieser Pakete 
% geladen worden sein und ein entsprechender Befehl existieren, dann wird 
% dieser umdefiniert.
%    \begin{macrocode}
\newcommand*\tud@cdmath@greek@var@set[1]{%
  \tud@math@greek@comp@set{var#1}{it#1}%
}
\newcommand*\tud@cdmath@greek@var@reset[1]{%
  \tud@math@greek@comp@reset{var#1}%
}
%    \end{macrocode}
% \end{macro}^^A \tud@math@greek@var@reset
% \end{macro}^^A \tud@math@greek@var@set
% \begin{macro}{\tud@math@greek@upright}
% \changes{v2.06}{2018/07/02}{neu}^^A
% \begin{macro}{\tud@math@greek@slanted}
% \changes{v2.06}{2018/07/02}{neu}^^A
% Mit den beiden Befehlen kann die Definition der griechischen Buchstaben 
% entweder auf aufrecht bzw. kursiv gesetzt werden.
%    \begin{macrocode}
\newcommand*\tud@math@greek@upright[1]{%
  \csletcs{#1}{up#1}%
  \csletcs{other#1}{it#1}%
}
\newcommand*\tud@math@greek@slanted[1]{%
  \csletcs{#1}{it#1}%
  \csletcs{other#1}{up#1}%
}
%    \end{macrocode}
% \end{macro}^^A \tud@math@greek@slanted
% \end{macro}^^A \tud@math@greek@upright
% Samtliche Schrifteinstellungen erfolgen erst am Ende der Präambel, um etwaige
% Einstellungen irgendwelcher Schriftpakete beachten zu können. Dabei erfolgt 
% das Setzen der Mathematikschriften durch \cs{AtBeginDocument} innerhalb von 
% \cs{AtEndPreamble} am Ende \emph{aller} \cs{AtBeginDocument}.
%    \begin{macrocode}
\AtEndPreamble{%
%    \end{macrocode}
% Falls das Paket \pkg{mweights} geladen wurde, werden im Bedarfsfall die vom
% Paket erwarteten Makros für die Schriftstärken der einzelnen Schriftfamilien 
% definiert. Beim Aufruf von \cs{tud@font@set} werden diese anschließend 
% gesichert.
%    \begin{macrocode}
  \if@tud@x@mweights@enabled%
    \ifundef{\bfseries@rm}{\edef\bfseries@rm{\bfdefault}}{}%
    \ifundef{\mdseries@rm}{\edef\mdseries@rm{\mddefault}}{}%
    \ifundef{\bfseries@sf}{\edef\bfseries@sf{\bfdefault}}{}%
    \ifundef{\mdseries@sf}{\edef\mdseries@sf{\mddefault}}{}%
    \ifundef{\bfseries@tt}{\edef\bfseries@tt{\bfdefault}}{}%
    \ifundef{\mdseries@tt}{\edef\mdseries@tt{\mddefault}}{}%
  \fi%
%    \end{macrocode}
% Hier wird auf das Vorhandensein aller benötigten Schriften geprüft.
%    \begin{macrocode}
  \if@tud@cdoldfont@active%
    \tud@cdfont@check[Univers]{%
      Light,,Bold,Black,%
      LightOblique,Oblique,BoldOblique,BlackOblique%
    }{\tud@cdfont@fam}{@tud@cdfont@fam@exist}%
    \tud@cdfont@check{DIN-Bold}{\tud@cdfont@db}{@tud@cdfont@db@exist}%
  \else%
    \tud@cdfont@check[OpenSans]{%
      Light,Regular,SemiBold,Bold,ExtraBold,%
      LightItalic,Italic,SemiBoldItalic,BoldItalic,ExtraBoldItalic%
    }{\tud@cdfont@fam@lf,\tud@cdfont@fam@osf}{@tud@cdfont@fam@exist}%
  \fi%
  \@tud@res@swafalse%
  \IfFileExists{oml\tud@cdfont@fam@lf.fd}{}{\@tud@res@swatrue}%
  \IfFileExists{oms\tud@cdfont@fam@lf.fd}{}{\@tud@res@swatrue}%
  \if@tud@res@swa%
    \tud@locked@bool@preset{@tud@cdmath@active}{false}%
  \fi%
%    \end{macrocode}
% Damit \pkg{tudscrfonts} reibungslos mit den nachfolgenden Definitionen
% funktioniert, wird im Bedarfsfall der Befehl \cs{sectfont} vorgehalten. 
%    \begin{macrocode}
%<package>  \ifdef{\sectfont}{}{\let\sectfont\relax}%
%    \end{macrocode}
% Nachdem auf die Schriften geprüft wurde, werden noch die Schriftschalter und 
% die dazugehörigen Befehle definiert.
%    \begin{macrocode}
  \tud@cdfont@init%
  \tud@ttfont@init%
%    \end{macrocode}
% Wurde das Paket \pkg{siunitx} geladen, so werden die Schrifteinstellungen mit 
% \opt{cdfont}|=|\val{false} deaktiviert und zu Beginn des Dokumentes auf die 
% aktuelle Einstellung gesetzt.
%    \begin{macrocode}
  \if@tud@x@siunitx@enabled%
    \if@tud@cdfont@active%
      \if@tud@cdfont@heavy%
        \def\tud@reserved{\TUDoptions{cdfont=heavy}}%
      \else%
        \def\tud@reserved{\TUDoptions{cdfont=true}}%
      \fi%
    \else%
      \def\tud@reserved{\TUDoptions{cdfont=false}}%
    \fi%
    \expandafter\AtBeginDocument\expandafter{\tud@reserved}%
    \TUDoptions{cdfont=false}%
  \fi%
%    \end{macrocode}
% Damit die Schrift im Dokument später noch umgestellt werden kann, werden
% sämtliche Einstellungen erst am Ende der Präambel gesichert. Deshalb wird
% der dafür verantwortliche Befehl \cs{tud@font@set} hier zum ersten Mal
% verwendet. Damit ist es möglich, die Schriften eventuell geladener Pakete wie
% \pkg{lmodern} oder \pkg{libertine} vorher zu sichern und später wieder zu
% aktivieren. 
%    \begin{macrocode}
  \tud@font@set%
%    \end{macrocode}
% Die Einstellungen für den Mathematiksatz erfolgen erst zu Dokumentbeginn, um
% insbesondere auf Einstellungen von \pkg{fontspec} reagieren zu können. Da es
% so gut wie keine freien OpenType"=Schriften für den mathmatischen Formelsatz
% gibt, werden \emph{immer} die PostScript"=Schriften verwendet, falls diese
% installiert sind. Wurde jedoch das Paket \pkg{bm} angefordert, so müssen die
% Mathematikschriften spätestens zum Ende der Präambel initialisiert werden.
%    \begin{macrocode}
  \if@tud@x@bm@requested%
    \tud@math@init%
    \UnPreventPackageFromLoading{bm}%
    \RequirePackage{bm}[2017/01/16]%
  \fi%
  \AtBeginDocument{\tud@math@init\let\tud@math@init\relax}%
}
%    \end{macrocode}
% Wurde ein Paket geladen, welches die \pkg{amsfonts}-Schriftfamilie definiert,
% so wird eine Information ausgegeben, dass im Zweifel das Paket \pkg{mdsymbol} 
% besser zu den Schriften des \CDs passt.
%    \begin{macrocode}
\AtEndPreamble{%
  \if@tud@cdoldfont@active\else%
    \@ifpackageloaded{mdsymbol}{}{%
      \@tud@res@swafalse%
      \ifundef{\symAMSa}{}{\@tud@res@swatrue}%
      \ifundef{\symAMSb}{}{\@tud@res@swatrue}%
      \if@tud@res@swa%
%<*class>
        \ClassInfoNoLine{\TUD@Class@Name}%
%</class>
%<*package>
        \PackageInfoNoLine{tudscrfonts}%
%</package>
        {%
          You should load package `mdsymbol' in order to\MessageBreak%
          get symbols matching the used math font%
        }%
      \fi%
    }%
  \fi%
}
%    \end{macrocode}
%
% \subsubsection{Schrifteinstellungen für Überschriften}
%
% \begin{macro}{\ifdin}
% \changes{v2.02}{2014/08/29}{Anpassungen für \pkg{fontspec}}^^A
% \changes{v2.06}{2018/07/10}{für \OpenSans hinfällig}^^A
% Der Befehl \cs{ifdin} ist für \OpenSans hinfällig.
%    \begin{macrocode}
\TUD@deprecated@cs{ifdin}{@secondoftwo}
%    \end{macrocode}
% Für die alten SChriften prüft er auf die Verwendung von \DIN. Davon abhängig 
% wird entweder das erste oder das zweite Argument ausgeführt. Dies kann für die
% Befehle aller Gliederungsebenen genutzt werden, um zwischen der Ausgabe im
% Dokument sowie Inhaltsverzeichnis und/oder Kolumnentitel zu unterscheiden.
%    \begin{macrocode}
\if@tud@cdoldfont@active
  \newrobustcmd*\ifdin{%
    \ifx\f@family\tud@cdfont@db%
      \expandafter\@firstoftwo%
    \else%
      \expandafter\@secondoftwo%
    \fi%
  }%
\fi
%    \end{macrocode}
% \end{macro}^^A \ifdin
% \begin{macro}{\tud@sec@fontface}
% \changes{v2.05}{2016/04/04}{Optionales Argument für \pkg{tudscrcomp}}^^A
% Mit \cs{tud@sec@fontface} wird die Schriftart für Überschriften im 
% Layout des \CDs definiert.
% \ToDo{gehört nicht in \pkg{tudscrfonts}, nur in Klassen oder fixfonts}[v2.06]
%    \begin{macrocode}
\newcommand*\tud@sec@fontface[1][]{\sectfont#1}
%    \end{macrocode}
% Mit \cs{tud@sec@fontface} wird die Schriftart für Überschriften im 
% Layout des \CDs definiert. Normalerweise wird hierfür \DIN verwendet, was
% allerdings per Option auch deaktiviert werden kann.
%    \begin{macrocode}
\if@tud@cdoldfont@active
  \renewcommand*\tud@sec@fontface[1][]{%
    \if@tud@cdoldfont@nodin%
      \sectfont#1%
    \else%
      \tud@db\tud@raggedright%
    \fi%
  }%
\fi
%    \end{macrocode}
% \end{macro}^^A \tud@sec@fontface
% \begin{macro}{\tud@makeuppercase}
% \begin{macro}{\tud@sec@strut}
% \changes{v2.06}{2018/07/24}{neu}^^A
% Für Abschnitte etc. liegt auf einer neuen Seite dier erste Grundlinie höher,
% als für Großbuchstaben in der größeren Schrift für \cs{section} etc. benötigt 
% wird. Deshalb wird die Grundlinie \enquote{notfallmäßig} so weit nach unten
% verschoben, wie für den höchsten Buchstabe benötigt wird. Mit dem Befehl 
% \cs{tud@sec@strut} wird für alle Überschriften dafür gesorgt, dass immer der 
% gleiche Freiraum bereitgehalten wird, unabhängig vom Inhalt der spezifischen 
% Überschrift. Hierfür wird \cs{strutbox} aus dem \LaTeXe"~Kernel~-- nachdem 
% \cs{strut} zuvor bereits einmal ausgeführt wird, um die Standardgröße der Box 
% zu verwenden~-- temporär umdefiniert.
%    \begin{macrocode}
\newrobustcmd*\tud@sec@strut{%
  \begingroup%
    \strut\setbox\strutbox\hbox{\vphantom{\"A\"O\"U}}\strut%
  \endgroup%
}
%    \end{macrocode}
% Der Befehl führt \cs{tud@makeuppercase} wird für alle Überschriften zur 
% Formatierung genutzt. Für die \OpenSans führt er lediglich \cs{tud@sec@strut}
% aus.
% \ToDo{\cs{tud@makeuppercase} umbenennen}[v2.06]
%    \begin{macrocode}
\newcommand*\tud@makeuppercase[2][]{\tud@sec@strut#2}
%    \end{macrocode}
% Für die alten Schriften hat der Befehl nur eine Funktionalität, wenn die
% Schriftfamilie \DIN aktiv ist. Dies ist insbesondere bei Überschriften der 
% Fall, welche zwingend in Majuskeln\footnote{Großbuchstaben} zu setzen sind. 
% Dafür wird der Befehl \cs{MakeTextUppercase} aus dem Paket \pkg{textcase} 
% genutzt, welcher den internen \LaTeX-Befehl \cs{MakeUppercase} verbessert.
% Das optionale Argument kann genutzt werden, wenn nicht die aktuelle, sondern
% ein durch \KOMAScript{} definiertes Schriftelement ausgewertet werden soll, 
% um zu erkennen, ob \DIN zum Einsatz kommt.
%    \begin{macrocode}
\if@tud@cdoldfont@active
  \RequirePackage{textcase}[2004/10/07]
  \renewcommand*\tud@makeuppercase[2][]{%
    \@tud@res@swafalse%
    \IfArgIsEmpty{#1}{%
      \ifdin{\@tud@res@swatrue}{}%
    }{%
      \begingroup%
        \usekomafont{#1}%
        \ifdin{\aftergroup\@tud@res@swatrue}{}%
      \endgroup%
    }%
    \tud@sec@strut%
    \if@tud@res@swa%
      \begingroup\MakeTextUppercase{#2}\endgroup%
    \else%
      #2%
    \fi%
  }%
\fi
%    \end{macrocode}
% \end{macro}^^A \tud@sec@strut
% \end{macro}^^A \tud@makeuppercase
%
% \iffalse
%<*class>
% \fi
%
% \subsubsection{Schriften für die speziellen Seitenstile}
%
% \begin{macro}{\tud@head@font@set}
% \changes{v2.02}{2014/07/08}{Unterstützung für \opt{cdhead} hinzugefügt}^^A
% \changes{v2.03}{2015/01/30}{Bugfix für die Änderung von \cs{spaceskip} durch
%   das Paket \pkg{ragged2e}}^^A
% \begin{macro}{\tud@head@cdfont@set}%
% \changes{v2.06}{2018/06/29}{neu}^^A
% \begin{macro}{\tud@head@font@light}
% \begin{macro}{\tud@head@font@bold}
% \begin{macro}{\tud@head@font@@bold}
% \changes{v2.05}{2016/04/17}{neu}^^A
% Die Schrift der Kopfzeile wird entweder in den Schriften des \CDs oder aber 
% in den serifenlosen Standardschriften gesetzt, wobei für beide Varianten
% sowohl eine fettgedruckte als auch eine normale Version benötigt wird. Die
% Schrifthöhe ist durch das \CD unabhängig von der gewählten Schriftgröße im
% Dokument vorgegeben und wird in Abhängigkeit von der Papiergröße gewählt und
% in \cs{tud@head@fontsize} gespeichert. Das Makro \cs{tud@head@font@@bold} 
% wird verwendet, um die Einstellung der fetten Schriften ohne die fixierte 
% Schriftgröße nutzen zu können.
%    \begin{macrocode}
\newcommand*\tud@head@font@light{}
\newcommand*\tud@head@font@bold{}
%<poster>\newcommand*\tud@head@font@@bold{}
\newcommand*\tud@head@font@set{%
%    \end{macrocode}
% Die Option \opt{cdhead} wird ausgewertet. Sollte diese inkompatibel gesetzt 
% sein, wird eine Warnung ausgegeben.
%    \begin{macrocode}
  \if@tud@cdfont@active\ifcase\tud@head@font@num\relax% false
    \if@tud@head@font@set%
      \ClassWarning{\TUD@Class@Name}{%
        It isn't possible to use `cdhead=false'\MessageBreak%
        together with `cdfont=true'%
      }%
    \fi%
    \if@tud@cdfont@heavy%
      \TUDoptions{cdhead=heavy}%
    \else%
      \TUDoptions{cdhead=true}%
    \fi%
  \fi\fi%
%    \end{macrocode}
% Die Schriften für die Kopfzeile werden hier definiert. Abhängig von 
% \cs{if@tud@cdfont@active} werden entweder die Schriften des \CDs oder die 
% serifenlose Standardschrift für die Kopfzeile verwendet. Die Rückfallebene:
%    \begin{macrocode}
  \if@tud@head@font@set%
    \if@tud@x@mweights@enabled%
      \DeclareFixedFont{\tud@head@font@light}{\encodingdefault}%
        {\sfdefault}{\mdseries@sf}{\updefault}{\tud@head@fontsize}%
      \DeclareFixedFont{\tud@head@font@bold}{\encodingdefault}%
        {\sfdefault}{\bfseries@sf}{\updefault}{\tud@head@fontsize}%
%<*poster>
      \protected\gdef\tud@head@font@@bold{%
        \usefont{\encodingdefault}{\sfdefault}{\bfseries@sf}{\updefault}%
      }%
%</poster>
    \else%
      \DeclareFixedFont{\tud@head@font@light}{\encodingdefault}%
        {\sfdefault}{\mddefault}{\updefault}{\tud@head@fontsize}%
      \DeclareFixedFont{\tud@head@font@bold}{\encodingdefault}%
        {\sfdefault}{\bfdefault}{\updefault}{\tud@head@fontsize}%
%<*poster>
      \protected\gdef\tud@head@font@@bold{%
        \usefont{\encodingdefault}{\sfdefault}{\bfdefault}{\updefault}%
      }%
%</poster>
    \fi%
    \ifcase\tud@head@font@num\relax\else% true/heavy
      \tud@head@cdfont@set%
    \fi%
    \global\@tud@head@font@setfalse%
    \global\@tud@head@text@settrue%
  \fi%
}
%    \end{macrocode}
% Hiermit werden die eigentlichen Schriften des \CDs gesetzt.
%    \begin{macrocode}
\newcommand*\tud@head@cdfont@set{%
  \if@tud@cdfont@fam@exist%
    \ifcase\tud@head@font@num\relax\or% true
      \DeclareFixedFont{\tud@head@font@light}{\encodingdefault}%
        {\tud@cdfont@fam@lf}{l}{n}{\tud@head@fontsize}%
      \DeclareFixedFont{\tud@head@font@bold}{\encodingdefault}%
        {\tud@cdfont@fam@lf}{sb}{n}{\tud@head@fontsize}%
%<*poster>
      \protected\gdef\tud@head@font@@bold{%
        \usefont{\encodingdefault}{\tud@cdfont@fam@lf}{sb}{n}%
      }%
%</poster>
    \or% heavy
      \DeclareFixedFont{\tud@head@font@light}{\encodingdefault}%
        {\tud@cdfont@fam@lf}{m}{n}{\tud@head@fontsize}%
      \DeclareFixedFont{\tud@head@font@bold}{\encodingdefault}%
        {\tud@cdfont@fam@lf}{b}{n}{\tud@head@fontsize}%
%<*poster>
      \protected\gdef\tud@head@font@@bold{%
        \usefont{\encodingdefault}{\tud@cdfont@fam@lf}{b}{n}%
      }%
%</poster>
    \fi%
  \fi%
}
%    \end{macrocode}
% Und noch die Variante für die alten Schriften.
%    \begin{macrocode}
\if@tud@cdoldfont@active
  \renewcommand*\tud@head@cdfont@set{%
    \if@tud@cdfont@fam@exist%
      \ifcase\tud@head@font@num\relax\or% true
        \DeclareFixedFont{\tud@head@font@light}{\encodingdefault}%
          {\tud@cdfont@fam@lf}{l}{n}{\tud@head@fontsize}%
      \or% heavy
        \DeclareFixedFont{\tud@head@font@light}{\encodingdefault}%
          {\tud@cdfont@fam@lf}{m}{n}{\tud@head@fontsize}%
      \fi%
      \DeclareFixedFont{\tud@head@font@bold}{\encodingdefault}%
        {\tud@cdfont@fam@lf}{b}{n}{\tud@head@fontsize}%
%<*poster>
      \protected\gdef\tud@head@font@@bold{%
        \usefont{\encodingdefault}{\tud@cdfont@fam@lf}{b}{n}%
      }%
%</poster>
    \fi%
  }%
\fi
%    \end{macrocode}
% \end{macro}^^A \tud@head@font@@bold
% \end{macro}^^A \tud@head@font@bold
% \end{macro}^^A \tud@head@font@light
% \end{macro}^^A \tud@head@cdfont@set
% \end{macro}^^A \tud@head@font@set
%
% \iffalse
%</class>
%</body>
% \fi
%
% \Finale
%
\endinput
